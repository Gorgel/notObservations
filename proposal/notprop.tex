%%%%%%%%%%%%%%%%%%%%%%%%%%%%%%%%%%%%%%%%%%%%%%%%%%%%%%%%%%%%%%%%%%%%%%%%%%%%%
%%                                                                         %%
%%               NOT PROPOSAL TEMPLATE  FOR PERIOD 48                      %%
%%                                                                         %%
%%                 USE WITH STYLEFILE: notsty48.sty                        %%
%%                                                                         %%
%%%%%%%%%%%%%%%%%%%%%%%%%%%%%%%%%%%%%%%%%%%%%%%%%%%%%%%%%%%%%%%%%%%%%%%%%%%%%

\documentclass[11pt]{article}           % Do not change
\usepackage{fullpage,graphicx}          % Do not change
%%%%%%%%%%%%%%%%%%%%%%%%%%%%%%%%%%%%%%%%%%%%%%%%%%%%%%%%%%%%%%%%%%%%%%%%%%%
%%                                                                       %%
%%            NOT PROPOSAL TEMPLATE STYLEFILE FOR PERIOD 48              %%
%%                                                                       %%
%%  Note to proposers: Do not edit this file!                            %%
%%                                                                       %%
%%  Submitted proposals will be processed by not staff using the correct %%
%%  style file. If you change this style-file, the proposal processing   %%
%%  will fail, and the proposal will be rejected.                        %%
%%                                                                       %%
%%  Name of this file: notsty48.sty                                      %%
%%                                                                       %%
%%%%%%%%%%%%%%%%%%%%%%%%%%%%%%%%%%%%%%%%%%%%%%%%%%%%%%%%%%%%%%%%%%%%%%%%%%%
\usepackage{lscape}
\usepackage{hyperref,ifthen,calc}
\usepackage{graphicx}
\usepackage{epsf}
\usepackage[english]{babel}
\def\deg{\hbox{$^\circ$}}
\def\fdeg{\hbox{$.\!\!^\circ$}}
\def\arcmin{\hbox{$^\prime$}}
\def\arcsec{\hbox{$^{\prime\prime}$}}
\def\fmag{\hbox{$.\!\!^m$}}
\def\thePeriod{48}

\setlength{\textwidth}{165mm}    
\setlength{\textheight}{267mm}    
\setlength{\topmargin}{-25mm}    
\setlength{\oddsidemargin}{0mm}    
\setlength{\evensidemargin}{-30mm}    
\setlength{\parindent}{0mm}
%\fboxsep=2mm


\def\putnumberHuge{\begin{flushright} {\large Proposal \Huge \it
\thePeriod -\propnumber}\end{flushright}}

\def\putnumberLarge{\begin{flushright} {\large Proposal \Large \it
\thePeriod -\propnumber}\end{flushright}}



\def\nothead{\vspace{-8ex}
\begin{center}
{\Large \hspace{-0.41cm} NORDIC OPTICAL TELESCOPE}\\[0.5mm]
\hspace{-0.41cm} APPLICATION FOR OBSERVING TIME\\[0.5mm]
\hspace{-0.41cm} OBSERVING PERIOD \thePeriod \\[0.5mm]
\hspace{-0.41cm} October 1, 2013 - April 1, 2014\\
\end{center}
\vspace{-0ex}}


\def\tithead{
{\bf \small 1. Title of proposal: }\\}
\newcommand{\proptitle}[1]{{#1}}
\newsavebox{\titbox}
\newenvironment{titpage}[1]
 {\begin{lrbox}{\titbox}\begin{minipage}[t][6.5ex][s]{42em}\tithead\Large}
 {\end{minipage}\end{lrbox}\fbox{\usebox{\titbox}}}



\def\abshead{
{\bf \small 2. Abstract: }\\}
\newcommand{\propabstract}[1]{{#1}}
\newsavebox{\absbox}
\newenvironment{abspage}[4]
 {\begin{lrbox}{\absbox}\begin{minipage}[t][30.5ex][s]{42em}\abshead}
 {\end{minipage}\end{lrbox}\fbox{\usebox{\absbox}}}



\def\adrhead{{\bf \small 3. Principal Investigator: }
{\footnotesize ({\bf NB:} The P.I. has full responsibility for
the content of this proposal!)}\\[1mm]}
\def\piname{\makebox[8em][t]{Name: }}
\def\piinst{\\[1mm] \makebox[8em][t] {Institute: }}
\def\picoun{\hspace{-0.12cm}, \makebox[0em][t] {}}
\def\piaddr{\\ \makebox[8em][t]{Postal address:}}
\def\piteln{\\[1mm] \makebox[8em][t]{Telephone:}}
\def\pifaxn{\\ \makebox[8em][t]{Fax:}}
\def\pimail{\\ \makebox[8em][t]{E-mail:}}
\newsavebox{\adrinvbox}
\newenvironment{adrinvpage}[1]
 {\begin{lrbox}{\adrinvbox}\begin{minipage}[t][20.5ex][s]{42em}\adrhead}
 {\end{minipage}\end{lrbox}\fbox{\usebox{\adrinvbox}}}




\def\coinvhead{{\bf \small 4. Co-investigators and affiliations: }}
\def\coinvest{\\\makebox[20em][t] }
\newsavebox{\coinvestbox}
\newenvironment{coinvestpage}[1]
 {\begin{lrbox}{\coinvestbox}\begin{minipage}[t][16ex][s]{42em}\coinvhead}
 {\end{minipage}\end{lrbox}\fbox{\usebox{\coinvestbox}}}




\def\omthead{{\bf \small 5. If this is a PhD thesis project at a Nordic
institute, please give name of student, supervisor and expected
time of completion: \hspace{1cm}}}
\newcommand{\thesis}[1]{{#1}}
\newsavebox{\omthesisbox}
\newenvironment{omthesispage}[1]
 {\begin{lrbox}{\omthesisbox}\begin{minipage}[t][7ex][s]{42em}\omthead}
 {\end{minipage}\end{lrbox}\fbox{\usebox{\omthesisbox}}}



\def\nighead{{\bf \small 6. Observing period(s) requested and preferred
scheduling:} \\
\makebox[4em][t]{\underline{Run}} 
\makebox[9em][t]{\underline{Instrument}}
\makebox[5em][t]{\underline{Time}}
\makebox[11em][t]{\underline{Month(s)/Date(s)}}
\makebox[4em][t]{\underline{Moon}}
\makebox[4em][t]{\underline{Seeing}}
\makebox[3em][t]{\underline{Sky}}}
\def\nrunid{\\ \makebox[4em][t]} 
\def\ninstr{\makebox[9em][t]} 
\def\ntimer{\makebox[5em][t]} 
\def\nmonth{\makebox[11em][t]}
\def\nmoonp{\makebox[4em][t]}
\def\nsemax{\makebox[4em][t]}
\def\nskyco{\makebox[3em][t]}
\newsavebox{\nightsbox}
\newenvironment{nightspage}[1]
 {\begin{lrbox}{\nightsbox}\begin{minipage}[t][22ex][s]{42em}\nighead}
 {\end{minipage}\end{lrbox}\fbox{\usebox{\nightsbox}}}


\def\numalr{\makebox[25.5em][t] {\bf \small 7. Number of nights already
awarded to project:} }
\def\numrem{\\ \makebox[25.8em][t] {\bf \small 8. Number of nights needed
to complete project:}}
\newsavebox{\numnightsbox}
\newenvironment{numnightspage}[1]
 {\begin{lrbox}{\numnightsbox}\begin{minipage}[t][4.8ex][s]{42em}}
 {\end{minipage}\end{lrbox}\fbox{\usebox{\numnightsbox}}}


\def\servicehead{
{\bf \small 9. If your project can not be done in service mode, please justify: }\\}
\newcommand{\service}[1]{{#1}}
\newsavebox{\servicebox}
\newenvironment{servicepage}[4]
 {\begin{lrbox}{\servicebox}\begin{minipage}[t][10ex][s]{42em}\servicehead}
 {\end{minipage}\end{lrbox}\fbox{\usebox{\servicebox}}}


\def\schedhead{
{\bf \small 10. Any other special constraints on the scheduling? }\\}
\newcommand{\schedconstr}[1]{{#1}}
\newsavebox{\schedbox}
\newenvironment{schedpage}[4]
 {\begin{lrbox}{\schedbox}\begin{minipage}[t][10ex][s]{42em}\schedhead}
 {\end{minipage}\end{lrbox}\fbox{\usebox{\schedbox}}}


\def\scihead{{\bf \small 11. Scientific justification for the proposal:
}\\[2mm]  }
\newcommand{\scijust}[1]{{#1}}
\newsavebox{\scienbox}
\newenvironment{scienpage}[4]
 {\begin{lrbox}{\scienbox}\begin{minipage}[t][152ex][s]{42em}\scihead}
 {\end{minipage}\end{lrbox}\fbox{\usebox{\scienbox}}}



\def\sciheadc{{\bf \small 12.  Scientific justification $- continued -$}
\\[2mm]  }
\newcommand{\scijustc}[1]{{#1}}
\newcommand{\captone}[1]{\small {\bf Fig. 1.}{#1} \\ }
\newcommand{\capttwo}[1]{\small {\bf Fig. 2.}{#1} \\ }
%\def\captone{\small {\bf Fig. 1.} }
%\def\capttwo{\small {\bf Fig. 2.} }
\newsavebox{\scienboxc}
\newenvironment{scienpagec}[4]
 {\begin{lrbox}{\scienboxc}\begin{minipage}[t][152ex][s]{42em}\sciheadc}
 {\end{minipage}\end{lrbox}\fbox{\usebox{\scienboxc}}}



\def\jushead{{\bf \small 13. Technical description of the observations.
 } (Please provide a self-contained case.)\\[1.5mm]}
\newcommand{\techjust}[1]{{#1}}
\newsavebox{\justificbox}
\newenvironment{justificpage}[4]
 {\begin{lrbox}{\justificbox}\begin{minipage}[t][100ex][s]{42em}\jushead}
 {\end{minipage}\end{lrbox}\fbox{\usebox{\justificbox}}}



\def\inshead{{\bf \small 14. Requested instrument setup(s): }\\
{\small \makebox[4em][t] {\underline{Run}}}
{\small \makebox[9em][t] {\underline{Instrument}}}
{\small \makebox[12em][t] {\underline{Mode}}}
{\small \makebox[12em][t] {\underline{Setup}}}\\[1mm]}
\newcommand{\NOTconfig}[4]{{\small \makebox[4em][t]  {#1} \makebox[9em][t] {#2}
\makebox[12em][t] {#3} \makebox[12em][t] {#4} \\ } }
\newsavebox{\instrembox}
\newenvironment{instrempage}[4]
 {\begin{lrbox}{\instrembox}\begin{minipage}[t][52ex][s]{42em}\inshead}
 {\end{minipage}\end{lrbox}\fbox{\usebox{\instrembox}}}



\def\objhead{{\bf \small 15. Target list with coordinates, or intervals
in R.A. and Decl. of (sample of) objects: }\\[3mm]  
{\small \makebox[4.4em][t]{Run}}
{\small \makebox[7.5em][t]{Name}}
{\small \makebox[5.5em][t]{$\alpha_{2000}$}}
{\small \makebox[5.5em][t]{$\delta_{2000}$}}
{\small \makebox[5.5em][t]{Magnitude}}
{\small \makebox[5.5em][t]{Diam($\arcmin$)}}
{\small \makebox[5.5em][t]{Additional Info.}}\\[1mm]}
\newcommand{\target}[7]{{\small \makebox[4.4em][t]  {#1} \makebox[7.5em][t] {#2}
\makebox[5.5em][t] {#3} \makebox[5.6em][t] {#4} \makebox[5.5em][t] {#5}
\makebox[5.5em][t] {#6} \makebox[5.5em][t] {#7} \\ } }
\newsavebox{\objlistbox}
\newenvironment{objlistpage}[1]
 {\begin{lrbox}{\objlistbox}\begin{minipage}[t][68ex][s]{42em}\objhead}
 {\end{minipage}\end{lrbox}\fbox{\usebox{\objlistbox}}}



\def\remark{\makebox[3em][t]{Remarks:} \makebox[4em][t]{ } }



\def\quahead{{\bf \small 16. Backup programme (or justification why
none is needed). \\}{\footnotesize ({\bf NB:} The backup programme
should also take unfavourable wind conditions into account)}\\[2mm]}
\newcommand{\backup}[1]{{#1}}
\newsavebox{\qualitybox}
\newenvironment{qualitypage}[4]
 {\begin{lrbox}{\qualitybox}\begin{minipage}[t][84ex][s]{42em}\quahead}
 {\end{minipage}\end{lrbox}\fbox{\usebox{\qualitybox}}}



\def\prehead{{\bf \small 17. List of observing periods, and publications from
NOT observations, within the last three years.}\\[2mm]}
\newcommand{\prevobs}[1]{{#1}}
\newsavebox{\prevobsbox}
\newenvironment{prevobspage}[4]
 {\begin{lrbox}{\prevobsbox}\begin{minipage}[t][96ex][s]{42em}\prehead}
 {\end{minipage}\end{lrbox}\fbox{\usebox{\prevobsbox}}}




\def\exthead{{\bf \small 18. Additional remarks not covered by the items
above, if any: }\\[2mm]}
\newcommand{\addinfo}[1]{{#1}}
\newsavebox{\extrabox}
\newenvironment{extrapage}[4]
 {\begin{lrbox}{\extrabox}\begin{minipage}[t][18ex][s]{42em}\exthead}
 {\end{minipage}\end{lrbox}\fbox{\usebox{\extrabox}}}




%%%%%%%%%%%%%%%%%%%%%%%%%%%%%%%%% PSFIG %%%%%%%%%%%%%%%%%%%%%%%%%%%%%%%%%%%%%%

% Psfig/TeX
\def\PsfigVersion{1.9}
% dvips version
%
% All psfig/tex software, documentation, and related files
% in this distribution of psfig/tex are
% Copyright 1987, 1988, 1991 Trevor J. Darrell
%
% Permission is granted for use and non-profit distribution of psfig/tex
% providing that this notice is clearly maintained. The right to
% distribute any portion of psfig/tex for profit or as part of any commercial
% product is specifically reserved for the author(s) of that portion.
%
% *** Feel free to make local modifications of psfig as you wish,
% *** but DO NOT post any changed or modified versions of ``psfig''
% *** directly to the net. Send them to me and I'll try to incorporate
% *** them into future versions. If you want to take the psfig code
% *** and make a new program (subject to the copyright above), distribute it,
% *** (and maintain it) that's fine, just don't call it psfig.
%
% Bugs and improvements to trevor@media.mit.edu.
%
% Thanks to Greg Hager (GDH) and Ned Batchelder for their contributions
% to the original version of this project.
%
% Modified by J. Daniel Smith on 9 October 1990 to accept the
% %%BoundingBox: comment with or without a space after the colon.  Stole
% file reading code from Tom Rokicki's EPSF.TEX file (see below).
%
% More modifications by J. Daniel Smith on 29 March 1991 to allow the
% the included PostScript figure to be rotated.  The amount of
% rotation is specified by the "angle=" parameter of the \psfig command.
%
% Modified by Robert Russell on June 25, 1991 to allow users to specify
% .ps filenames which don't yet exist, provided they explicitly provide
% boundingbox information via the \psfig command. Note: This will only work
% if the "file=" parameter follows all four "bb???=" parameters in the
% command. This is due to the order in which psfig interprets these params.
%
%  3 Jul 1991JDScheck if file already read in once
%  4 Sep 1991JDSfixed incorrect computation of rotated
%bounding box
% 25 Sep 1991GVRexpanded synopsis of \psfig
% 14 Oct 1991JDS\fbox code from LaTeX so \psdraft works with TeX
%changed \typeout to \ps@typeout
% 17 Oct 1991JDSadded \psscalefirst and \psrotatefirst
%

% From: gvr@cs.brown.edu (George V. Reilly)
%
% \psdraftdraws an outline box, but doesn't include the figure
%in the DVI file.  Useful for previewing.
%
% \psfullincludes the figure in the DVI file (default).
%
% \psscalefirst width= or height= specifies the size of the figure
% before rotation.
% \psrotatefirst (default) width= or height= specifies the size of the
%  figure after rotation.  Asymetric figures will
%  appear to shrink.
%
% \psfigurepath#1sets the path to search for the figure
%
% \psfig
% usage: \psfig{file=, figure=, height=, width=,
%bbllx=, bblly=, bburx=, bbury=,
%rheight=, rwidth=, clip=, angle=, silent=}
%
%"file" is the filename.  If no path name is specified and the
%file is not found in the current directory,
%it will be looked for in directory \psfigurepath.
%"figure" is a synonym for "file".
%By default, the width and height of the figure are taken from
%the BoundingBox of the figure.
%If "width" is specified, the figure is scaled so that it has
%the specified width.  Its height changes proportionately.
%If "height" is specified, the figure is scaled so that it has
%the specified height.  Its width changes proportionately.
%If both "width" and "height" are specified, the figure is scaled
%anamorphically.
%"bbllx", "bblly", "bburx", and "bbury" control the PostScript
%BoundingBox.  If these four values are specified
%               *before* the "file" option, the PSFIG will not try to
%               open the PostScript file.
%"rheight" and "rwidth" are the reserved height and width
%of the figure, i.e., how big TeX actually thinks
%the figure is.  They default to "width" and "height".
%The "clip" option ensures that no portion of the figure will
%appear outside its BoundingBox.  "clip=" is a switch and
%takes no value, but the `=' must be present.
%The "angle" option specifies the angle of rotation (degrees, ccw).
%The "silent" option makes \psfig work silently.
%

% check to see if macros already loaded in (maybe some other file says
% "\input psfig") ...
\ifx\undefined\psfig\else\endinput\fi

%
% from a suggestion by eijkhout@csrd.uiuc.edu to allow
% loading as a style file. Changed to avoid problems
% with amstex per suggestion by jbence@math.ucla.edu

\let\LaTeXAtSign=\@
\let\@=\relax
\edef\psfigRestoreAt{\catcode`\@=\number\catcode`@\relax}
%\edef\psfigRestoreAt{\catcode`@=\number\catcode`@\relax}
\catcode`\@=11\relax
\newwrite\@unused
\def\ps@typeout#1{{\let\protect\string\immediate\write\@unused{#1}}}
\ps@typeout{psfig/tex \PsfigVersion}

%% Here's how you define your figure path.  Should be set up with null
%% default and a user useable definition.

\def\figurepath{./}
\def\psfigurepath#1{\edef\figurepath{#1}}

%
% @psdo control structure -- similar to Latex @for.
% I redefined these with different names so that psfig can
% be used with TeX as well as LaTeX, and so that it will not
% be vunerable to future changes in LaTeX's internal
% control structure,
%
\def\@nnil{\@nil}
\def\@empty{}
\def\@psdonoop#1\@@#2#3{}
\def\@psdo#1:=#2\do#3{\edef\@psdotmp{#2}\ifx\@psdotmp\@empty \else
    \expandafter\@psdoloop#2,\@nil,\@nil\@@#1{#3}\fi}
\def\@psdoloop#1,#2,#3\@@#4#5{\def#4{#1}\ifx #4\@nnil \else
       #5\def#4{#2}\ifx #4\@nnil \else#5\@ipsdoloop #3\@@#4{#5}\fi\fi}
\def\@ipsdoloop#1,#2\@@#3#4{\def#3{#1}\ifx #3\@nnil
       \let\@nextwhile=\@psdonoop \else
      #4\relax\let\@nextwhile=\@ipsdoloop\fi\@nextwhile#2\@@#3{#4}}
\def\@tpsdo#1:=#2\do#3{\xdef\@psdotmp{#2}\ifx\@psdotmp\@empty \else
    \@tpsdoloop#2\@nil\@nil\@@#1{#3}\fi}
\def\@tpsdoloop#1#2\@@#3#4{\def#3{#1}\ifx #3\@nnil
       \let\@nextwhile=\@psdonoop \else
      #4\relax\let\@nextwhile=\@tpsdoloop\fi\@nextwhile#2\@@#3{#4}}
%
% \fbox is defined in latex.tex; so if \fbox is undefined, assume that
% we are not in LaTeX.
% Perhaps this could be done better???
\ifx\undefined\fbox
% \fbox code from modified slightly from LaTeX
\newdimen\fboxrule
\newdimen\fboxsep
\newdimen\ps@tempdima
\newbox\ps@tempboxa
\fboxsep = 3pt
\fboxrule = .4pt
\long\def\fbox#1{\leavevmode\setbox\ps@tempboxa\hbox{#1}\ps@tempdima\fboxrule
    \advance\ps@tempdima \fboxsep \advance\ps@tempdima \dp\ps@tempboxa
   \hbox{\lower \ps@tempdima\hbox
  {\vbox{\hrule height \fboxrule
          \hbox{\vrule width \fboxrule \hskip\fboxsep
          \vbox{\vskip\fboxsep \box\ps@tempboxa\vskip\fboxsep}\hskip
                 \fboxsep\vrule width \fboxrule}
                 \hrule height \fboxrule}}}}
\fi
%
%%%%%%%%%%%%%%%%%%%%%%%%%%%%%%%%%%%%%%%%%%%%%%%%%%%%%%%%%%%%%%%%%%%
% file reading stuff from epsf.tex
%   EPSF.TEX macro file:
%   Written by Tomas Rokicki of Radical Eye Software, 29 Mar 1989.
%   Revised by Don Knuth, 3 Jan 1990.
%   Revised by Tomas Rokicki to accept bounding boxes with no
%      space after the colon, 18 Jul 1990.
%   Portions modified/removed for use in PSFIG package by
%      J. Daniel Smith, 9 October 1990.
%
\newread\ps@stream
\newif\ifnot@eof       % continue looking for the bounding box?
\newif\if@noisy        % report what you're making?
\newif\if@atend        % %%BoundingBox: has (at end) specification
\newif\if@psfile       % does this look like a PostScript file?
%
% PostScript files should start with `%!'
%
{\catcode`\%=12\global\gdef\epsf@start{%!}}
\def\epsf@PS{PS}
%
\def\epsf@getbb#1{%
%
%   The first thing we need to do is to open the
%   PostScript file, if possible.
%
\openin\ps@stream=#1
\ifeof\ps@stream\ps@typeout{Error, File #1 not found}\else
%
%   Okay, we got it. Now we'll scan lines until we find one that doesn't
%   start with %. We're looking for the bounding box comment.
%
   {\not@eoftrue \chardef\other=12
    \def\do##1{\catcode`##1=\other}\dospecials \catcode`\ =10
    \loop
       \if@psfile
  \read\ps@stream to \epsf@fileline
       \else{
  \obeyspaces
          \read\ps@stream to \epsf@tmp\global\let\epsf@fileline\epsf@tmp}
       \fi
       \ifeof\ps@stream\not@eoffalse\else
%
%   Check the first line for `%!'.  Issue a warning message if its not
%   there, since the file might not be a PostScript file.
%
       \if@psfile\else
       \expandafter\epsf@test\epsf@fileline:. \\%
       \fi
%
%   We check to see if the first character is a % sign;
%   if so, we look further and stop only if the line begins with
%   `%%BoundingBox:' and the `(atend)' specification was not found.
%   That is, the only way to stop is when the end of file is reached,
%   or a `%%BoundingBox: llx lly urx ury' line is found.
%
          \expandafter\epsf@aux\epsf@fileline:. \\%
       \fi
   \ifnot@eof\repeat
   }\closein\ps@stream\fi}%
%
% This tests if the file we are reading looks like a PostScript file.
%
\long\def\epsf@test#1#2#3:#4\\{\def\epsf@testit{#1#2}
\ifx\epsf@testit\epsf@start\else
\ps@typeout{Warning! File does not start with `\epsf@start'.  It may not be a PostScript file.}
\fi
\@psfiletrue} % don't test after 1st line
%
%   We still need to define the tricky \epsf@aux macro. This requires
%   a couple of magic constants for comparison purposes.
%
{\catcode`\%=12\global\let\epsf@percent=%\global\def\epsf@bblit{%BoundingBox}}
%
%
%   So we're ready to check for `%BoundingBox:' and to grab the
%   values if they are found.  We continue searching if `(at end)'
%   was found after the `%BoundingBox:'.
%
\long\def\epsf@aux#1#2:#3\\{\ifx#1\epsf@percent
   \def\epsf@testit{#2}\ifx\epsf@testit\epsf@bblit
\@atendfalse
        \epsf@atend #3 . \\%
\if@atend
   \if@verbose{
\ps@typeout{psfig: found `(atend)'; continuing search}
   }\fi
        \else
        \epsf@grab #3 . . . \\%
        \not@eoffalse
        \global\no@bbfalse
        \fi
   \fi\fi}%
%
%   Here we grab the values and stuff them in the appropriate definitions.
%
\def\epsf@grab #1 #2 #3 #4 #5\\{%
   \global\def\epsf@llx{#1}\ifx\epsf@llx\empty
      \epsf@grab #2 #3 #4 #5 .\\\else
   \global\def\epsf@lly{#2}%
   \global\def\epsf@urx{#3}\global\def\epsf@ury{#4}\fi}%
%
% Determine if the stuff following the %%BoundingBox is `(atend)'
% J. Daniel Smith.  Copied from \epsf@grab above.
%
\def\epsf@atendlit{(atend)}
\def\epsf@atend #1 #2 #3\\{%
   \def\epsf@tmp{#1}\ifx\epsf@tmp\empty
      \epsf@atend #2 #3 .\\\else
   \ifx\epsf@tmp\epsf@atendlit\@atendtrue\fi\fi}


% End of file reading stuff from epsf.tex
%%%%%%%%%%%%%%%%%%%%%%%%%%%%%%%%%%%%%%%%%%%%%%%%%%%%%%%%%%%%%%%%%%%

%%%%%%%%%%%%%%%%%%%%%%%%%%%%%%%%%%%%%%%%%%%%%%%%%%%%%%%%%%%%%%%%%%%
% trigonometry stuff from "trig.tex"
\chardef\psletter = 11 % won't conflict with \begin{letter} now...
\chardef\other = 12

\newif \ifdebug %%% turn me on to see TeX hard at work ...
\newif\ifc@mpute %%% don't need to compute some values
\c@mputetrue % but assume that we do

\let\then = \relax
\def\r@dian{pt }
\let\r@dians = \r@dian
\let\dimensionless@nit = \r@dian
\let\dimensionless@nits = \dimensionless@nit
\def\internal@nit{sp }
\let\internal@nits = \internal@nit
\newif\ifstillc@nverging
\def \Mess@ge #1{\ifdebug \then \message {#1} \fi}

{ %%% Things that need abnormal catcodes %%%
\catcode `\@ = \psletter
\gdef \nodimen {\expandafter \n@dimen \the \dimen}
\gdef \term #1 #2 #3%
       {\edef \t@ {\the #1}%%% freeze parameter 1 (count, by value)
\edef \t@@ {\expandafter \n@dimen \the #2\r@dian}%
   %%% freeze parameter 2 (dimen, by value)
\t@rm {\t@} {\t@@} {#3}%
       }
\gdef \t@rm #1 #2 #3%
       {{%
\count 0 = 0
\dimen 0 = 1 \dimensionless@nit
\dimen 2 = #2\relax
\Mess@ge {Calculating term #1 of \nodimen 2}%
\loop
\ifnum\count 0 < #1
\then\advance \count 0 by 1
\Mess@ge {Iteration \the \count 0 \space}%
\Multiply \dimen 0 by {\dimen 2}%
\Mess@ge {After multiplication, term = \nodimen 0}%
\Divide \dimen 0 by {\count 0}%
\Mess@ge {After division, term = \nodimen 0}%
\repeat
\Mess@ge {Final value for term #1 of
\nodimen 2 \space is \nodimen 0}%
\xdef \Term {#3 = \nodimen 0 \r@dians}%
\aftergroup \Term
       }}
\catcode `\p = \other
\catcode `\t = \other
\gdef \n@dimen #1pt{#1} %%% throw away the ``pt''
}

\def \Divide #1by #2{\divide #1 by #2} %%% just a synonym

\def \Multiply #1by #2%%% allows division of a dimen by a dimen
       {{%%% should really freeze parameter 2 (dimen, passed by value)
\count 0 = #1\relax
\count 2 = #2\relax
\count 4 = 65536
\Mess@ge {Before scaling, count 0 = \the \count 0 \space and
count 2 = \the \count 2}%
\ifnum\count 0 > 32767 %%% do our best to avoid overflow
\then\divide \count 0 by 4
\divide \count 4 by 4
\else\ifnum\count 0 < -32767
\then\divide \count 0 by 4
\divide \count 4 by 4
\else
\fi
\fi
\ifnum\count 2 > 32767 %%% while retaining reasonable accuracy
\then\divide \count 2 by 4
\divide \count 4 by 4
\else\ifnum\count 2 < -32767
\then\divide \count 2 by 4
\divide \count 4 by 4
\else
\fi
\fi
\multiply \count 0 by \count 2
\divide \count 0 by \count 4
\xdef \product {#1 = \the \count 0 \internal@nits}%
\aftergroup \product
       }}

\def\r@duce{\ifdim\dimen0 > 90\r@dian \then   % sin(x+90) = sin(180-x)
\multiply\dimen0 by -1
\advance\dimen0 by 180\r@dian
\r@duce
    \else \ifdim\dimen0 < -90\r@dian \then  % sin(-x) = sin(360+x)
\advance\dimen0 by 360\r@dian
\r@duce
\fi
    \fi}

\def\Sine#1%
       {{%
\dimen 0 = #1 \r@dian
\r@duce
\ifdim\dimen0 = -90\r@dian \then
   \dimen4 = -1\r@dian
   \c@mputefalse
\fi
\ifdim\dimen0 = 90\r@dian \then
   \dimen4 = 1\r@dian
   \c@mputefalse
\fi
\ifdim\dimen0 = 0\r@dian \then
   \dimen4 = 0\r@dian
   \c@mputefalse
\fi
%
\ifc@mpute \then
        % convert degrees to radians
\divide\dimen0 by 180
\dimen0=3.141592654\dimen0
%
\dimen 2 = 3.1415926535897963\r@dian %%% a well-known constant
\divide\dimen 2 by 2 %%% we only deal with -pi/2 : pi/2
\Mess@ge {Sin: calculating Sin of \nodimen 0}%
\count 0 = 1 %%% see power-series expansion for sine
\dimen 2 = 1 \r@dian %%% ditto
\dimen 4 = 0 \r@dian %%% ditto
\loop
\ifnum\dimen 2 = 0 %%% then we've done
\then\stillc@nvergingfalse
\else\stillc@nvergingtrue
\fi
\ifstillc@nverging %%% then calculate next term
\then\term {\count 0} {\dimen 0} {\dimen 2}%
\advance \count 0 by 2
\count 2 = \count 0
\divide \count 2 by 2
\ifodd\count 2 %%% signs alternate
\then\advance \dimen 4 by \dimen 2
\else\advance \dimen 4 by -\dimen 2
\fi
\repeat
\fi
\xdef \sine {\nodimen 4}%
       }}

% Now the Cosine can be calculated easily by calling \Sine
\def\Cosine#1{\ifx\sine\UnDefined\edef\Savesine{\relax}\else
             \edef\Savesine{\sine}\fi
{\dimen0=#1\r@dian\advance\dimen0 by 90\r@dian
 \Sine{\nodimen 0}
 \xdef\cosine{\sine}
 \xdef\sine{\Savesine}}}     
% end of trig stuff
%%%%%%%%%%%%%%%%%%%%%%%%%%%%%%%%%%%%%%%%%%%%%%%%%%%%%%%%%%%%%%%%%%%%

\def\psdraft{
\def\@psdraft{0}
%\ps@typeout{draft level now is \@psdraft \space . }
}
\def\psfull{
\def\@psdraft{100}
%\ps@typeout{draft level now is \@psdraft \space . }
}

\psfull

\newif\if@scalefirst
\def\psscalefirst{\@scalefirsttrue}
\def\psrotatefirst{\@scalefirstfalse}
\psrotatefirst

\newif\if@draftbox
\def\psnodraftbox{
\@draftboxfalse
}
\def\psdraftbox{
\@draftboxtrue
}
\@draftboxtrue

\newif\if@prologfile
\newif\if@postlogfile
\def\pssilent{
\@noisyfalse
}
\def\psnoisy{
\@noisytrue
}
\psnoisy
%%% These are for the option list.
%%% A specification of the form a = b maps to calling \@p@@sa{b}
\newif\if@bbllx
\newif\if@bblly
\newif\if@bburx
\newif\if@bbury
\newif\if@height
\newif\if@width
\newif\if@rheight
\newif\if@rwidth
\newif\if@angle
\newif\if@clip
\newif\if@verbose
\def\@p@@sclip#1{\@cliptrue}


\newif\if@decmpr

%%% GDH 7/26/87 -- changed so that it first looks in the local directory,
%%% then in a specified global directory for the ps file.
%%% RPR 6/25/91 -- changed so that it defaults to user-supplied name if
%%% boundingbox info is specified, assuming graphic will be created by
%%% print time.
%%% TJD 10/19/91 -- added bbfile vs. file distinction, and @decmpr flag

\def\@p@@sfigure#1{\def\@p@sfile{null}\def\@p@sbbfile{null}
        \openin1=#1.bb
\ifeof1\closein1
        \openin1=\figurepath#1.bb
\ifeof1\closein1
        \openin1=#1
\ifeof1\closein1%
       \openin1=\figurepath#1
\ifeof1
   \ps@typeout{Error, File #1 not found}
\if@bbllx\if@bblly
   \if@bburx\if@bbury
      \def\@p@sfile{#1}%
      \def\@p@sbbfile{#1}%
\@decmprfalse
     \fi\fi\fi\fi
\else\closein1
    \def\@p@sfile{\figurepath#1}%
    \def\@p@sbbfile{\figurepath#1}%
\@decmprfalse
                       \fi%
 \else\closein1%
\def\@p@sfile{#1}
\def\@p@sbbfile{#1}
\@decmprfalse
 \fi
\else
\def\@p@sfile{\figurepath#1}
\def\@p@sbbfile{\figurepath#1.bb}
\@decmprtrue
\fi
\else
\def\@p@sfile{#1}
\def\@p@sbbfile{#1.bb}
\@decmprtrue
\fi}

\def\@p@@sfile#1{\@p@@sfigure{#1}}

\def\@p@@sbbllx#1{
%\ps@typeout{bbllx is #1}
\@bbllxtrue
\dimen100=#1
\edef\@p@sbbllx{\number\dimen100}
}
\def\@p@@sbblly#1{
%\ps@typeout{bblly is #1}
\@bbllytrue
\dimen100=#1
\edef\@p@sbblly{\number\dimen100}
}
\def\@p@@sbburx#1{
%\ps@typeout{bburx is #1}
\@bburxtrue
\dimen100=#1
\edef\@p@sbburx{\number\dimen100}
}
\def\@p@@sbbury#1{
%\ps@typeout{bbury is #1}
\@bburytrue
\dimen100=#1
\edef\@p@sbbury{\number\dimen100}
}
\def\@p@@sheight#1{
\@heighttrue
\dimen100=#1
   \edef\@p@sheight{\number\dimen100}
%\ps@typeout{Height is \@p@sheight}
}
\def\@p@@swidth#1{
%\ps@typeout{Width is #1}
\@widthtrue
\dimen100=#1
\edef\@p@swidth{\number\dimen100}
}
\def\@p@@srheight#1{
%\ps@typeout{Reserved height is #1}
\@rheighttrue
\dimen100=#1
\edef\@p@srheight{\number\dimen100}
}
\def\@p@@srwidth#1{
%\ps@typeout{Reserved width is #1}
\@rwidthtrue
\dimen100=#1
\edef\@p@srwidth{\number\dimen100}
}
\def\@p@@sangle#1{
%\ps@typeout{Rotation is #1}
\@angletrue
%\dimen100=#1
\edef\@p@sangle{#1} %\number\dimen100}
}
\def\@p@@ssilent#1{
\@verbosefalse
}
\def\@p@@sprolog#1{\@prologfiletrue\def\@prologfileval{#1}}
\def\@p@@spostlog#1{\@postlogfiletrue\def\@postlogfileval{#1}}
\def\@cs@name#1{\csname #1\endcsname}
\def\@setparms#1=#2,{\@cs@name{@p@@s#1}{#2}}
%
% initialize the defaults (size the size of the figure)
%
\def\ps@init@parms{
\@bbllxfalse \@bbllyfalse
\@bburxfalse \@bburyfalse
\@heightfalse \@widthfalse
\@rheightfalse \@rwidthfalse
\def\@p@sbbllx{}\def\@p@sbblly{}
\def\@p@sbburx{}\def\@p@sbbury{}
\def\@p@sheight{}\def\@p@swidth{}
\def\@p@srheight{}\def\@p@srwidth{}
\def\@p@sangle{0}
\def\@p@sfile{} \def\@p@sbbfile{}
\def\@p@scost{10}
\def\@sc{}
\@prologfilefalse
\@postlogfilefalse
\@clipfalse
\if@noisy
\@verbosetrue
\else
\@verbosefalse
\fi
}
%
% Go through the options setting things up.
%
\def\parse@ps@parms#1{
 \@psdo\@psfiga:=#1\do
   {\expandafter\@setparms\@psfiga,}}
%
% Compute bb height and width
%
\newif\ifno@bb
\def\bb@missing{
\if@verbose{
\ps@typeout{psfig: searching \@p@sbbfile \space  for bounding box}
}\fi
\no@bbtrue
\epsf@getbb{\@p@sbbfile}
        \ifno@bb \else \bb@cull\epsf@llx\epsf@lly\epsf@urx\epsf@ury\fi
}
\def\bb@cull#1#2#3#4{
\dimen100=#1 bp\edef\@p@sbbllx{\number\dimen100}
\dimen100=#2 bp\edef\@p@sbblly{\number\dimen100}
\dimen100=#3 bp\edef\@p@sbburx{\number\dimen100}
\dimen100=#4 bp\edef\@p@sbbury{\number\dimen100}
\no@bbfalse
}
% rotate point (#1,#2) about (0,0).
% The sine and cosine of the angle are already stored in \sine and
% \cosine.  The result is placed in (\p@intvaluex, \p@intvaluey).
\newdimen\p@intvaluex
\newdimen\p@intvaluey
\def\rotate@#1#2{{\dimen0=#1 sp\dimen1=#2 sp
%            calculate x' = x \cos\theta - y \sin\theta
  \global\p@intvaluex=\cosine\dimen0
  \dimen3=\sine\dimen1
  \global\advance\p@intvaluex by -\dimen3
% calculate y' = x \sin\theta + y \cos\theta
  \global\p@intvaluey=\sine\dimen0
  \dimen3=\cosine\dimen1
  \global\advance\p@intvaluey by \dimen3
  }}
\def\compute@bb{
\no@bbfalse
\if@bbllx \else \no@bbtrue \fi
\if@bblly \else \no@bbtrue \fi
\if@bburx \else \no@bbtrue \fi
\if@bbury \else \no@bbtrue \fi
\ifno@bb \bb@missing \fi
\ifno@bb \ps@typeout{FATAL ERROR: no bb supplied or found}
\no-bb-error
\fi
%
%\ps@typeout{BB: \@p@sbbllx, \@p@sbblly, \@p@sbburx, \@p@sbbury}
%
% store height/width of original (unrotated) bounding box
\count203=\@p@sbburx
\count204=\@p@sbbury
\advance\count203 by -\@p@sbbllx
\advance\count204 by -\@p@sbblly
\edef\ps@bbw{\number\count203}
\edef\ps@bbh{\number\count204}
%\ps@typeout{ psbbh = \ps@bbh, psbbw = \ps@bbw }
\if@angle
\Sine{\@p@sangle}\Cosine{\@p@sangle}
        {\dimen100=\maxdimen\xdef\r@p@sbbllx{\number\dimen100}
    \xdef\r@p@sbblly{\number\dimen100}
                    \xdef\r@p@sbburx{-\number\dimen100}
    \xdef\r@p@sbbury{-\number\dimen100}}
%
% Need to rotate all four points and take the X-Y extremes of the new
% points as the new bounding box.
                        \def\minmaxtest{
   \ifnum\number\p@intvaluex<\r@p@sbbllx
      \xdef\r@p@sbbllx{\number\p@intvaluex}\fi
   \ifnum\number\p@intvaluex>\r@p@sbburx
      \xdef\r@p@sbburx{\number\p@intvaluex}\fi
   \ifnum\number\p@intvaluey<\r@p@sbblly
      \xdef\r@p@sbblly{\number\p@intvaluey}\fi
   \ifnum\number\p@intvaluey>\r@p@sbbury
      \xdef\r@p@sbbury{\number\p@intvaluey}\fi
   }
%lower left
\rotate@{\@p@sbbllx}{\@p@sbblly}
\minmaxtest
%upper left
\rotate@{\@p@sbbllx}{\@p@sbbury}
\minmaxtest
%lower right
\rotate@{\@p@sbburx}{\@p@sbblly}
\minmaxtest
%upper right
\rotate@{\@p@sbburx}{\@p@sbbury}
\minmaxtest
\edef\@p@sbbllx{\r@p@sbbllx}\edef\@p@sbblly{\r@p@sbblly}
\edef\@p@sbburx{\r@p@sbburx}\edef\@p@sbbury{\r@p@sbbury}
%\ps@typeout{rotated BB: \r@p@sbbllx, \r@p@sbblly, \r@p@sbburx, \r@p@sbbury}
\fi
\count203=\@p@sbburx
\count204=\@p@sbbury
\advance\count203 by -\@p@sbbllx
\advance\count204 by -\@p@sbblly
\edef\@bbw{\number\count203}
\edef\@bbh{\number\count204}
%\ps@typeout{ bbh = \@bbh, bbw = \@bbw }
}
%
% \in@hundreds performs #1 * (#2 / #3) correct to the hundreds,
%then leaves the result in @result
%
\def\in@hundreds#1#2#3{\count240=#2 \count241=#3
     \count100=\count240% 100 is first digit #2/#3
     \divide\count100 by \count241
     \count101=\count100
     \multiply\count101 by \count241
     \advance\count240 by -\count101
     \multiply\count240 by 10
     \count101=\count240%101 is second digit of #2/#3
     \divide\count101 by \count241
     \count102=\count101
     \multiply\count102 by \count241
     \advance\count240 by -\count102
     \multiply\count240 by 10
     \count102=\count240% 102 is the third digit
     \divide\count102 by \count241
     \count200=#1\count205=0
     \count201=\count200
\multiply\count201 by \count100
 \advance\count205 by \count201
     \count201=\count200
\divide\count201 by 10
\multiply\count201 by \count101
\advance\count205 by \count201
%
     \count201=\count200
\divide\count201 by 100
\multiply\count201 by \count102
\advance\count205 by \count201
%
     \edef\@result{\number\count205}
}
\def\compute@wfromh{
% computing : width = height * (bbw / bbh)
\in@hundreds{\@p@sheight}{\@bbw}{\@bbh}
%\ps@typeout{ \@p@sheight * \@bbw / \@bbh, = \@result }
\edef\@p@swidth{\@result}
%\ps@typeout{w from h: width is \@p@swidth}
}
\def\compute@hfromw{
% computing : height = width * (bbh / bbw)
        \in@hundreds{\@p@swidth}{\@bbh}{\@bbw}
%\ps@typeout{ \@p@swidth * \@bbh / \@bbw = \@result }
\edef\@p@sheight{\@result}
%\ps@typeout{h from w : height is \@p@sheight}
}
\def\compute@handw{
\if@height
\if@width
\else
\compute@wfromh
\fi
\else
\if@width
\compute@hfromw
\else
\edef\@p@sheight{\@bbh}
\edef\@p@swidth{\@bbw}
\fi
\fi
}
\def\compute@resv{
\if@rheight \else \edef\@p@srheight{\@p@sheight} \fi
\if@rwidth \else \edef\@p@srwidth{\@p@swidth} \fi
%\ps@typeout{rheight = \@p@srheight, rwidth = \@p@srwidth}
}
%
% Compute any missing values
\def\compute@sizes{
\compute@bb
\if@scalefirst\if@angle
% at this point the bounding box has been adjsuted correctly for
% rotation.  PSFIG does all of its scaling using \@bbh and \@bbw.  If
% a width= or height= was specified along with \psscalefirst, then the
% width=/height= value needs to be adjusted to match the new (rotated)
% bounding box size (specifed in \@bbw and \@bbh).
%    \ps@bbw       width=
%    -------  =  ----------
%    \@bbw       new width=
% so `new width=' = (width= * \@bbw) / \ps@bbw; where \ps@bbw is the
% width of the original (unrotated) bounding box.
\if@width
   \in@hundreds{\@p@swidth}{\@bbw}{\ps@bbw}
   \edef\@p@swidth{\@result}
\fi
\if@height
   \in@hundreds{\@p@sheight}{\@bbh}{\ps@bbh}
   \edef\@p@sheight{\@result}
\fi
\fi\fi
\compute@handw
\compute@resv}

%
% \psfig
% usage : \psfig{file=, height=, width=, bbllx=, bblly=, bburx=, bbury=,
%rheight=, rwidth=, clip=}
%
% "clip=" is a switch and takes no value, but the `=' must be present.
\def\psfig#1{\vbox {
% do a zero width hard space so that a single
% \psfig in a centering enviornment will behave nicely
%{\setbox0=\hbox{\ }\ \hskip-\wd0}
%
\ps@init@parms
\parse@ps@parms{#1}
\compute@sizes
%
\ifnum\@p@scost<\@psdraft{
%
\special{ps::[begin] \@p@swidth \space \@p@sheight \space
\@p@sbbllx \space \@p@sbblly \space
\@p@sbburx \space \@p@sbbury \space
startTexFig \space }
\if@angle
\special {ps:: \@p@sangle \space rotate \space}
\fi
\if@clip{
\if@verbose{
\ps@typeout{(clip)}
}\fi
\special{ps:: doclip \space }
}\fi
\if@prologfile
    \special{ps: plotfile \@prologfileval \space } \fi
\if@decmpr{
\if@verbose{
\ps@typeout{psfig: including \@p@sfile.Z \space }
}\fi
\special{ps: plotfile "`zcat \@p@sfile.Z" \space }
}\else{
\if@verbose{
\ps@typeout{psfig: including \@p@sfile \space }
}\fi
\special{ps: plotfile \@p@sfile \space }
}\fi
\if@postlogfile
    \special{ps: plotfile \@postlogfileval \space } \fi
\special{ps::[end] endTexFig \space }
% Create the vbox to reserve the space for the figure.
\vbox to \@p@srheight sp{
% 1/92 TJD Changed from "true sp" to "sp" for magnification.
\hbox to \@p@srwidth sp{
\hss
}
\vss
}
}\else{
% draft figure, just reserve the space and print the
% path name.
\if@draftbox{
% Verbose draft: print file name in box
\hbox{\frame{\vbox to \@p@srheight sp{
\vss
\hbox to \@p@srwidth sp{ \hss \@p@sfile \hss }
\vss
}}}
}\else{
% Non-verbose draft
\vbox to \@p@srheight sp{
\vss
\hbox to \@p@srwidth sp{\hss}
\vss
}
}\fi



}\fi
}}
\psfigRestoreAt
\let\@=\LaTeXAtSign



%%%%%%%%%%%%%%%%%%%%%%%%%%%%%%%% end of file %%%%%%%%%%%%%%%%%%%%%%%%%%%%%%%%

                    
\begin{document}
\def\propnumber{}                       % Only to be filled in by NOT staff
\putnumberHuge
\nothead

%%%%%%%%%%%%%%%%%%%%%%%%%%%%%%%%%%%%%%%%%%%%%%%%%%%%%%%%%%%%%%%%%%%%%%%%%%%%%
%%                                                                         %%
%%                    *** NOTE TO APPLICANTS ***                           %%
%%                                                                         %%
%%%%%%%%%%%%%%%%%%%%%%%%%%%%%%%%%%%%%%%%%%%%%%%%%%%%%%%%%%%%%%%%%%%%%%%%%%%%%
%%                                                                         %%
%%          NOT PROPOSAL TEMPLATE FILE FOR OBSERVING PERIOD 48             %%
%%                                                                         %%
%%                   OCTOBER 1, 2013 - APRIL 1, 2014                       %%
%%                                                                         %%
%%                                                                         %%
%% Please take care to fill in the fields of this proposal form as         %%
%% indicated, following the instructions and advice provided in the        %%
%% header of each section. Run LaTex on your completed form and verify     %%
%% the result before submitting to check that it runs correctly and        %%
%% do not overfill any of the boxes, and that it produces a total of no    %%
%% more than six (6) printed pages.                                        %%
%%                                                                         %%
%% Be sure to always use the correct version of the not-style file for     %%
%% the period in question. This template is only valid for period 48.      %%
%%                                                                         %%
%% Never change the format of the template or style file. Proposals        %%
%% that do not comply with the correct version of the style and            %%
%% template files will be rejected.                                        %%
%%                                                                         %%
%% Name the file PIname.tex (e.g. johanson.tex) and any figure file(s)     %%
%% as PInameA.ps (and PInameB.ps). After verifying that the proposal       %%
%% can be properly processed, submit the file(s) as (separate) attached    %%
%% file(s) by e-mail to the address:                                       %%
%%                                                                         %%
%%                          proposal@not.iac.es                            %%
%%                                                                         %%
%% with the word ``Proposal'' in the 'Subject' field and as text in the    %%
%% body of the message. The latter is important when you use certain       %%
%% mailers as the proposal might otherwise not be parsed correctly by      %%
%% our automatic procedure.                                                %%
%%                                                                         %%
%% Do not compress the files or combine them in a tar file. Do not         %%
%% submit the style file.                                                  %%
%%                                                                         %%
%% Any questions regarding the proposal procedure may be submitted to      %%
%% the same e-mail address, giving ``Question'' as the 'Subject'.          %%
%%                                                                         %%
%% If you submit more than one proposal, please name the file              %%
%% PIname1.tex, PIname2.tex, etc., and any figure files accordingly.       %%
%% Only one proposal should be submitted at the time.                      %%
%%                                                                         %%
%% For more information on the Nordic Optical Telescope see:               %%
%%                                                                         %%
%%                       http://www.not.iac.es/                            %%
%%                                                                         %%
%%%%%%%%%%%%%%%%%%%%%%%%%%%%%%%%%%%%%%%%%%%%%%%%%%%%%%%%%%%%%%%%%%%%%%%%%%%%%


\begin{titpage}{}
%%%%%%%%%%%%%%%%%%%%%%%%%%%%%%%%%%%%%%%%%%%%%%%%%%%%%%%%%%%%%%%%%%%%%%%%%%%%%
%                          PROPOSAL TITLE
% Type title of proposal in the { } below - one line only!
%
\proptitle{Measuring the Rotation Curve of the Elusive NGC 5963: The Adventure.}
%%%%%%%%%%%%%%%%%%%%%%%%%%%%%%%%%%%%%%%%%%%%%%%%%%%%%%%%%%%%%%%%%%%%%%%%%%%%%
\end{titpage}


\begin{abspage}[][]{}
%%%%%%%%%%%%%%%%%%%%%%%%%%%%%%%%%%%%%%%%%%%%%%%%%%%%%%%%%%%%%%%%%%%%%%%%%%%%%
%%                             ABSTRACT
%
% Please type the Abstract of the proposal into the { } below
% Do not exceed the space provided 
%
\propabstract{}
%%%%%%%%%%%%%%%%%%%%%%%%%%%%%%%%%%%%%%%%%%%%%%%%%%%%%%%%%%%%%%%%%%%%%%%%%%%%%
\end{abspage}


\begin{adrinvpage}{}
%%%%%%%%%%%%%%%%%%%%%%%%%%%%%%%%%%%%%%%%%%%%%%%%%%%%%%%%%%%%%%%%%%%%%%%%%%%%%
%%                      PRINCIPAL INVESTIGATOR
%
% Name and address of Principal Investigator (PI)
% 
% NB: The PI has full responsibility for the content of this proposal!
%
% Please fill in the appropriate { } below:
%
\piname{}               % name of PI
\piinst{}               % PI institute
\picoun{SE}               % PI country (ISO code: DK,FI,IS,NO,SE,..) 
\piaddr{}               % PI postal address
\piteln{}               % PI telephone number
\pifaxn{}               % PI fax number
\pimail{}               % PI email address
%%%%%%%%%%%%%%%%%%%%%%%%%%%%%%%%%%%%%%%%%%%%%%%%%%%%%%%%%%%%%%%%%%%%%%%%%%%%%
\end{adrinvpage}


\begin{coinvestpage}{}
%%%%%%%%%%%%%%%%%%%%%%%%%%%%%%%%%%%%%%%%%%%%%%%%%%%%%%%%%%%%%%%%%%%%%%%%%%%%%
%%                          CO-INVESTIGATORS
% 
% Name and institute of co-investigators
% Please fill in the { } { } fields below (2 Co-Is per line):
%
% There is room for up to 10 CoIs. Even if the project involves more 
% than 10 CoIs, please do not list more than 10 CoIs 
%
%        {Name1, Institute1}  {Name2, Institute2}
%
\coinvest{ }{ }  % {Name1, Institute1}   {Name2, Institute2} 
\coinvest{ }{ }  % {Name3, Institute3}   etc
\coinvest{ }{ }
\coinvest{ }{ }
\coinvest{ }{ }
% 
%%%%%%%%%%%%%%%%%%%%%%%%%%%%%%%%%%%%%%%%%%%%%%%%%%%%%%%%%%%%%%%%%%%%%%%%%%%%%
\end{coinvestpage}


\begin{omthesispage}{}
%%%%%%%%%%%%%%%%%%%%%%%%%%%%%%%%%%%%%%%%%%%%%%%%%%%%%%%%%%%%%%%%%%%%%%%%%%%%%
%%                        THESIS PROJECTS
%
% If this proposal concerns a PhD thesis work at Nordic Institute,
% please provide: name of the student, institute, name of supervisor, 
% and expected time of completion.
%
% Please type in the { } field below:
\thesis{} 
%%%%%%%%%%%%%%%%%%%%%%%%%%%%%%%%%%%%%%%%%%%%%%%%%%%%%%%%%%%%%%%%%%%%%%%%%%%%%
\end{omthesispage}


\begin{nightspage}{}
%%%%%%%%%%%%%%%%%%%%%%%%%%%%%%%%%%%%%%%%%%%%%%%%%%%%%%%%%%%%%%%%%%%%%%%%%%%%%
%%                       REQUESTED OBSERVING RUN(S)
%% 
%% NB: In the following, an ``Observing run'' is a single, contiguous block 
%% of time with a single instrument. If your project requires more than one
%% such run, e.g. at different times and/or with different instruments, 
%% please identify each run as A, B, C,... and specify your requirements 
%% for each on a separate line as specified below.
%%
%%%%%%%%%%%%%%%%%%%%%%%%%%%%%%%%%%%%%%%%%%%%%%%%%%%%%%%%%%%%%%%%%%%%%%%%%%%%%
%%
%% Give requested no. of nights/hours as a number (not word) and specify
%% the unit: as N (nights) or H (hours) 
%%
%% Indicate desired Moon phases as D=dark/G=grey/N=no restriction
%% 
%% Indicate the seeing requirements: 0.7 (max 0.7 arcsec seeing), 
%%      1.0 (max 1.0 arcsec), 1.3 (max 1.3 arcsec), or N (no restriction)
%%
%% If the programme require specific sky condition, enter these here
%%      P = photometric conditions required
%%      C = clear conditions required
%%      T = thin clouds/cirrus acceptable
%% 
%% Please fill in the relevant information in the {   } fields below, and 
%% duplicate the entire block if more than one run is requested
%%
%% N.B. A maximum of 6 runs per proposal can be specified
%%
%%%%%%%%%%%%%%%%%%%%%%%%%%%%%%%%%%%%%%%%%%%%%%%%%%%%%%%%%%%%%%%%%%%%%%%%%%%%%
%%
%%%%%%%%%%%%%%%%%%%%%%%%%%%%%% Run A %%%%%%%%%%%%%%%%%%%%%%%%%%%%%%%%%%%%%%%%
%
\nrunid{A }      % Put your run id (A, B, C, ...) here
\ninstr{ALFOSC }      % Put instrument name here
\ntimer{XXX H }      % Put requested time, in numbers, with unit (N or H), e.g 5 N
\nmonth{May }      % Put preferred month(s) here
\nmoonp{B }      % Put requested moon phase here: D, G or N
\nsemax{ }      % Put maximum allowed seeing here: (0.7,1.0,1.5,N)
\nskyco{ }      % Put required photometric condition here
%
%%%%%%%%%%%%%%%%%%%%%%%%%%%%%%%%%%%%%%%%%%%%%%%%%%%%%%%%%%%%%%%%%%%%%%%%%%%%%
\end{nightspage}

\begin{numnightspage}{}
%%%%%%%%%%%%%%%%%%%%%%%%%%%%%%%%%%%%%%%%%%%%%%%%%%%%%%%%%%%%%%%%%%%%%%%%%%%%%
%%                  TIME BEFORE/AFTER PRESENT REQUEST
%%
% Number of nights already awarded to project. More details, e.g. on 
% instrumentation and outcome of previous observations can be given in box 17
%
% Please type in the { } field below:
% 
\numalr{}
%
% Number of nights needed to complete project (excluding those requested).
%
\numrem{}
%%%%%%%%%%%%%%%%%%%%%%%%%%%%%%%%%%%%%%%%%%%%%%%%%%%%%%%%%%%%%%%%%%%%%%%%%%%%%
\end{numnightspage}


\begin{servicepage}[][]{}
%%%%%%%%%%%%%%%%%%%%%%%%%%%%%%%%%%%%%%%%%%%%%%%%%%%%%%%%%%%%%%%%%%%%%%%%%%%%%
%%                         SERVICE CONSTRAINTS
%
%
% All projects will be considered for execution in service/queue mode. In
% case your project can not be done in service/queue mode, please give a
% justification in the { } field below:
\service{ }
%%%%%%%%%%%%%%%%%%%%%%%%%%%%%%%%%%%%%%%%%%%%%%%%%%%%%%%%%%%%%%%%%%%%%%%%%%%%%
\end{servicepage}


\begin{schedpage}[][]{}
%%%%%%%%%%%%%%%%%%%%%%%%%%%%%%%%%%%%%%%%%%%%%%%%%%%%%%%%%%%%%%%%%%%%%%%%%%%%%
%%                         SCHEDULING CONSTRAINTS
%
%
% Any other special constraints on the scheduling?
%
% E.g. time critical scheduling, or required baseline vs phase coverage
% for monitoring programs, response time for target of opportunity, 
% simultaneous observations, impossible dates, etc... 
%
% Please type in the { } field below:
%
\schedconstr{ }
%%%%%%%%%%%%%%%%%%%%%%%%%%%%%%%%%%%%%%%%%%%%%%%%%%%%%%%%%%%%%%%%%%%%%%%%%%%%%
\end{schedpage}


%%%%%%%%%%%%%%%%%%%%%%%%%%%%%%%%%%%%%%%%%%%%%%%%%%%%%%%%%%%%%%%%%%%%%%%%%%%%%
%%
%%              !!!!!!!!!!!!! PAGE 2 !!!!!!!!!!!!!
%%
%%%%%%%%%%%%%%%%%%%%%%%%%%%%%%%%%%%%%%%%%%%%%%%%%%%%%%%%%%%%%%%%%%%%%%%%%%%%%
\newpage        % Page 2  
\putnumberLarge

\begin{scienpage}[][]{}
%%%%%%%%%%%%%%%%%%%%%%%%%%%%%%%%%%%%%%%%%%%%%%%%%%%%%%%%%%%%%%%%%%%%%%%%%%%%%
%%                     SCIENTIFIC JUSTIFICATION                            %%
%%                                                                         %%
%% Note: This should be self-contained and not refer to previous proposals.%%
%%                                                                         %%
%% Describe first the scientific background and main goals of the proposal.%%
%% As OPC members cannot be experts in every field, it is CRUCIAL that you %% 
%% outline the general scientific context CLEARLY and in a manner that can %% 
%% be understood also by a non-specialist in your field.                   %% 
%%                                                                         %% 
%% Then argue - equally clearly! - how your proposed observing project     %% 
%% will contribute significantly to advancing our understanding of the     %% 
%% issue. Key references to the literature should be given.                %% 
%%                                                                         %%
%% Finally, describe how the data reduction and scientific analysis will   %%
%% be done, and document that the team possesses the required expertise.   %%
%%                                                                         %%
%% All text and figures should fit on the following two pages (page 2 and  %%
%% page 3), but text may spill over on page 3. All figures and references  %%
%% should be placed on page 3.                                             %%
%%                                                                         %%
%%%%%%%%%%%%%%%%%%%%%%%%%%%%%%%%%%%%%%%%%%%%%%%%%%%%%%%%%%%%%%%%%%%%%%%%%%%%%
%%
% Please type your text into the { } field below
\scijust{Dark matter was first termed in a paper from 1933 [ref] by Fritz
Zwicky. He used the virial theorem to calculate the gravitational mass of
the galaxies in the Coma cluster and found a discrepancy between the
measured mass and their expected luminosity. He referred to this
"missing mass" as "dunkle materie". Today astronomers have accumulated
convincing evidence of dark matter from independent observations such
as galaxy rotation curves, gravitational lensing, measurements of
the cosmic microwave background, baryon acoustic oscillations
, supernovae distance measurements, Lyman-alpha forest measurements 
of distant galaxies and in structure formation scenarios.
\par
According to the spectacularly successful Planck mission, the dark matter
part of the energy in the universe is a staggering 26.8\%
compared with the 4.9\% of ordinary matter. Even though the consensus among scientist
today is that dark matter consists of Weakly Interacting Massive Particles
(WIMPs), no official detections of these elusive particles have been made and
the hunt for these particles is one of the major undertakings of modern physics.
In what better way to make aspiring student of astronomy more comfortable
with observational instruments, than for them to "see" for
themselves what the "fuss" is all about? The reproducibility of science is
after all one of the fundamental pillars of science itself.
By the guidance of past and present mentors we therefore propose to use the
NOT telescope
to measure the rotation curve of NGC 5963, fit a light+dark mass profile 
to the acquired data and determine the stellar/dark matter mass components of
this galaxy.
 
   }
%%%%%%%%%%%%%%%%%%%%%%%%%%%%%%%%%%%%%%%%%%%%%%%%%%%%%%%%%%%%%%%%%%%%%%%%%%%%%
\end{scienpage}

%%%%%%%%%%%%%%%%%%%%%%%%%%%%%%%%%%%%%%%%%%%%%%%%%%%%%%%%%%%%%%%%%%%%%%%%%%%%%
%%
%%              !!!!!!!!!!!!! PAGE 3 !!!!!!!!!!!!!
%%
%%%%%%%%%%%%%%%%%%%%%%%%%%%%%%%%%%%%%%%%%%%%%%%%%%%%%%%%%%%%%%%%%%%%%%%%%%%%%
\newpage        % Page 3 
\putnumberLarge

\begin{scienpagec}[][]{}
%%%%%%%%%%%%%%%%%%%%%%%%%%%%%%%%%%%%%%%%%%%%%%%%%%%%%%%%%%%%%%%%%%%%%%%%%%%%%
%%              SCIENTIFIC JUSTIFICATION (CONTINUED)                       %%
%%                                                                         %%
%% Place References and any Figures here.                                  %%
%%                                                                         %%
%%%%%%%%%%%%%%%%%%%%%%%%%%%%%%%%%%%%%%%%%%%%%%%%%%%%%%%%%%%%%%%%%%%%%%%%%%%%%
% 
% Please type the rest of your text into the { } field below:
\scijustc{ }
%
%%%%%%%%%%%%%%%%%%%%%%%%%%%%%%%%%%%%%%%%%%%%%%%%%%%%%%%%%%%%%%%%%%%%%%%%%%%%%
%%                                FIGURES:                                 %%
%%                                                                         %%
%% Up to two postscript figures may be included                            %%
%%                                                                         %%
%% NB colour figures are not supported. All figures will be printed in     %%
%%    black and white and any colour information in the figures will be    %%
%%    disregarded                                                          %%
%%                                                                         %%
%% To enter a figure, uncomment the lines below, fill in the name of       %%
%% your .eps file, and provide a short caption where indicated             %%
%%                                                                         %%
%%%%%%%%%%%%%%%%%%%%%%%%%%%%%%%%%%%%%%%%%%%%%%%%%%%%%%%%%%%%%%%%%%%%%%%%%%%%%
%%
%%
%% NB: There should be no spaces in the \psfig argument below 
%% -  otherwise psfig will fail!
%%
%
%   Figure 1:   change 'PInameA.eps' to name of the file containing your 
%               figure. You may need to adjust the width and angle below,
%               and possibly the bounding box in the postscript-file.
%               Provide a short caption in the \captone{} field. Be sure 
%               to leave an empty line between the psfig and caption
%               command
%
%\psfig{file=PInameA.eps,width=12cm,angle=0,clip=}
%
%\captone{}
%
%
%   Figure 2:   change 'PInameB.eps' to name of the file containing your 
%               figure. You may need to adjust the width and angle below,
%               and possibly the bounding box in the postscript-file.
%               Provide a short caption in the \capttwo{} field. Be sure 
%               to leave an empty line between the psfig and caption
%               command
%
%\psfig{file=PInameB.eps,width=12cm,angle=0,clip=}
%
%\capttwo{}
%
%
%%%%%%%%%%%%%%%%%%%%%%%%%%%%%%%%%%%%%%%%%%%%%%%%%%%%%%%%%%%%%%%%%%%%%%%%%%%%%
\end{scienpagec}



%%%%%%%%%%%%%%%%%%%%%%%%%%%%%%%%%%%%%%%%%%%%%%%%%%%%%%%%%%%%%%%%%%%%%%%%%%%%%
%%
%%              !!!!!!!!!!!!! PAGE 4 !!!!!!!!!!!!!
%%
%%%%%%%%%%%%%%%%%%%%%%%%%%%%%%%%%%%%%%%%%%%%%%%%%%%%%%%%%%%%%%%%%%%%%%%%%%%%%
\newpage  % Page 4 
\putnumberLarge

\begin{justificpage}[][]{}
%%%%%%%%%%%%%%%%%%%%%%%%%%%%%%%%%%%%%%%%%%%%%%%%%%%%%%%%%%%%%%%%%%%%%%%%%%%%%
%%                   TECHNICAL JUSTIFICATION
%
%  Describe how the observations will be performed, so the feasibility of the 
%  project becomes clear. This should be self-contained and not rely on 
%  information given elsewhere. Describe the S/N ratio calculations you have 
%  used to justify the number of nights and the Moon phases you request 
%  (no correction for expected weather conditions should be applied). Also,
%  describe why the instrumental set-up is adequate for the objective of the
%  proposed observations (e.g., if the resolution of spectra is adequate to 
%  resolve the spectroscopic features you want to study).
%
% Please type in the { } field below:
\techjust{ }
%%%%%%%%%%%%%%%%%%%%%%%%%%%%%%%%%%%%%%%%%%%%%%%%%%%%%%%%%%%%%%%%%%%%%%%%%%%%%
\end{justificpage}




\begin{instrempage}[][]{}
%%%%%%%%%%%%%%%%%%%%%%%%%%%%%%%%%%%%%%%%%%%%%%%%%%%%%%%%%%%%%%%%%%%%%%%%%%%%%
%
%
%                       INSTRUMENT CONFIGURATIONS:
%
% Please specify the instrument configurations you want to use as fully as 
% possible, using the setup definitions provided below. You must uncomment
% the relevant setup definitions, i.e. remove the `%' sign in front of them,
% in order to make these lines take effect.
%
% Uncomment only the lines related to instrument configuration(s) needed
% for the acquisition of your planned  observations. Detailed information 
% on the available instruments is provided in the relevant users' manuals 
% (see http://www.not.iac.es/observing/proposals/).
%
% You also need to specify the Run ids (as defined in box 6) for which the 
% selected setups are valid. Put the run ID in the first {}, for example:
%
%       \NOTconfig{A}{ALFOSC}{Standard imaging-filters}{UBVRi}
%
%
%-----------------------------------------------------------------------
%----------------------------- ALFOSC ----------------------------------
%-----------------------------------------------------------------------
%
%\NOTconfig{}{ALFOSC}{Standard imaging-filters}{UBVRi}
%\NOTconfig{}{ALFOSC}{Imaging-filters for ALFOSC}{provide filter No.}
%\NOTconfig{}{ALFOSC}{Imaging-filters for FASU}{provide filter No.}
%
%\NOTconfig{}{ALFOSC}{Fast-photometry}{Multi-windowing mode}
%
%\NOTconfig{}{ALFOSC}{Lin-Pol-imaging}{Polaroids}
%\NOTconfig{}{ALFOSC}{Lin-Pol-imaging}{Calcite+half-wave-plate}
%\NOTconfig{}{ALFOSC}{Cir-Pol-imaging}{Calcite+quarter-wave-plate}
%\NOTconfig{}{ALFOSC}{Lin-Pol-imaging}{WeDoWo}
%
%\NOTconfig{}{ALFOSC}{Lin-Pol-spectroscopy}{Calcite+half-wave-plate}
%\NOTconfig{}{ALFOSC}{Cir-Pol-spectroscopy}{Calcite+quarter-wave-plate}
%\NOTconfig{}{ALFOSC}{Pol-spectroscopy}{give grism number(s)}
%\NOTconfig{}{ALFOSC}{Pol-spectroscopy}{Polarimetric slitlet\#1.0"}
%\NOTconfig{}{ALFOSC}{Pol-spectroscopy}{Polarimetric slitlet\#1.4"}
%\NOTconfig{}{ALFOSC}{Pol-spectroscopy}{Polarimetric slitlet\#1.8"}
%
%\NOTconfig{}{ALFOSC}{Spectroscopy}{ADC}
%
%\NOTconfig{}{ALFOSC}{Spectro-long-slit}{give grism number(s)}
%\NOTconfig{}{ALFOSC}{Spectro-long-slit}{give slit width(s)}
%\NOTconfig{}{ALFOSC}{Spectro-long-slit}{Grism\#13:480-580 (3rd-order)}
%\NOTconfig{}{ALFOSC}{Spectro-long-slit}{provide 2nd-order blocking filter No.}
%
%\NOTconfig{}{ALFOSC}{Multi-Object-Spectro}{provide HERE the number of masks}
%\NOTconfig{}{ALFOSC}{Multi-Object-Spectro}{give grism number(s)}
%\NOTconfig{}{ALFOSC}{Multi-Object-Spectro}{provide required slitwidth(s)}
%\NOTconfig{}{ALFOSC}{Multi-Object-Spectro}{Pre-imaging required}
%
%\NOTconfig{}{ALFOSC}{Spectro-Echelle}{Echelle Grism\#9}
%\NOTconfig{}{ALFOSC}{Spectro-Echelle}{Echelle Grism\#13}
%\NOTconfig{}{ALFOSC}{Spectro-Echelle}{0.7arcsec-slit}
%\NOTconfig{}{ALFOSC}{Spectro-Echelle}{0.8arcsec-slit}
%\NOTconfig{}{ALFOSC}{Spectro-Echelle}{1.0arcsec-slit}
%\NOTconfig{}{ALFOSC}{Spectro-Echelle}{1.2arcsec-slit}
%\NOTconfig{}{ALFOSC}{Spectro-Echelle}{1.6arcsec-slit}
%\NOTconfig{}{ALFOSC}{Spectro-Echelle}{1.8arcsec-slit}
%\NOTconfig{}{ALFOSC}{Spectro-Echelle}{2.2arcsec-slit}
%\NOTconfig{}{ALFOSC}{Spectro-Echelle}{Cross-disperser\#10}
%\NOTconfig{}{ALFOSC}{Spectro-Echelle}{Cross-disperser\#11}
%\NOTconfig{}{ALFOSC}{Spectro-Echelle}{Cross-disperser\#12}
%
%
%-----------------------------------------------------------------------
%----------------------------- NOTCam-----------------------------------
%-----------------------------------------------------------------------
%
%\NOTconfig{}{NOTCam}{Standard imaging-filters}{ZYJHKs}
%\NOTconfig{}{NOTCam}{Imaging-filters for NOTCam}{provide filter No.}
%
%\NOTconfig{}{NOTCam}{Imaging}{Wide-field camera}
%\NOTconfig{}{NOTCam}{Imaging}{High-res camera}
%
%\NOTconfig{}{NOTCam}{Spectro-long-slit}{Grism\#1}
%\NOTconfig{}{NOTCam}{Spectro-long-slit}{Wide-field camera slit 0.6''}
%\NOTconfig{}{NOTCam}{Spectro-long-slit}{High-res camera slit 0.2''}
%
%\NOTconfig{}{NOTCam}{Spectroscopy}{ADC}
%
%\NOTconfig{}{NOTCam}{Spectroscopy}{Wide-field camera}
%\NOTconfig{}{NOTCam}{Spectroscopy}{High-res camera}
%
%
%-----------------------------------------------------------------------
%----------------------------- FIES ------------------------------------
%-----------------------------------------------------------------------
%
% Please note that the simultaneous-ThAr mode can only be used in
% combination with either the High-res or Med-res fiber.
%
% Please note that the spec-pol mode can only be used in combination
% with the Med-Res fiber. The useful wavelength range is 370-630 nm.
%
%\NOTconfig{}{FIES}{Spectro-Echelle}{Low-Res Fiber}
%\NOTconfig{}{FIES}{Spectro-Echelle}{Med-Res Fiber}
%\NOTconfig{}{FIES}{Spectro-Echelle}{High-Res Fiber}
%
%\NOTconfig{}{FIES}{Spectro-Echelle}{Simultaneous-ThAr mode}
%
%\NOTconfig{}{FIES}{Spectro-Echelle}{Spec-Pol mode}
%
%\NOTconfig{}{FIES}{Spectro-Echelle}{ADC}
%
%
%-----------------------------------------------------------------------
%----------------------------- MOSCA -----------------------------------
%-----------------------------------------------------------------------
%
%\NOTconfig{}{MOSCA}{Standard imaging-filters}{UBVRI}
%\NOTconfig{}{MOSCA}{Standard imaging-filters}{ugriz}
%\NOTconfig{}{MOSCA}{Imaging-filters for FASU}{provide filter No.}
%
%
%-----------------------------------------------------------------------
%----------------------------- StanCam ---------------------------------
%-----------------------------------------------------------------------
%
%\NOTconfig{}{StanCam}{Standard imaging-filters}{UBVRIz}
%\NOTconfig{}{StanCam}{Imaging-filters for StanCam}{provide filter No.}
%
%
%-----------------------------------------------------------------------
%----------------------------- TurPol ----------------------------------
%-----------------------------------------------------------------------
%
% TurPol is not a common-user instrument. Normal support at the telescope 
% is provided, but only limited trouble-shooting can be made by NOT staff.
%
%\NOTconfig{}{TurPol}{Lin-Pol}{Half-wave-plate}
%\NOTconfig{}{TurPol}{Cir+Lin-Pol}{Quarter-wave-plate}
%\NOTconfig{}{TurPol}{Standard photometry}{UBVRI}
% 
%
%-----------------------------------------------------------------------
%----------------------------- Visitor Instruments ---------------------
%-----------------------------------------------------------------------
%
% If you wish to use a special instrument for the project, please contact
% the NOT director (Johannes Andersen, email: ja@astro.ku.dk), or the Head
% of Operations (Thomas Augusteijn, email: tau@not.iac.es) well in advance
% of submitting your proposal.
%
%\NOTconfig{}{Name of instrument}{key capability}{link to instrument URL}
%
%
%%%%%%%%%%%%%%%%%%%%%%%%%%%%%%%%%%%%%%%%%%%%%%%%%%%%%%%%%%%%%%%%%%%%%%%%%%%%%
%%                      REMARKS ON INSTRUMENT SETUP
%%
% If you have any special remarks or requirements, e.g. use of non standard
% observing modes, own equipment etc., please uncomment the %\remark{} line 
% below and provide the information in the {} field. 
% If necessary, more detailed information can be provided in BOX 18.
%
%\remark{} 
%%%%%%%%%%%%%%%%%%%%%%%%%%%%%%%%%%%%%%%%%%%%%%%%%%%%%%%%%%%%%%%%%%%%%%%%%%%%%
\end{instrempage}


%%%%%%%%%%%%%%%%%%%%%%%%%%%%%%%%%%%%%%%%%%%%%%%%%%%%%%%%%%%%%%%%%%%%%%%%%%%%%
%%
%%              !!!!!!!!!!!!! PAGE 5 !!!!!!!!!!!!!
%%
%%%%%%%%%%%%%%%%%%%%%%%%%%%%%%%%%%%%%%%%%%%%%%%%%%%%%%%%%%%%%%%%%%%%%%%%%%%%%
\newpage        % Page 5 
\putnumberLarge

\begin{objlistpage}{}
%%%%%%%%%%%%%%%%%%%%%%%%%%%%%%%%%%%%%%%%%%%%%%%%%%%%%%%%%%%%%%%%%%%%%%%%%%%%%
%%                            TARGET LIST                                  %% 
%% Target list with coordinates, or intervals in R.A. and Decl.            %%
%% of (sample of) objects:                                                 %%
%%                                                                         %%
%%%%%%%%%%%%%%%%%%%%%%%%%%%%%%%%%%%%%%%%%%%%%%%%%%%%%%%%%%%%%%%%%%%%%%%%%%%%%
%%                                                                         %%
%% Please fill in the relevant information for one object in the {   }     %%
%% fields below. Duplicate the line up to 20 objects; otherwise give a     %%
%% sample list and describe the full target list in the Remarks.           %%
%%                                                                         %%
%% Please specify the passband in which object magnitudes are given        %%
%%                                                                         %%
%% Targets are specified with the \target{#1}{#2}{#3}{#4}{#5}{#6}{#7}      %%
%% command, where the 7 arguments are:                                     %%
%%                                                                         %%
%%  #1: Run IDs (as defined box 6) for which target is relevant            %%
%%  #2: Short target name                                                  %%
%%  #3: Right ascension (2000)                                             %%
%%  #4: Declination (2000)                                                 %%
%%  #5: Target magnitude                                                   %%
%%  #6: Target diameter (in arcmin)                                        %% 
%%  #7: Additional Info                                                    %%
%%                                                                         %%
%%   Example:                                                              %%
%%                                                                         %%
%%   \target{A}{M31}{00:42:44}{+41:16:09}{V=4.4}{190}{$v_r=-300$ km/s}     %%
%%                                                                         %%
%%%%%%%%%%%%%%%%%%%%%%%%%%%%%%%%%%%%%%%%%%%%%%%%%%%%%%%%%%%%%%%%%%%%%%%%%%%%%
\target{}{}{}{}{}{}{} % duplicate as needed
%%%%%%%%%%%%%%%%%%%%%%%%%%%%%%%%%%%%%%%%%%%%%%%%%%%%%%%%%%%%%%%%%%%%%%%%%%%%%
%% Please type any Remarks on the target list in the {   } field below: 
%%
\remark{ }
%%%%%%%%%%%%%%%%%%%%%%%%%%%%%%%%%%%%%%%%%%%%%%%%%%%%%%%%%%%%%%%%%%%%%%%%%%%%%
\end{objlistpage}

\begin{qualitypage}[][]{}
%%%%%%%%%%%%%%%%%%%%%%%%%%%%%%%%%%%%%%%%%%%%%%%%%%%%%%%%%%%%%%%%%%%%%%%%%%%%%
%%                       BACKUP PROGRAMME
%%
% For projects needing excellent image quality or photometric conditions, 
% give a short description of possible backup programme. A backup programme 
% is also needed in case an unfavourable wind speed or direction may prevent
% you from executing your main programme.    
%
% - If no backup programme is provided, justify why none is needed. Failure
%   to ensure that the telescope will be used productively under all 
%   circumstances will lower the rating of your proposal!
%
% - The instrumental setup for the backup programme should normally be the 
%   same as for the main programme; however, standby instrumentation (e.g. 
%   StanCam or FIES) may be used instead. Please specify the instrument you 
%   will use.
% 
% Please type your description in the { } field below:
%
\backup{ }
%%%%%%%%%%%%%%%%%%%%%%%%%%%%%%%%%%%%%%%%%%%%%%%%%%%%%%%%%%%%%%%%%%%%%%%%%%%%%
\end{qualitypage}


%%%%%%%%%%%%%%%%%%%%%%%%%%%%%%%%%%%%%%%%%%%%%%%%%%%%%%%%%%%%%%%%%%%%%%%%%%%%%
%%
%%           !!!!!!!!!  PAGE 6 = Last page  !!!!!!!!!!!!!
%%
%%%%%%%%%%%%%%%%%%%%%%%%%%%%%%%%%%%%%%%%%%%%%%%%%%%%%%%%%%%%%%%%%%%%%%%%%%%%%
\newpage        % Page 6 
\putnumberLarge

\begin{prevobspage}[][]{}
%%%%%%%%%%%%%%%%%%%%%%%%%%%%%%%%%%%%%%%%%%%%%%%%%%%%%%%%%%%%%%%%%%%%%%%%%%%%%
%%              PREVIOUS OBSERVING PERIODS AND RESULTS
%%
% Please list your observing periods at the NOT within the last three years 
% and provide a brief status report on your analysis of the data. List also 
% your publications from NOT observations during the last three years. 
%
%%  Please type this information in the { } field below:
%% 
\prevobs{ }
%%%%%%%%%%%%%%%%%%%%%%%%%%%%%%%%%%%%%%%%%%%%%%%%%%%%%%%%%%%%%%%%%%%%%%%%%%%%%
\end{prevobspage}

\begin{extrapage}[][]{}
%%%%%%%%%%%%%%%%%%%%%%%%%%%%%%%%%%%%%%%%%%%%%%%%%%%%%%%%%%%%%%%%%%%%%%%%%%%%%
%%              ADDITIONAL INFO
%%
% Use this box to provide any additional information/remarks not covered 
% by the items above, e.g. related proposals to other telescopes, dates to 
% be avoided for the observations, etc. Do not use if not necessary.  
%
%%  Please type in the { } field below:
%% 
\addinfo{ }
%%%%%%%%%%%%%%%%%%%%%%%%%%%%%%%%%%%%%%%%%%%%%%%%%%%%%%%%%%%%%%%%%%%%%%%%%%%%%
\end{extrapage}


\end{document}
%%%%%%%%%%%%%%%%%%%%%%%%%%%%%%%%%%%%%%%%%%%%%%%%%%%%%%%%%%%%%%%%%%%%%%%%%%%%%
% 
%                       End of template file
% 
%%%%%%%%%%%%%%%%%%%%%%%%%%%%%%%%%%%%%%%%%%%%%%%%%%%%%%%%%%%%%%%%%%%%%%%%%%%%%

