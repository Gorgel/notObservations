%%%%%%%%%%%%%%%%%%%%%%%%%%%%%%%%%%%%%%%%%%%%%%%%%%%%%%%%%%%%%%%%%%%%%%%%%%%%%
%%                                                                         %%
%%               NOT PROPOSAL TEMPLATE  FOR PERIOD 48                      %%
%%                                                                         %%
%%                 USE WITH STYLEFILE: notsty48.sty                        %%
%%                                                                         %%
%%%%%%%%%%%%%%%%%%%%%%%%%%%%%%%%%%%%%%%%%%%%%%%%%%%%%%%%%%%%%%%%%%%%%%%%%%%%%

\documentclass[11pt]{article}           % Do not change
\usepackage{fullpage,graphicx}          % Do not change
%%%%%%%%%%%%%%%%%%%%%%%%%%%%%%%%%%%%%%%%%%%%%%%%%%%%%%%%%%%%%%%%%%%%%%%%%%%%%
%%                                                                         %%
%%               NOT PROPOSAL TEMPLATE  FOR PERIOD 48                      %%
%%                                                                         %%
%%                 USE WITH STYLEFILE: notsty48.sty                        %%
%%                                                                         %%
%%%%%%%%%%%%%%%%%%%%%%%%%%%%%%%%%%%%%%%%%%%%%%%%%%%%%%%%%%%%%%%%%%%%%%%%%%%%%

\documentclass[11pt]{article}           % Do not change
\usepackage{fullpage,graphicx}          % Do not change
%%%%%%%%%%%%%%%%%%%%%%%%%%%%%%%%%%%%%%%%%%%%%%%%%%%%%%%%%%%%%%%%%%%%%%%%%%%%%
%%                                                                         %%
%%               NOT PROPOSAL TEMPLATE  FOR PERIOD 48                      %%
%%                                                                         %%
%%                 USE WITH STYLEFILE: notsty48.sty                        %%
%%                                                                         %%
%%%%%%%%%%%%%%%%%%%%%%%%%%%%%%%%%%%%%%%%%%%%%%%%%%%%%%%%%%%%%%%%%%%%%%%%%%%%%

\documentclass[11pt]{article}           % Do not change
\usepackage{fullpage,graphicx}          % Do not change
%%%%%%%%%%%%%%%%%%%%%%%%%%%%%%%%%%%%%%%%%%%%%%%%%%%%%%%%%%%%%%%%%%%%%%%%%%%%%
%%                                                                         %%
%%               NOT PROPOSAL TEMPLATE  FOR PERIOD 48                      %%
%%                                                                         %%
%%                 USE WITH STYLEFILE: notsty48.sty                        %%
%%                                                                         %%
%%%%%%%%%%%%%%%%%%%%%%%%%%%%%%%%%%%%%%%%%%%%%%%%%%%%%%%%%%%%%%%%%%%%%%%%%%%%%

\documentclass[11pt]{article}           % Do not change
\usepackage{fullpage,graphicx}          % Do not change
\input{notprop.sty}                    
\begin{document}
\def\propnumber{}                       % Only to be filled in by NOT staff
\putnumberHuge
\nothead

%%%%%%%%%%%%%%%%%%%%%%%%%%%%%%%%%%%%%%%%%%%%%%%%%%%%%%%%%%%%%%%%%%%%%%%%%%%%%
%%                                                                         %%
%%                    *** NOTE TO APPLICANTS ***                           %%
%%                                                                         %%
%%%%%%%%%%%%%%%%%%%%%%%%%%%%%%%%%%%%%%%%%%%%%%%%%%%%%%%%%%%%%%%%%%%%%%%%%%%%%
%%                                                                         %%
%%          NOT PROPOSAL TEMPLATE FILE FOR OBSERVING PERIOD 48             %%
%%                                                                         %%
%%                   OCTOBER 1, 2013 - APRIL 1, 2014                       %%
%%                                                                         %%
%%                                                                         %%
%% Please take care to fill in the fields of this proposal form as         %%
%% indicated, following the instructions and advice provided in the        %%
%% header of each section. Run LaTex on your completed form and verify     %%
%% the result before submitting to check that it runs correctly and        %%
%% do not overfill any of the boxes, and that it produces a total of no    %%
%% more than six (6) printed pages.                                        %%
%%                                                                         %%
%% Be sure to always use the correct version of the not-style file for     %%
%% the period in question. This template is only valid for period 48.      %%
%%                                                                         %%
%% Never change the format of the template or style file. Proposals        %%
%% that do not comply with the correct version of the style and            %%
%% template files will be rejected.                                        %%
%%                                                                         %%
%% Name the file PIname.tex (e.g. johanson.tex) and any figure file(s)     %%
%% as PInameA.ps (and PInameB.ps). After verifying that the proposal       %%
%% can be properly processed, submit the file(s) as (separate) attached    %%
%% file(s) by e-mail to the address:                                       %%
%%                                                                         %%
%%                          proposal@not.iac.es                            %%
%%                                                                         %%
%% with the word ``Proposal'' in the 'Subject' field and as text in the    %%
%% body of the message. The latter is important when you use certain       %%
%% mailers as the proposal might otherwise not be parsed correctly by      %%
%% our automatic procedure.                                                %%
%%                                                                         %%
%% Do not compress the files or combine them in a tar file. Do not         %%
%% submit the style file.                                                  %%
%%                                                                         %%
%% Any questions regarding the proposal procedure may be submitted to      %%
%% the same e-mail address, giving ``Question'' as the 'Subject'.          %%
%%                                                                         %%
%% If you submit more than one proposal, please name the file              %%
%% PIname1.tex, PIname2.tex, etc., and any figure files accordingly.       %%
%% Only one proposal should be submitted at the time.                      %%
%%                                                                         %%
%% For more information on the Nordic Optical Telescope see:               %%
%%                                                                         %%
%%                       http://www.not.iac.es/                            %%
%%                                                                         %%
%%%%%%%%%%%%%%%%%%%%%%%%%%%%%%%%%%%%%%%%%%%%%%%%%%%%%%%%%%%%%%%%%%%%%%%%%%%%%


\begin{titpage}{}
%%%%%%%%%%%%%%%%%%%%%%%%%%%%%%%%%%%%%%%%%%%%%%%%%%%%%%%%%%%%%%%%%%%%%%%%%%%%%
%                          PROPOSAL TITLE
% Type title of proposal in the { } below - one line only!
%
\proptitle{Measuring the Rotation Curve of the Elusive NGC 5963: The Adventure.}
%%%%%%%%%%%%%%%%%%%%%%%%%%%%%%%%%%%%%%%%%%%%%%%%%%%%%%%%%%%%%%%%%%%%%%%%%%%%%
\end{titpage}


\begin{abspage}[][]{}
%%%%%%%%%%%%%%%%%%%%%%%%%%%%%%%%%%%%%%%%%%%%%%%%%%%%%%%%%%%%%%%%%%%%%%%%%%%%%
%%                             ABSTRACT
%
% Please type the Abstract of the proposal into the { } below
% Do not exceed the space provided 
%
\propabstract{}
%%%%%%%%%%%%%%%%%%%%%%%%%%%%%%%%%%%%%%%%%%%%%%%%%%%%%%%%%%%%%%%%%%%%%%%%%%%%%
\end{abspage}


\begin{adrinvpage}{}
%%%%%%%%%%%%%%%%%%%%%%%%%%%%%%%%%%%%%%%%%%%%%%%%%%%%%%%%%%%%%%%%%%%%%%%%%%%%%
%%                      PRINCIPAL INVESTIGATOR
%
% Name and address of Principal Investigator (PI)
% 
% NB: The PI has full responsibility for the content of this proposal!
%
% Please fill in the appropriate { } below:
%
\piname{}               % name of PI
\piinst{}               % PI institute
\picoun{SE}               % PI country (ISO code: DK,FI,IS,NO,SE,..) 
\piaddr{}               % PI postal address
\piteln{}               % PI telephone number
\pifaxn{}               % PI fax number
\pimail{}               % PI email address
%%%%%%%%%%%%%%%%%%%%%%%%%%%%%%%%%%%%%%%%%%%%%%%%%%%%%%%%%%%%%%%%%%%%%%%%%%%%%
\end{adrinvpage}


\begin{coinvestpage}{}
%%%%%%%%%%%%%%%%%%%%%%%%%%%%%%%%%%%%%%%%%%%%%%%%%%%%%%%%%%%%%%%%%%%%%%%%%%%%%
%%                          CO-INVESTIGATORS
% 
% Name and institute of co-investigators
% Please fill in the { } { } fields below (2 Co-Is per line):
%
% There is room for up to 10 CoIs. Even if the project involves more 
% than 10 CoIs, please do not list more than 10 CoIs 
%
%        {Name1, Institute1}  {Name2, Institute2}
%
\coinvest{ }{ }  % {Name1, Institute1}   {Name2, Institute2} 
\coinvest{ }{ }  % {Name3, Institute3}   etc
\coinvest{ }{ }
\coinvest{ }{ }
\coinvest{ }{ }
% 
%%%%%%%%%%%%%%%%%%%%%%%%%%%%%%%%%%%%%%%%%%%%%%%%%%%%%%%%%%%%%%%%%%%%%%%%%%%%%
\end{coinvestpage}


\begin{omthesispage}{}
%%%%%%%%%%%%%%%%%%%%%%%%%%%%%%%%%%%%%%%%%%%%%%%%%%%%%%%%%%%%%%%%%%%%%%%%%%%%%
%%                        THESIS PROJECTS
%
% If this proposal concerns a PhD thesis work at Nordic Institute,
% please provide: name of the student, institute, name of supervisor, 
% and expected time of completion.
%
% Please type in the { } field below:
\thesis{} 
%%%%%%%%%%%%%%%%%%%%%%%%%%%%%%%%%%%%%%%%%%%%%%%%%%%%%%%%%%%%%%%%%%%%%%%%%%%%%
\end{omthesispage}


\begin{nightspage}{}
%%%%%%%%%%%%%%%%%%%%%%%%%%%%%%%%%%%%%%%%%%%%%%%%%%%%%%%%%%%%%%%%%%%%%%%%%%%%%
%%                       REQUESTED OBSERVING RUN(S)
%% 
%% NB: In the following, an ``Observing run'' is a single, contiguous block 
%% of time with a single instrument. If your project requires more than one
%% such run, e.g. at different times and/or with different instruments, 
%% please identify each run as A, B, C,... and specify your requirements 
%% for each on a separate line as specified below.
%%
%%%%%%%%%%%%%%%%%%%%%%%%%%%%%%%%%%%%%%%%%%%%%%%%%%%%%%%%%%%%%%%%%%%%%%%%%%%%%
%%
%% Give requested no. of nights/hours as a number (not word) and specify
%% the unit: as N (nights) or H (hours) 
%%
%% Indicate desired Moon phases as D=dark/G=grey/N=no restriction
%% 
%% Indicate the seeing requirements: 0.7 (max 0.7 arcsec seeing), 
%%      1.0 (max 1.0 arcsec), 1.3 (max 1.3 arcsec), or N (no restriction)
%%
%% If the programme require specific sky condition, enter these here
%%      P = photometric conditions required
%%      C = clear conditions required
%%      T = thin clouds/cirrus acceptable
%% 
%% Please fill in the relevant information in the {   } fields below, and 
%% duplicate the entire block if more than one run is requested
%%
%% N.B. A maximum of 6 runs per proposal can be specified
%%
%%%%%%%%%%%%%%%%%%%%%%%%%%%%%%%%%%%%%%%%%%%%%%%%%%%%%%%%%%%%%%%%%%%%%%%%%%%%%
%%
%%%%%%%%%%%%%%%%%%%%%%%%%%%%%% Run A %%%%%%%%%%%%%%%%%%%%%%%%%%%%%%%%%%%%%%%%
%
\nrunid{A }      % Put your run id (A, B, C, ...) here
\ninstr{ALFOSC }      % Put instrument name here
\ntimer{XXX H }      % Put requested time, in numbers, with unit (N or H), e.g 5 N
\nmonth{May }      % Put preferred month(s) here
\nmoonp{B }      % Put requested moon phase here: D, G or N
\nsemax{ }      % Put maximum allowed seeing here: (0.7,1.0,1.5,N)
\nskyco{ }      % Put required photometric condition here
%
%%%%%%%%%%%%%%%%%%%%%%%%%%%%%%%%%%%%%%%%%%%%%%%%%%%%%%%%%%%%%%%%%%%%%%%%%%%%%
\end{nightspage}

\begin{numnightspage}{}
%%%%%%%%%%%%%%%%%%%%%%%%%%%%%%%%%%%%%%%%%%%%%%%%%%%%%%%%%%%%%%%%%%%%%%%%%%%%%
%%                  TIME BEFORE/AFTER PRESENT REQUEST
%%
% Number of nights already awarded to project. More details, e.g. on 
% instrumentation and outcome of previous observations can be given in box 17
%
% Please type in the { } field below:
% 
\numalr{}
%
% Number of nights needed to complete project (excluding those requested).
%
\numrem{}
%%%%%%%%%%%%%%%%%%%%%%%%%%%%%%%%%%%%%%%%%%%%%%%%%%%%%%%%%%%%%%%%%%%%%%%%%%%%%
\end{numnightspage}


\begin{servicepage}[][]{}
%%%%%%%%%%%%%%%%%%%%%%%%%%%%%%%%%%%%%%%%%%%%%%%%%%%%%%%%%%%%%%%%%%%%%%%%%%%%%
%%                         SERVICE CONSTRAINTS
%
%
% All projects will be considered for execution in service/queue mode. In
% case your project can not be done in service/queue mode, please give a
% justification in the { } field below:
\service{ }
%%%%%%%%%%%%%%%%%%%%%%%%%%%%%%%%%%%%%%%%%%%%%%%%%%%%%%%%%%%%%%%%%%%%%%%%%%%%%
\end{servicepage}


\begin{schedpage}[][]{}
%%%%%%%%%%%%%%%%%%%%%%%%%%%%%%%%%%%%%%%%%%%%%%%%%%%%%%%%%%%%%%%%%%%%%%%%%%%%%
%%                         SCHEDULING CONSTRAINTS
%
%
% Any other special constraints on the scheduling?
%
% E.g. time critical scheduling, or required baseline vs phase coverage
% for monitoring programs, response time for target of opportunity, 
% simultaneous observations, impossible dates, etc... 
%
% Please type in the { } field below:
%
\schedconstr{ }
%%%%%%%%%%%%%%%%%%%%%%%%%%%%%%%%%%%%%%%%%%%%%%%%%%%%%%%%%%%%%%%%%%%%%%%%%%%%%
\end{schedpage}


%%%%%%%%%%%%%%%%%%%%%%%%%%%%%%%%%%%%%%%%%%%%%%%%%%%%%%%%%%%%%%%%%%%%%%%%%%%%%
%%
%%              !!!!!!!!!!!!! PAGE 2 !!!!!!!!!!!!!
%%
%%%%%%%%%%%%%%%%%%%%%%%%%%%%%%%%%%%%%%%%%%%%%%%%%%%%%%%%%%%%%%%%%%%%%%%%%%%%%
\newpage        % Page 2  
\putnumberLarge

\begin{scienpage}[][]{}
%%%%%%%%%%%%%%%%%%%%%%%%%%%%%%%%%%%%%%%%%%%%%%%%%%%%%%%%%%%%%%%%%%%%%%%%%%%%%
%%                     SCIENTIFIC JUSTIFICATION                            %%
%%                                                                         %%
%% Note: This should be self-contained and not refer to previous proposals.%%
%%                                                                         %%
%% Describe first the scientific background and main goals of the proposal.%%
%% As OPC members cannot be experts in every field, it is CRUCIAL that you %% 
%% outline the general scientific context CLEARLY and in a manner that can %% 
%% be understood also by a non-specialist in your field.                   %% 
%%                                                                         %% 
%% Then argue - equally clearly! - how your proposed observing project     %% 
%% will contribute significantly to advancing our understanding of the     %% 
%% issue. Key references to the literature should be given.                %% 
%%                                                                         %%
%% Finally, describe how the data reduction and scientific analysis will   %%
%% be done, and document that the team possesses the required expertise.   %%
%%                                                                         %%
%% All text and figures should fit on the following two pages (page 2 and  %%
%% page 3), but text may spill over on page 3. All figures and references  %%
%% should be placed on page 3.                                             %%
%%                                                                         %%
%%%%%%%%%%%%%%%%%%%%%%%%%%%%%%%%%%%%%%%%%%%%%%%%%%%%%%%%%%%%%%%%%%%%%%%%%%%%%
%%
% Please type your text into the { } field below
\scijust{Dark matter was first termed in a paper from 1933 [ref] by Fritz
Zwicky. He used the virial theorem to calculate the gravitational mass of
the galaxies in the Coma cluster and found a discrepancy between the
measured mass and their expected luminosity. He referred to this
"missing mass" as "dunkle materie". Today astronomers have accumulated
convincing evidence of dark matter from independent observations such
as galaxy rotation curves, gravitational lensing, measurements of
the cosmic microwave background, baryon acoustic oscillations
, supernovae distance measurements, Lyman-alpha forest measurements 
of distant galaxies and in structure formation scenarios.
\par
According to the spectacularly successful Planck mission, the dark matter
part of the energy in the universe is a staggering 26.8\%
compared with the 4.9\% of ordinary matter. Even though the consensus among scientist
today is that dark matter consists of Weakly Interacting Massive Particles
(WIMPs), no official detections of these elusive particles have been made and
the hunt for these particles is one of the major undertakings of modern physics.
In what better way to make aspiring student of astronomy more comfortable
with observational instruments, than for them to "see" for
themselves what the "fuss" is all about? The reproducibility of science is
after all one of the fundamental pillars of science itself.
By the guidance of past and present mentors we therefore propose to use the
NOT telescope
to measure the rotation curve of NGC 5963, fit a light+dark mass profile 
to the acquired data and determine the stellar/dark matter mass components of
this galaxy. \\

\noindent \textbf{The need for a new observation and its selection:} The target in consideration, NGC 5963 is of the type Low Surface Brightness (LSB) galaxy (Romanishin, Strom \& Strom 1982 ApJ 252, 77) which are usual targets for dark matter studies due to their peculiar mass to light ratio. NGC 5963 is no exception to such studies (e.g. Bosma et al. 1988, A\& A 198, 100). However, the latest  direct observations of NGC 5963 we could find in the literature were taken over a decade ago (Simon et al. 2004 ASPC, 327, 18), which speaks for the acquisition of newer observations. 
\par
Another virtue of the selected target is that if photometric images of good enough quality of the galaxy is provided, they might give insight to its anomaly underluminous nature (Zackrisson et al. in preparation). NGC 5963 strongly deviates from the expected Tully-Fisher (TF) relation (Springob et al. 2007 ApJS, 172, 599) by being underluminous and/or having far greater non-baryonic mass than expected. Newer observations with the NOT may aid in uncovering why this is so.
 
   }
%%%%%%%%%%%%%%%%%%%%%%%%%%%%%%%%%%%%%%%%%%%%%%%%%%%%%%%%%%%%%%%%%%%%%%%%%%%%%
\end{scienpage}

%%%%%%%%%%%%%%%%%%%%%%%%%%%%%%%%%%%%%%%%%%%%%%%%%%%%%%%%%%%%%%%%%%%%%%%%%%%%%
%%
%%              !!!!!!!!!!!!! PAGE 3 !!!!!!!!!!!!!
%%
%%%%%%%%%%%%%%%%%%%%%%%%%%%%%%%%%%%%%%%%%%%%%%%%%%%%%%%%%%%%%%%%%%%%%%%%%%%%%
\newpage        % Page 3 
\putnumberLarge

\begin{scienpagec}[][]{}
%%%%%%%%%%%%%%%%%%%%%%%%%%%%%%%%%%%%%%%%%%%%%%%%%%%%%%%%%%%%%%%%%%%%%%%%%%%%%
%%              SCIENTIFIC JUSTIFICATION (CONTINUED)                       %%
%%                                                                         %%
%% Place References and any Figures here.                                  %%
%%                                                                         %%
%%%%%%%%%%%%%%%%%%%%%%%%%%%%%%%%%%%%%%%%%%%%%%%%%%%%%%%%%%%%%%%%%%%%%%%%%%%%%
% 
% Please type the rest of your text into the { } field below:
\scijustc{ }
%
%%%%%%%%%%%%%%%%%%%%%%%%%%%%%%%%%%%%%%%%%%%%%%%%%%%%%%%%%%%%%%%%%%%%%%%%%%%%%
%%                                FIGURES:                                 %%
%%                                                                         %%
%% Up to two postscript figures may be included                            %%
%%                                                                         %%
%% NB colour figures are not supported. All figures will be printed in     %%
%%    black and white and any colour information in the figures will be    %%
%%    disregarded                                                          %%
%%                                                                         %%
%% To enter a figure, uncomment the lines below, fill in the name of       %%
%% your .eps file, and provide a short caption where indicated             %%
%%                                                                         %%
%%%%%%%%%%%%%%%%%%%%%%%%%%%%%%%%%%%%%%%%%%%%%%%%%%%%%%%%%%%%%%%%%%%%%%%%%%%%%
%%
%%
%% NB: There should be no spaces in the \psfig argument below 
%% -  otherwise psfig will fail!
%%
%
%   Figure 1:   change 'PInameA.eps' to name of the file containing your 
%               figure. You may need to adjust the width and angle below,
%               and possibly the bounding box in the postscript-file.
%               Provide a short caption in the \captone{} field. Be sure 
%               to leave an empty line between the psfig and caption
%               command
%
%\psfig{file=PInameA.eps,width=12cm,angle=0,clip=}
%
%\captone{}
%
%
%   Figure 2:   change 'PInameB.eps' to name of the file containing your 
%               figure. You may need to adjust the width and angle below,
%               and possibly the bounding box in the postscript-file.
%               Provide a short caption in the \capttwo{} field. Be sure 
%               to leave an empty line between the psfig and caption
%               command
%
%\psfig{file=PInameB.eps,width=12cm,angle=0,clip=}
%
%\capttwo{}
%
%
%%%%%%%%%%%%%%%%%%%%%%%%%%%%%%%%%%%%%%%%%%%%%%%%%%%%%%%%%%%%%%%%%%%%%%%%%%%%%
\end{scienpagec}



%%%%%%%%%%%%%%%%%%%%%%%%%%%%%%%%%%%%%%%%%%%%%%%%%%%%%%%%%%%%%%%%%%%%%%%%%%%%%
%%
%%              !!!!!!!!!!!!! PAGE 4 !!!!!!!!!!!!!
%%
%%%%%%%%%%%%%%%%%%%%%%%%%%%%%%%%%%%%%%%%%%%%%%%%%%%%%%%%%%%%%%%%%%%%%%%%%%%%%
\newpage  % Page 4 
\putnumberLarge

\begin{justificpage}[][]{}
%%%%%%%%%%%%%%%%%%%%%%%%%%%%%%%%%%%%%%%%%%%%%%%%%%%%%%%%%%%%%%%%%%%%%%%%%%%%%
%%                   TECHNICAL JUSTIFICATION
%
%  Describe how the observations will be performed, so the feasibility of the 
%  project becomes clear. This should be self-contained and not rely on 
%  information given elsewhere. Describe the S/N ratio calculations you have 
%  used to justify the number of nights and the Moon phases you request 
%  (no correction for expected weather conditions should be applied). Also,
%  describe why the instrumental set-up is adequate for the objective of the
%  proposed observations (e.g., if the resolution of spectra is adequate to 
%  resolve the spectroscopic features you want to study).
%
% Please type in the { } field below:
\techjust{ }
%%%%%%%%%%%%%%%%%%%%%%%%%%%%%%%%%%%%%%%%%%%%%%%%%%%%%%%%%%%%%%%%%%%%%%%%%%%%%
\end{justificpage}




\begin{instrempage}[][]{}
%%%%%%%%%%%%%%%%%%%%%%%%%%%%%%%%%%%%%%%%%%%%%%%%%%%%%%%%%%%%%%%%%%%%%%%%%%%%%
%
%
%                       INSTRUMENT CONFIGURATIONS:
%
% Please specify the instrument configurations you want to use as fully as 
% possible, using the setup definitions provided below. You must uncomment
% the relevant setup definitions, i.e. remove the `%' sign in front of them,
% in order to make these lines take effect.
%
% Uncomment only the lines related to instrument configuration(s) needed
% for the acquisition of your planned  observations. Detailed information 
% on the available instruments is provided in the relevant users' manuals 
% (see http://www.not.iac.es/observing/proposals/).
%
% You also need to specify the Run ids (as defined in box 6) for which the 
% selected setups are valid. Put the run ID in the first {}, for example:
%
%       \NOTconfig{A}{ALFOSC}{Standard imaging-filters}{UBVRi}
%
%
%-----------------------------------------------------------------------
%----------------------------- ALFOSC ----------------------------------
%-----------------------------------------------------------------------
%
\NOTconfig{}{ALFOSC}{Standard imaging-filters}{UBVRi}
%\NOTconfig{}{ALFOSC}{Imaging-filters for ALFOSC}{provide filter No.}
%\NOTconfig{}{ALFOSC}{Imaging-filters for FASU}{provide filter No.}
%
%\NOTconfig{}{ALFOSC}{Fast-photometry}{Multi-windowing mode}
%
%\NOTconfig{}{ALFOSC}{Lin-Pol-imaging}{Polaroids}
%\NOTconfig{}{ALFOSC}{Lin-Pol-imaging}{Calcite+half-wave-plate}
%\NOTconfig{}{ALFOSC}{Cir-Pol-imaging}{Calcite+quarter-wave-plate}
%\NOTconfig{}{ALFOSC}{Lin-Pol-imaging}{WeDoWo}
%
%\NOTconfig{}{ALFOSC}{Lin-Pol-spectroscopy}{Calcite+half-wave-plate}
%\NOTconfig{}{ALFOSC}{Cir-Pol-spectroscopy}{Calcite+quarter-wave-plate}
%\NOTconfig{}{ALFOSC}{Pol-spectroscopy}{give grism number(s)}
%\NOTconfig{}{ALFOSC}{Pol-spectroscopy}{Polarimetric slitlet\#1.0"}
%\NOTconfig{}{ALFOSC}{Pol-spectroscopy}{Polarimetric slitlet\#1.4"}
%\NOTconfig{}{ALFOSC}{Pol-spectroscopy}{Polarimetric slitlet\#1.8"}
%
%\NOTconfig{}{ALFOSC}{Spectroscopy}{ADC}
%
%\NOTconfig{}{ALFOSC}{Spectro-long-slit}{give grism number(s)}
\NOTconfig{}{ALFOSC}{Spectro-long-slit}{1.0, 1.3}
\NOTconfig{}{ALFOSC}{Spectro-long-slit}{Grism\#8}
%\NOTconfig{}{ALFOSC}{Spectro-long-slit}{provide 2nd-order blocking filter No.}
%
%\NOTconfig{}{ALFOSC}{Multi-Object-Spectro}{provide HERE the number of masks}
%\NOTconfig{}{ALFOSC}{Multi-Object-Spectro}{give grism number(s)}
%\NOTconfig{}{ALFOSC}{Multi-Object-Spectro}{provide required slitwidth(s)}
%\NOTconfig{}{ALFOSC}{Multi-Object-Spectro}{Pre-imaging required}
%
%\NOTconfig{}{ALFOSC}{Spectro-Echelle}{Echelle Grism\#9}
%\NOTconfig{}{ALFOSC}{Spectro-Echelle}{Echelle Grism\#13}
%\NOTconfig{}{ALFOSC}{Spectro-Echelle}{0.7arcsec-slit}
%\NOTconfig{}{ALFOSC}{Spectro-Echelle}{0.8arcsec-slit}
%\NOTconfig{}{ALFOSC}{Spectro-Echelle}{1.0arcsec-slit}
%\NOTconfig{}{ALFOSC}{Spectro-Echelle}{1.2arcsec-slit}
%\NOTconfig{}{ALFOSC}{Spectro-Echelle}{1.6arcsec-slit}
%\NOTconfig{}{ALFOSC}{Spectro-Echelle}{1.8arcsec-slit}
%\NOTconfig{}{ALFOSC}{Spectro-Echelle}{2.2arcsec-slit}
%\NOTconfig{}{ALFOSC}{Spectro-Echelle}{Cross-disperser\#10}
%\NOTconfig{}{ALFOSC}{Spectro-Echelle}{Cross-disperser\#11}
%\NOTconfig{}{ALFOSC}{Spectro-Echelle}{Cross-disperser\#12}
%
%
%-----------------------------------------------------------------------
%----------------------------- NOTCam-----------------------------------
%-----------------------------------------------------------------------
%
%\NOTconfig{}{NOTCam}{Standard imaging-filters}{ZYJHKs}
%\NOTconfig{}{NOTCam}{Imaging-filters for NOTCam}{provide filter No.}
%
%\NOTconfig{}{NOTCam}{Imaging}{Wide-field camera}
%\NOTconfig{}{NOTCam}{Imaging}{High-res camera}
%
%\NOTconfig{}{NOTCam}{Spectro-long-slit}{Grism\#1}
%\NOTconfig{}{NOTCam}{Spectro-long-slit}{Wide-field camera slit 0.6''}
%\NOTconfig{}{NOTCam}{Spectro-long-slit}{High-res camera slit 0.2''}
%
%\NOTconfig{}{NOTCam}{Spectroscopy}{ADC}
%
%\NOTconfig{}{NOTCam}{Spectroscopy}{Wide-field camera}
%\NOTconfig{}{NOTCam}{Spectroscopy}{High-res camera}
%
%
%-----------------------------------------------------------------------
%----------------------------- FIES ------------------------------------
%-----------------------------------------------------------------------
%
% Please note that the simultaneous-ThAr mode can only be used in
% combination with either the High-res or Med-res fiber.
%
% Please note that the spec-pol mode can only be used in combination
% with the Med-Res fiber. The useful wavelength range is 370-630 nm.
%
%\NOTconfig{}{FIES}{Spectro-Echelle}{Low-Res Fiber}
%\NOTconfig{}{FIES}{Spectro-Echelle}{Med-Res Fiber}
%\NOTconfig{}{FIES}{Spectro-Echelle}{High-Res Fiber}
%
%\NOTconfig{}{FIES}{Spectro-Echelle}{Simultaneous-ThAr mode}
%
%\NOTconfig{}{FIES}{Spectro-Echelle}{Spec-Pol mode}
%
%\NOTconfig{}{FIES}{Spectro-Echelle}{ADC}
%
%
%-----------------------------------------------------------------------
%----------------------------- MOSCA -----------------------------------
%-----------------------------------------------------------------------
%
%\NOTconfig{}{MOSCA}{Standard imaging-filters}{UBVRI}
%\NOTconfig{}{MOSCA}{Standard imaging-filters}{ugriz}
%\NOTconfig{}{MOSCA}{Imaging-filters for FASU}{provide filter No.}
%
%
%-----------------------------------------------------------------------
%----------------------------- StanCam ---------------------------------
%-----------------------------------------------------------------------
%
%\NOTconfig{}{StanCam}{Standard imaging-filters}{UBVRIz}
%\NOTconfig{}{StanCam}{Imaging-filters for StanCam}{provide filter No.}
%
%
%-----------------------------------------------------------------------
%----------------------------- TurPol ----------------------------------
%-----------------------------------------------------------------------
%
% TurPol is not a common-user instrument. Normal support at the telescope 
% is provided, but only limited trouble-shooting can be made by NOT staff.
%
%\NOTconfig{}{TurPol}{Lin-Pol}{Half-wave-plate}
%\NOTconfig{}{TurPol}{Cir+Lin-Pol}{Quarter-wave-plate}
%\NOTconfig{}{TurPol}{Standard photometry}{UBVRI}
% 
%
%-----------------------------------------------------------------------
%----------------------------- Visitor Instruments ---------------------
%-----------------------------------------------------------------------
%
% If you wish to use a special instrument for the project, please contact
% the NOT director (Johannes Andersen, email: ja@astro.ku.dk), or the Head
% of Operations (Thomas Augusteijn, email: tau@not.iac.es) well in advance
% of submitting your proposal.
%
%\NOTconfig{}{Name of instrument}{key capability}{link to instrument URL}
%
%
%%%%%%%%%%%%%%%%%%%%%%%%%%%%%%%%%%%%%%%%%%%%%%%%%%%%%%%%%%%%%%%%%%%%%%%%%%%%%
%%                      REMARKS ON INSTRUMENT SETUP
%%
% If you have any special remarks or requirements, e.g. use of non standard
% observing modes, own equipment etc., please uncomment the %\remark{} line 
% below and provide the information in the {} field. 
% If necessary, more detailed information can be provided in BOX 18.
%
%\remark{} 
%%%%%%%%%%%%%%%%%%%%%%%%%%%%%%%%%%%%%%%%%%%%%%%%%%%%%%%%%%%%%%%%%%%%%%%%%%%%%
\end{instrempage}


%%%%%%%%%%%%%%%%%%%%%%%%%%%%%%%%%%%%%%%%%%%%%%%%%%%%%%%%%%%%%%%%%%%%%%%%%%%%%
%%
%%              !!!!!!!!!!!!! PAGE 5 !!!!!!!!!!!!!
%%
%%%%%%%%%%%%%%%%%%%%%%%%%%%%%%%%%%%%%%%%%%%%%%%%%%%%%%%%%%%%%%%%%%%%%%%%%%%%%
\newpage        % Page 5 
\putnumberLarge

\begin{objlistpage}{}
%%%%%%%%%%%%%%%%%%%%%%%%%%%%%%%%%%%%%%%%%%%%%%%%%%%%%%%%%%%%%%%%%%%%%%%%%%%%%
%%                            TARGET LIST                                  %% 
%% Target list with coordinates, or intervals in R.A. and Decl.            %%
%% of (sample of) objects:                                                 %%
%%                                                                         %%
%%%%%%%%%%%%%%%%%%%%%%%%%%%%%%%%%%%%%%%%%%%%%%%%%%%%%%%%%%%%%%%%%%%%%%%%%%%%%
%%                                                                         %%
%% Please fill in the relevant information for one object in the {   }     %%
%% fields below. Duplicate the line up to 20 objects; otherwise give a     %%
%% sample list and describe the full target list in the Remarks.           %%
%%                                                                         %%
%% Please specify the passband in which object magnitudes are given        %%
%%                                                                         %%
%% Targets are specified with the \target{#1}{#2}{#3}{#4}{#5}{#6}{#7}      %%
%% command, where the 7 arguments are:                                     %%
%%                                                                         %%
%%  #1: Run IDs (as defined box 6) for which target is relevant            %%
%%  #2: Short target name                                                  %%
%%  #3: Right ascension (2000)                                             %%
%%  #4: Declination (2000)                                                 %%
%%  #5: Target magnitude                                                   %%
%%  #6: Target diameter (in arcmin)                                        %% 
%%  #7: Additional Info                                                    %%
%%                                                                         %%
%%   Example:                                                              %%
%%                                                                         %%
%%   \target{A}{M31}{00:42:44}{+41:16:09}{V=4.4}{190}{$v_r=-300$ km/s}     %%
%%                                                                         %%
%%%%%%%%%%%%%%%%%%%%%%%%%%%%%%%%%%%%%%%%%%%%%%%%%%%%%%%%%%%%%%%%%%%%%%%%%%%%%
\target{A}{NGC 5963}{15 33 27.8}{+56 33 35}{B=12.70}{1.52}{} % duplicate as needed
%%%%%%%%%%%%%%%%%%%%%%%%%%%%%%%%%%%%%%%%%%%%%%%%%%%%%%%%%%%%%%%%%%%%%%%%%%%%%
%% Please type any Remarks on the target list in the {   } field below: 
%%
\remark{ }
%%%%%%%%%%%%%%%%%%%%%%%%%%%%%%%%%%%%%%%%%%%%%%%%%%%%%%%%%%%%%%%%%%%%%%%%%%%%%
\end{objlistpage}

\begin{qualitypage}[][]{}
%%%%%%%%%%%%%%%%%%%%%%%%%%%%%%%%%%%%%%%%%%%%%%%%%%%%%%%%%%%%%%%%%%%%%%%%%%%%%
%%                       BACKUP PROGRAMME
%%
% For projects needing excellent image quality or photometric conditions, 
% give a short description of possible backup programme. A backup programme 
% is also needed in case an unfavourable wind speed or direction may prevent
% you from executing your main programme.    
%
% - If no backup programme is provided, justify why none is needed. Failure
%   to ensure that the telescope will be used productively under all 
%   circumstances will lower the rating of your proposal!
%
% - The instrumental setup for the backup programme should normally be the 
%   same as for the main programme; however, standby instrumentation (e.g. 
%   StanCam or FIES) may be used instead. Please specify the instrument you 
%   will use.
% 
% Please type your description in the { } field below:
%
\backup{ }
%%%%%%%%%%%%%%%%%%%%%%%%%%%%%%%%%%%%%%%%%%%%%%%%%%%%%%%%%%%%%%%%%%%%%%%%%%%%%
\end{qualitypage}


%%%%%%%%%%%%%%%%%%%%%%%%%%%%%%%%%%%%%%%%%%%%%%%%%%%%%%%%%%%%%%%%%%%%%%%%%%%%%
%%
%%           !!!!!!!!!  PAGE 6 = Last page  !!!!!!!!!!!!!
%%
%%%%%%%%%%%%%%%%%%%%%%%%%%%%%%%%%%%%%%%%%%%%%%%%%%%%%%%%%%%%%%%%%%%%%%%%%%%%%
\newpage        % Page 6 
\putnumberLarge

\begin{prevobspage}[][]{}
%%%%%%%%%%%%%%%%%%%%%%%%%%%%%%%%%%%%%%%%%%%%%%%%%%%%%%%%%%%%%%%%%%%%%%%%%%%%%
%%              PREVIOUS OBSERVING PERIODS AND RESULTS
%%
% Please list your observing periods at the NOT within the last three years 
% and provide a brief status report on your analysis of the data. List also 
% your publications from NOT observations during the last three years. 
%
%%  Please type this information in the { } field below:
%% 
\prevobs{ }
%%%%%%%%%%%%%%%%%%%%%%%%%%%%%%%%%%%%%%%%%%%%%%%%%%%%%%%%%%%%%%%%%%%%%%%%%%%%%
\end{prevobspage}

\begin{extrapage}[][]{}
%%%%%%%%%%%%%%%%%%%%%%%%%%%%%%%%%%%%%%%%%%%%%%%%%%%%%%%%%%%%%%%%%%%%%%%%%%%%%
%%              ADDITIONAL INFO
%%
% Use this box to provide any additional information/remarks not covered 
% by the items above, e.g. related proposals to other telescopes, dates to 
% be avoided for the observations, etc. Do not use if not necessary.  
%
%%  Please type in the { } field below:
%% 
\addinfo{ }
%%%%%%%%%%%%%%%%%%%%%%%%%%%%%%%%%%%%%%%%%%%%%%%%%%%%%%%%%%%%%%%%%%%%%%%%%%%%%
\end{extrapage}


\end{document}
%%%%%%%%%%%%%%%%%%%%%%%%%%%%%%%%%%%%%%%%%%%%%%%%%%%%%%%%%%%%%%%%%%%%%%%%%%%%%
% 
%                       End of template file
% 
%%%%%%%%%%%%%%%%%%%%%%%%%%%%%%%%%%%%%%%%%%%%%%%%%%%%%%%%%%%%%%%%%%%%%%%%%%%%%

                    
\begin{document}
\def\propnumber{}                       % Only to be filled in by NOT staff
\putnumberHuge
\nothead

%%%%%%%%%%%%%%%%%%%%%%%%%%%%%%%%%%%%%%%%%%%%%%%%%%%%%%%%%%%%%%%%%%%%%%%%%%%%%
%%                                                                         %%
%%                    *** NOTE TO APPLICANTS ***                           %%
%%                                                                         %%
%%%%%%%%%%%%%%%%%%%%%%%%%%%%%%%%%%%%%%%%%%%%%%%%%%%%%%%%%%%%%%%%%%%%%%%%%%%%%
%%                                                                         %%
%%          NOT PROPOSAL TEMPLATE FILE FOR OBSERVING PERIOD 48             %%
%%                                                                         %%
%%                   OCTOBER 1, 2013 - APRIL 1, 2014                       %%
%%                                                                         %%
%%                                                                         %%
%% Please take care to fill in the fields of this proposal form as         %%
%% indicated, following the instructions and advice provided in the        %%
%% header of each section. Run LaTex on your completed form and verify     %%
%% the result before submitting to check that it runs correctly and        %%
%% do not overfill any of the boxes, and that it produces a total of no    %%
%% more than six (6) printed pages.                                        %%
%%                                                                         %%
%% Be sure to always use the correct version of the not-style file for     %%
%% the period in question. This template is only valid for period 48.      %%
%%                                                                         %%
%% Never change the format of the template or style file. Proposals        %%
%% that do not comply with the correct version of the style and            %%
%% template files will be rejected.                                        %%
%%                                                                         %%
%% Name the file PIname.tex (e.g. johanson.tex) and any figure file(s)     %%
%% as PInameA.ps (and PInameB.ps). After verifying that the proposal       %%
%% can be properly processed, submit the file(s) as (separate) attached    %%
%% file(s) by e-mail to the address:                                       %%
%%                                                                         %%
%%                          proposal@not.iac.es                            %%
%%                                                                         %%
%% with the word ``Proposal'' in the 'Subject' field and as text in the    %%
%% body of the message. The latter is important when you use certain       %%
%% mailers as the proposal might otherwise not be parsed correctly by      %%
%% our automatic procedure.                                                %%
%%                                                                         %%
%% Do not compress the files or combine them in a tar file. Do not         %%
%% submit the style file.                                                  %%
%%                                                                         %%
%% Any questions regarding the proposal procedure may be submitted to      %%
%% the same e-mail address, giving ``Question'' as the 'Subject'.          %%
%%                                                                         %%
%% If you submit more than one proposal, please name the file              %%
%% PIname1.tex, PIname2.tex, etc., and any figure files accordingly.       %%
%% Only one proposal should be submitted at the time.                      %%
%%                                                                         %%
%% For more information on the Nordic Optical Telescope see:               %%
%%                                                                         %%
%%                       http://www.not.iac.es/                            %%
%%                                                                         %%
%%%%%%%%%%%%%%%%%%%%%%%%%%%%%%%%%%%%%%%%%%%%%%%%%%%%%%%%%%%%%%%%%%%%%%%%%%%%%


\begin{titpage}{}
%%%%%%%%%%%%%%%%%%%%%%%%%%%%%%%%%%%%%%%%%%%%%%%%%%%%%%%%%%%%%%%%%%%%%%%%%%%%%
%                          PROPOSAL TITLE
% Type title of proposal in the { } below - one line only!
%
\proptitle{Measuring the Rotation Curve of the Elusive NGC 5963: The Adventure.}
%%%%%%%%%%%%%%%%%%%%%%%%%%%%%%%%%%%%%%%%%%%%%%%%%%%%%%%%%%%%%%%%%%%%%%%%%%%%%
\end{titpage}


\begin{abspage}[][]{}
%%%%%%%%%%%%%%%%%%%%%%%%%%%%%%%%%%%%%%%%%%%%%%%%%%%%%%%%%%%%%%%%%%%%%%%%%%%%%
%%                             ABSTRACT
%
% Please type the Abstract of the proposal into the { } below
% Do not exceed the space provided 
%
\propabstract{}
%%%%%%%%%%%%%%%%%%%%%%%%%%%%%%%%%%%%%%%%%%%%%%%%%%%%%%%%%%%%%%%%%%%%%%%%%%%%%
\end{abspage}


\begin{adrinvpage}{}
%%%%%%%%%%%%%%%%%%%%%%%%%%%%%%%%%%%%%%%%%%%%%%%%%%%%%%%%%%%%%%%%%%%%%%%%%%%%%
%%                      PRINCIPAL INVESTIGATOR
%
% Name and address of Principal Investigator (PI)
% 
% NB: The PI has full responsibility for the content of this proposal!
%
% Please fill in the appropriate { } below:
%
\piname{}               % name of PI
\piinst{}               % PI institute
\picoun{SE}               % PI country (ISO code: DK,FI,IS,NO,SE,..) 
\piaddr{}               % PI postal address
\piteln{}               % PI telephone number
\pifaxn{}               % PI fax number
\pimail{}               % PI email address
%%%%%%%%%%%%%%%%%%%%%%%%%%%%%%%%%%%%%%%%%%%%%%%%%%%%%%%%%%%%%%%%%%%%%%%%%%%%%
\end{adrinvpage}


\begin{coinvestpage}{}
%%%%%%%%%%%%%%%%%%%%%%%%%%%%%%%%%%%%%%%%%%%%%%%%%%%%%%%%%%%%%%%%%%%%%%%%%%%%%
%%                          CO-INVESTIGATORS
% 
% Name and institute of co-investigators
% Please fill in the { } { } fields below (2 Co-Is per line):
%
% There is room for up to 10 CoIs. Even if the project involves more 
% than 10 CoIs, please do not list more than 10 CoIs 
%
%        {Name1, Institute1}  {Name2, Institute2}
%
\coinvest{ }{ }  % {Name1, Institute1}   {Name2, Institute2} 
\coinvest{ }{ }  % {Name3, Institute3}   etc
\coinvest{ }{ }
\coinvest{ }{ }
\coinvest{ }{ }
% 
%%%%%%%%%%%%%%%%%%%%%%%%%%%%%%%%%%%%%%%%%%%%%%%%%%%%%%%%%%%%%%%%%%%%%%%%%%%%%
\end{coinvestpage}


\begin{omthesispage}{}
%%%%%%%%%%%%%%%%%%%%%%%%%%%%%%%%%%%%%%%%%%%%%%%%%%%%%%%%%%%%%%%%%%%%%%%%%%%%%
%%                        THESIS PROJECTS
%
% If this proposal concerns a PhD thesis work at Nordic Institute,
% please provide: name of the student, institute, name of supervisor, 
% and expected time of completion.
%
% Please type in the { } field below:
\thesis{} 
%%%%%%%%%%%%%%%%%%%%%%%%%%%%%%%%%%%%%%%%%%%%%%%%%%%%%%%%%%%%%%%%%%%%%%%%%%%%%
\end{omthesispage}


\begin{nightspage}{}
%%%%%%%%%%%%%%%%%%%%%%%%%%%%%%%%%%%%%%%%%%%%%%%%%%%%%%%%%%%%%%%%%%%%%%%%%%%%%
%%                       REQUESTED OBSERVING RUN(S)
%% 
%% NB: In the following, an ``Observing run'' is a single, contiguous block 
%% of time with a single instrument. If your project requires more than one
%% such run, e.g. at different times and/or with different instruments, 
%% please identify each run as A, B, C,... and specify your requirements 
%% for each on a separate line as specified below.
%%
%%%%%%%%%%%%%%%%%%%%%%%%%%%%%%%%%%%%%%%%%%%%%%%%%%%%%%%%%%%%%%%%%%%%%%%%%%%%%
%%
%% Give requested no. of nights/hours as a number (not word) and specify
%% the unit: as N (nights) or H (hours) 
%%
%% Indicate desired Moon phases as D=dark/G=grey/N=no restriction
%% 
%% Indicate the seeing requirements: 0.7 (max 0.7 arcsec seeing), 
%%      1.0 (max 1.0 arcsec), 1.3 (max 1.3 arcsec), or N (no restriction)
%%
%% If the programme require specific sky condition, enter these here
%%      P = photometric conditions required
%%      C = clear conditions required
%%      T = thin clouds/cirrus acceptable
%% 
%% Please fill in the relevant information in the {   } fields below, and 
%% duplicate the entire block if more than one run is requested
%%
%% N.B. A maximum of 6 runs per proposal can be specified
%%
%%%%%%%%%%%%%%%%%%%%%%%%%%%%%%%%%%%%%%%%%%%%%%%%%%%%%%%%%%%%%%%%%%%%%%%%%%%%%
%%
%%%%%%%%%%%%%%%%%%%%%%%%%%%%%% Run A %%%%%%%%%%%%%%%%%%%%%%%%%%%%%%%%%%%%%%%%
%
\nrunid{A }      % Put your run id (A, B, C, ...) here
\ninstr{ALFOSC }      % Put instrument name here
\ntimer{XXX H }      % Put requested time, in numbers, with unit (N or H), e.g 5 N
\nmonth{May }      % Put preferred month(s) here
\nmoonp{B }      % Put requested moon phase here: D, G or N
\nsemax{ }      % Put maximum allowed seeing here: (0.7,1.0,1.5,N)
\nskyco{ }      % Put required photometric condition here
%
%%%%%%%%%%%%%%%%%%%%%%%%%%%%%%%%%%%%%%%%%%%%%%%%%%%%%%%%%%%%%%%%%%%%%%%%%%%%%
\end{nightspage}

\begin{numnightspage}{}
%%%%%%%%%%%%%%%%%%%%%%%%%%%%%%%%%%%%%%%%%%%%%%%%%%%%%%%%%%%%%%%%%%%%%%%%%%%%%
%%                  TIME BEFORE/AFTER PRESENT REQUEST
%%
% Number of nights already awarded to project. More details, e.g. on 
% instrumentation and outcome of previous observations can be given in box 17
%
% Please type in the { } field below:
% 
\numalr{}
%
% Number of nights needed to complete project (excluding those requested).
%
\numrem{}
%%%%%%%%%%%%%%%%%%%%%%%%%%%%%%%%%%%%%%%%%%%%%%%%%%%%%%%%%%%%%%%%%%%%%%%%%%%%%
\end{numnightspage}


\begin{servicepage}[][]{}
%%%%%%%%%%%%%%%%%%%%%%%%%%%%%%%%%%%%%%%%%%%%%%%%%%%%%%%%%%%%%%%%%%%%%%%%%%%%%
%%                         SERVICE CONSTRAINTS
%
%
% All projects will be considered for execution in service/queue mode. In
% case your project can not be done in service/queue mode, please give a
% justification in the { } field below:
\service{ }
%%%%%%%%%%%%%%%%%%%%%%%%%%%%%%%%%%%%%%%%%%%%%%%%%%%%%%%%%%%%%%%%%%%%%%%%%%%%%
\end{servicepage}


\begin{schedpage}[][]{}
%%%%%%%%%%%%%%%%%%%%%%%%%%%%%%%%%%%%%%%%%%%%%%%%%%%%%%%%%%%%%%%%%%%%%%%%%%%%%
%%                         SCHEDULING CONSTRAINTS
%
%
% Any other special constraints on the scheduling?
%
% E.g. time critical scheduling, or required baseline vs phase coverage
% for monitoring programs, response time for target of opportunity, 
% simultaneous observations, impossible dates, etc... 
%
% Please type in the { } field below:
%
\schedconstr{ }
%%%%%%%%%%%%%%%%%%%%%%%%%%%%%%%%%%%%%%%%%%%%%%%%%%%%%%%%%%%%%%%%%%%%%%%%%%%%%
\end{schedpage}


%%%%%%%%%%%%%%%%%%%%%%%%%%%%%%%%%%%%%%%%%%%%%%%%%%%%%%%%%%%%%%%%%%%%%%%%%%%%%
%%
%%              !!!!!!!!!!!!! PAGE 2 !!!!!!!!!!!!!
%%
%%%%%%%%%%%%%%%%%%%%%%%%%%%%%%%%%%%%%%%%%%%%%%%%%%%%%%%%%%%%%%%%%%%%%%%%%%%%%
\newpage        % Page 2  
\putnumberLarge

\begin{scienpage}[][]{}
%%%%%%%%%%%%%%%%%%%%%%%%%%%%%%%%%%%%%%%%%%%%%%%%%%%%%%%%%%%%%%%%%%%%%%%%%%%%%
%%                     SCIENTIFIC JUSTIFICATION                            %%
%%                                                                         %%
%% Note: This should be self-contained and not refer to previous proposals.%%
%%                                                                         %%
%% Describe first the scientific background and main goals of the proposal.%%
%% As OPC members cannot be experts in every field, it is CRUCIAL that you %% 
%% outline the general scientific context CLEARLY and in a manner that can %% 
%% be understood also by a non-specialist in your field.                   %% 
%%                                                                         %% 
%% Then argue - equally clearly! - how your proposed observing project     %% 
%% will contribute significantly to advancing our understanding of the     %% 
%% issue. Key references to the literature should be given.                %% 
%%                                                                         %%
%% Finally, describe how the data reduction and scientific analysis will   %%
%% be done, and document that the team possesses the required expertise.   %%
%%                                                                         %%
%% All text and figures should fit on the following two pages (page 2 and  %%
%% page 3), but text may spill over on page 3. All figures and references  %%
%% should be placed on page 3.                                             %%
%%                                                                         %%
%%%%%%%%%%%%%%%%%%%%%%%%%%%%%%%%%%%%%%%%%%%%%%%%%%%%%%%%%%%%%%%%%%%%%%%%%%%%%
%%
% Please type your text into the { } field below
\scijust{Dark matter was first termed in a paper from 1933 [ref] by Fritz
Zwicky. He used the virial theorem to calculate the gravitational mass of
the galaxies in the Coma cluster and found a discrepancy between the
measured mass and their expected luminosity. He referred to this
"missing mass" as "dunkle materie". Today astronomers have accumulated
convincing evidence of dark matter from independent observations such
as galaxy rotation curves, gravitational lensing, measurements of
the cosmic microwave background, baryon acoustic oscillations
, supernovae distance measurements, Lyman-alpha forest measurements 
of distant galaxies and in structure formation scenarios.
\par
According to the spectacularly successful Planck mission, the dark matter
part of the energy in the universe is a staggering 26.8\%
compared with the 4.9\% of ordinary matter. Even though the consensus among scientist
today is that dark matter consists of Weakly Interacting Massive Particles
(WIMPs), no official detections of these elusive particles have been made and
the hunt for these particles is one of the major undertakings of modern physics.
In what better way to make aspiring student of astronomy more comfortable
with observational instruments, than for them to "see" for
themselves what the "fuss" is all about? The reproducibility of science is
after all one of the fundamental pillars of science itself.
By the guidance of past and present mentors we therefore propose to use the
NOT telescope
to measure the rotation curve of NGC 5963, fit a light+dark mass profile 
to the acquired data and determine the stellar/dark matter mass components of
this galaxy. \\

\noindent \textbf{The need for a new observation and its selection:} The target in consideration, NGC 5963 is of the type Low Surface Brightness (LSB) galaxy (Romanishin, Strom \& Strom 1982 ApJ 252, 77) which are usual targets for dark matter studies due to their peculiar mass to light ratio. NGC 5963 is no exception to such studies (e.g. Bosma et al. 1988, A\& A 198, 100). However, the latest  direct observations of NGC 5963 we could find in the literature were taken over a decade ago (Simon et al. 2004 ASPC, 327, 18), which speaks for the acquisition of newer observations. 
\par
Another virtue of the selected target is that if photometric images of good enough quality of the galaxy is provided, they might give insight to its anomaly underluminous nature (Zackrisson et al. in preparation). NGC 5963 strongly deviates from the expected Tully-Fisher (TF) relation (Springob et al. 2007 ApJS, 172, 599) by being underluminous and/or having far greater non-baryonic mass than expected. Newer observations with the NOT may aid in uncovering why this is so.
 
   }
%%%%%%%%%%%%%%%%%%%%%%%%%%%%%%%%%%%%%%%%%%%%%%%%%%%%%%%%%%%%%%%%%%%%%%%%%%%%%
\end{scienpage}

%%%%%%%%%%%%%%%%%%%%%%%%%%%%%%%%%%%%%%%%%%%%%%%%%%%%%%%%%%%%%%%%%%%%%%%%%%%%%
%%
%%              !!!!!!!!!!!!! PAGE 3 !!!!!!!!!!!!!
%%
%%%%%%%%%%%%%%%%%%%%%%%%%%%%%%%%%%%%%%%%%%%%%%%%%%%%%%%%%%%%%%%%%%%%%%%%%%%%%
\newpage        % Page 3 
\putnumberLarge

\begin{scienpagec}[][]{}
%%%%%%%%%%%%%%%%%%%%%%%%%%%%%%%%%%%%%%%%%%%%%%%%%%%%%%%%%%%%%%%%%%%%%%%%%%%%%
%%              SCIENTIFIC JUSTIFICATION (CONTINUED)                       %%
%%                                                                         %%
%% Place References and any Figures here.                                  %%
%%                                                                         %%
%%%%%%%%%%%%%%%%%%%%%%%%%%%%%%%%%%%%%%%%%%%%%%%%%%%%%%%%%%%%%%%%%%%%%%%%%%%%%
% 
% Please type the rest of your text into the { } field below:
\scijustc{ }
%
%%%%%%%%%%%%%%%%%%%%%%%%%%%%%%%%%%%%%%%%%%%%%%%%%%%%%%%%%%%%%%%%%%%%%%%%%%%%%
%%                                FIGURES:                                 %%
%%                                                                         %%
%% Up to two postscript figures may be included                            %%
%%                                                                         %%
%% NB colour figures are not supported. All figures will be printed in     %%
%%    black and white and any colour information in the figures will be    %%
%%    disregarded                                                          %%
%%                                                                         %%
%% To enter a figure, uncomment the lines below, fill in the name of       %%
%% your .eps file, and provide a short caption where indicated             %%
%%                                                                         %%
%%%%%%%%%%%%%%%%%%%%%%%%%%%%%%%%%%%%%%%%%%%%%%%%%%%%%%%%%%%%%%%%%%%%%%%%%%%%%
%%
%%
%% NB: There should be no spaces in the \psfig argument below 
%% -  otherwise psfig will fail!
%%
%
%   Figure 1:   change 'PInameA.eps' to name of the file containing your 
%               figure. You may need to adjust the width and angle below,
%               and possibly the bounding box in the postscript-file.
%               Provide a short caption in the \captone{} field. Be sure 
%               to leave an empty line between the psfig and caption
%               command
%
%\psfig{file=PInameA.eps,width=12cm,angle=0,clip=}
%
%\captone{}
%
%
%   Figure 2:   change 'PInameB.eps' to name of the file containing your 
%               figure. You may need to adjust the width and angle below,
%               and possibly the bounding box in the postscript-file.
%               Provide a short caption in the \capttwo{} field. Be sure 
%               to leave an empty line between the psfig and caption
%               command
%
%\psfig{file=PInameB.eps,width=12cm,angle=0,clip=}
%
%\capttwo{}
%
%
%%%%%%%%%%%%%%%%%%%%%%%%%%%%%%%%%%%%%%%%%%%%%%%%%%%%%%%%%%%%%%%%%%%%%%%%%%%%%
\end{scienpagec}



%%%%%%%%%%%%%%%%%%%%%%%%%%%%%%%%%%%%%%%%%%%%%%%%%%%%%%%%%%%%%%%%%%%%%%%%%%%%%
%%
%%              !!!!!!!!!!!!! PAGE 4 !!!!!!!!!!!!!
%%
%%%%%%%%%%%%%%%%%%%%%%%%%%%%%%%%%%%%%%%%%%%%%%%%%%%%%%%%%%%%%%%%%%%%%%%%%%%%%
\newpage  % Page 4 
\putnumberLarge

\begin{justificpage}[][]{}
%%%%%%%%%%%%%%%%%%%%%%%%%%%%%%%%%%%%%%%%%%%%%%%%%%%%%%%%%%%%%%%%%%%%%%%%%%%%%
%%                   TECHNICAL JUSTIFICATION
%
%  Describe how the observations will be performed, so the feasibility of the 
%  project becomes clear. This should be self-contained and not rely on 
%  information given elsewhere. Describe the S/N ratio calculations you have 
%  used to justify the number of nights and the Moon phases you request 
%  (no correction for expected weather conditions should be applied). Also,
%  describe why the instrumental set-up is adequate for the objective of the
%  proposed observations (e.g., if the resolution of spectra is adequate to 
%  resolve the spectroscopic features you want to study).
%
% Please type in the { } field below:
\techjust{ }
%%%%%%%%%%%%%%%%%%%%%%%%%%%%%%%%%%%%%%%%%%%%%%%%%%%%%%%%%%%%%%%%%%%%%%%%%%%%%
\end{justificpage}




\begin{instrempage}[][]{}
%%%%%%%%%%%%%%%%%%%%%%%%%%%%%%%%%%%%%%%%%%%%%%%%%%%%%%%%%%%%%%%%%%%%%%%%%%%%%
%
%
%                       INSTRUMENT CONFIGURATIONS:
%
% Please specify the instrument configurations you want to use as fully as 
% possible, using the setup definitions provided below. You must uncomment
% the relevant setup definitions, i.e. remove the `%' sign in front of them,
% in order to make these lines take effect.
%
% Uncomment only the lines related to instrument configuration(s) needed
% for the acquisition of your planned  observations. Detailed information 
% on the available instruments is provided in the relevant users' manuals 
% (see http://www.not.iac.es/observing/proposals/).
%
% You also need to specify the Run ids (as defined in box 6) for which the 
% selected setups are valid. Put the run ID in the first {}, for example:
%
%       \NOTconfig{A}{ALFOSC}{Standard imaging-filters}{UBVRi}
%
%
%-----------------------------------------------------------------------
%----------------------------- ALFOSC ----------------------------------
%-----------------------------------------------------------------------
%
\NOTconfig{}{ALFOSC}{Standard imaging-filters}{UBVRi}
%\NOTconfig{}{ALFOSC}{Imaging-filters for ALFOSC}{provide filter No.}
%\NOTconfig{}{ALFOSC}{Imaging-filters for FASU}{provide filter No.}
%
%\NOTconfig{}{ALFOSC}{Fast-photometry}{Multi-windowing mode}
%
%\NOTconfig{}{ALFOSC}{Lin-Pol-imaging}{Polaroids}
%\NOTconfig{}{ALFOSC}{Lin-Pol-imaging}{Calcite+half-wave-plate}
%\NOTconfig{}{ALFOSC}{Cir-Pol-imaging}{Calcite+quarter-wave-plate}
%\NOTconfig{}{ALFOSC}{Lin-Pol-imaging}{WeDoWo}
%
%\NOTconfig{}{ALFOSC}{Lin-Pol-spectroscopy}{Calcite+half-wave-plate}
%\NOTconfig{}{ALFOSC}{Cir-Pol-spectroscopy}{Calcite+quarter-wave-plate}
%\NOTconfig{}{ALFOSC}{Pol-spectroscopy}{give grism number(s)}
%\NOTconfig{}{ALFOSC}{Pol-spectroscopy}{Polarimetric slitlet\#1.0"}
%\NOTconfig{}{ALFOSC}{Pol-spectroscopy}{Polarimetric slitlet\#1.4"}
%\NOTconfig{}{ALFOSC}{Pol-spectroscopy}{Polarimetric slitlet\#1.8"}
%
%\NOTconfig{}{ALFOSC}{Spectroscopy}{ADC}
%
%\NOTconfig{}{ALFOSC}{Spectro-long-slit}{give grism number(s)}
\NOTconfig{}{ALFOSC}{Spectro-long-slit}{1.0, 1.3}
\NOTconfig{}{ALFOSC}{Spectro-long-slit}{Grism\#8}
%\NOTconfig{}{ALFOSC}{Spectro-long-slit}{provide 2nd-order blocking filter No.}
%
%\NOTconfig{}{ALFOSC}{Multi-Object-Spectro}{provide HERE the number of masks}
%\NOTconfig{}{ALFOSC}{Multi-Object-Spectro}{give grism number(s)}
%\NOTconfig{}{ALFOSC}{Multi-Object-Spectro}{provide required slitwidth(s)}
%\NOTconfig{}{ALFOSC}{Multi-Object-Spectro}{Pre-imaging required}
%
%\NOTconfig{}{ALFOSC}{Spectro-Echelle}{Echelle Grism\#9}
%\NOTconfig{}{ALFOSC}{Spectro-Echelle}{Echelle Grism\#13}
%\NOTconfig{}{ALFOSC}{Spectro-Echelle}{0.7arcsec-slit}
%\NOTconfig{}{ALFOSC}{Spectro-Echelle}{0.8arcsec-slit}
%\NOTconfig{}{ALFOSC}{Spectro-Echelle}{1.0arcsec-slit}
%\NOTconfig{}{ALFOSC}{Spectro-Echelle}{1.2arcsec-slit}
%\NOTconfig{}{ALFOSC}{Spectro-Echelle}{1.6arcsec-slit}
%\NOTconfig{}{ALFOSC}{Spectro-Echelle}{1.8arcsec-slit}
%\NOTconfig{}{ALFOSC}{Spectro-Echelle}{2.2arcsec-slit}
%\NOTconfig{}{ALFOSC}{Spectro-Echelle}{Cross-disperser\#10}
%\NOTconfig{}{ALFOSC}{Spectro-Echelle}{Cross-disperser\#11}
%\NOTconfig{}{ALFOSC}{Spectro-Echelle}{Cross-disperser\#12}
%
%
%-----------------------------------------------------------------------
%----------------------------- NOTCam-----------------------------------
%-----------------------------------------------------------------------
%
%\NOTconfig{}{NOTCam}{Standard imaging-filters}{ZYJHKs}
%\NOTconfig{}{NOTCam}{Imaging-filters for NOTCam}{provide filter No.}
%
%\NOTconfig{}{NOTCam}{Imaging}{Wide-field camera}
%\NOTconfig{}{NOTCam}{Imaging}{High-res camera}
%
%\NOTconfig{}{NOTCam}{Spectro-long-slit}{Grism\#1}
%\NOTconfig{}{NOTCam}{Spectro-long-slit}{Wide-field camera slit 0.6''}
%\NOTconfig{}{NOTCam}{Spectro-long-slit}{High-res camera slit 0.2''}
%
%\NOTconfig{}{NOTCam}{Spectroscopy}{ADC}
%
%\NOTconfig{}{NOTCam}{Spectroscopy}{Wide-field camera}
%\NOTconfig{}{NOTCam}{Spectroscopy}{High-res camera}
%
%
%-----------------------------------------------------------------------
%----------------------------- FIES ------------------------------------
%-----------------------------------------------------------------------
%
% Please note that the simultaneous-ThAr mode can only be used in
% combination with either the High-res or Med-res fiber.
%
% Please note that the spec-pol mode can only be used in combination
% with the Med-Res fiber. The useful wavelength range is 370-630 nm.
%
%\NOTconfig{}{FIES}{Spectro-Echelle}{Low-Res Fiber}
%\NOTconfig{}{FIES}{Spectro-Echelle}{Med-Res Fiber}
%\NOTconfig{}{FIES}{Spectro-Echelle}{High-Res Fiber}
%
%\NOTconfig{}{FIES}{Spectro-Echelle}{Simultaneous-ThAr mode}
%
%\NOTconfig{}{FIES}{Spectro-Echelle}{Spec-Pol mode}
%
%\NOTconfig{}{FIES}{Spectro-Echelle}{ADC}
%
%
%-----------------------------------------------------------------------
%----------------------------- MOSCA -----------------------------------
%-----------------------------------------------------------------------
%
%\NOTconfig{}{MOSCA}{Standard imaging-filters}{UBVRI}
%\NOTconfig{}{MOSCA}{Standard imaging-filters}{ugriz}
%\NOTconfig{}{MOSCA}{Imaging-filters for FASU}{provide filter No.}
%
%
%-----------------------------------------------------------------------
%----------------------------- StanCam ---------------------------------
%-----------------------------------------------------------------------
%
%\NOTconfig{}{StanCam}{Standard imaging-filters}{UBVRIz}
%\NOTconfig{}{StanCam}{Imaging-filters for StanCam}{provide filter No.}
%
%
%-----------------------------------------------------------------------
%----------------------------- TurPol ----------------------------------
%-----------------------------------------------------------------------
%
% TurPol is not a common-user instrument. Normal support at the telescope 
% is provided, but only limited trouble-shooting can be made by NOT staff.
%
%\NOTconfig{}{TurPol}{Lin-Pol}{Half-wave-plate}
%\NOTconfig{}{TurPol}{Cir+Lin-Pol}{Quarter-wave-plate}
%\NOTconfig{}{TurPol}{Standard photometry}{UBVRI}
% 
%
%-----------------------------------------------------------------------
%----------------------------- Visitor Instruments ---------------------
%-----------------------------------------------------------------------
%
% If you wish to use a special instrument for the project, please contact
% the NOT director (Johannes Andersen, email: ja@astro.ku.dk), or the Head
% of Operations (Thomas Augusteijn, email: tau@not.iac.es) well in advance
% of submitting your proposal.
%
%\NOTconfig{}{Name of instrument}{key capability}{link to instrument URL}
%
%
%%%%%%%%%%%%%%%%%%%%%%%%%%%%%%%%%%%%%%%%%%%%%%%%%%%%%%%%%%%%%%%%%%%%%%%%%%%%%
%%                      REMARKS ON INSTRUMENT SETUP
%%
% If you have any special remarks or requirements, e.g. use of non standard
% observing modes, own equipment etc., please uncomment the %\remark{} line 
% below and provide the information in the {} field. 
% If necessary, more detailed information can be provided in BOX 18.
%
%\remark{} 
%%%%%%%%%%%%%%%%%%%%%%%%%%%%%%%%%%%%%%%%%%%%%%%%%%%%%%%%%%%%%%%%%%%%%%%%%%%%%
\end{instrempage}


%%%%%%%%%%%%%%%%%%%%%%%%%%%%%%%%%%%%%%%%%%%%%%%%%%%%%%%%%%%%%%%%%%%%%%%%%%%%%
%%
%%              !!!!!!!!!!!!! PAGE 5 !!!!!!!!!!!!!
%%
%%%%%%%%%%%%%%%%%%%%%%%%%%%%%%%%%%%%%%%%%%%%%%%%%%%%%%%%%%%%%%%%%%%%%%%%%%%%%
\newpage        % Page 5 
\putnumberLarge

\begin{objlistpage}{}
%%%%%%%%%%%%%%%%%%%%%%%%%%%%%%%%%%%%%%%%%%%%%%%%%%%%%%%%%%%%%%%%%%%%%%%%%%%%%
%%                            TARGET LIST                                  %% 
%% Target list with coordinates, or intervals in R.A. and Decl.            %%
%% of (sample of) objects:                                                 %%
%%                                                                         %%
%%%%%%%%%%%%%%%%%%%%%%%%%%%%%%%%%%%%%%%%%%%%%%%%%%%%%%%%%%%%%%%%%%%%%%%%%%%%%
%%                                                                         %%
%% Please fill in the relevant information for one object in the {   }     %%
%% fields below. Duplicate the line up to 20 objects; otherwise give a     %%
%% sample list and describe the full target list in the Remarks.           %%
%%                                                                         %%
%% Please specify the passband in which object magnitudes are given        %%
%%                                                                         %%
%% Targets are specified with the \target{#1}{#2}{#3}{#4}{#5}{#6}{#7}      %%
%% command, where the 7 arguments are:                                     %%
%%                                                                         %%
%%  #1: Run IDs (as defined box 6) for which target is relevant            %%
%%  #2: Short target name                                                  %%
%%  #3: Right ascension (2000)                                             %%
%%  #4: Declination (2000)                                                 %%
%%  #5: Target magnitude                                                   %%
%%  #6: Target diameter (in arcmin)                                        %% 
%%  #7: Additional Info                                                    %%
%%                                                                         %%
%%   Example:                                                              %%
%%                                                                         %%
%%   \target{A}{M31}{00:42:44}{+41:16:09}{V=4.4}{190}{$v_r=-300$ km/s}     %%
%%                                                                         %%
%%%%%%%%%%%%%%%%%%%%%%%%%%%%%%%%%%%%%%%%%%%%%%%%%%%%%%%%%%%%%%%%%%%%%%%%%%%%%
\target{A}{NGC 5963}{15 33 27.8}{+56 33 35}{B=12.70}{1.52}{} % duplicate as needed
%%%%%%%%%%%%%%%%%%%%%%%%%%%%%%%%%%%%%%%%%%%%%%%%%%%%%%%%%%%%%%%%%%%%%%%%%%%%%
%% Please type any Remarks on the target list in the {   } field below: 
%%
\remark{ }
%%%%%%%%%%%%%%%%%%%%%%%%%%%%%%%%%%%%%%%%%%%%%%%%%%%%%%%%%%%%%%%%%%%%%%%%%%%%%
\end{objlistpage}

\begin{qualitypage}[][]{}
%%%%%%%%%%%%%%%%%%%%%%%%%%%%%%%%%%%%%%%%%%%%%%%%%%%%%%%%%%%%%%%%%%%%%%%%%%%%%
%%                       BACKUP PROGRAMME
%%
% For projects needing excellent image quality or photometric conditions, 
% give a short description of possible backup programme. A backup programme 
% is also needed in case an unfavourable wind speed or direction may prevent
% you from executing your main programme.    
%
% - If no backup programme is provided, justify why none is needed. Failure
%   to ensure that the telescope will be used productively under all 
%   circumstances will lower the rating of your proposal!
%
% - The instrumental setup for the backup programme should normally be the 
%   same as for the main programme; however, standby instrumentation (e.g. 
%   StanCam or FIES) may be used instead. Please specify the instrument you 
%   will use.
% 
% Please type your description in the { } field below:
%
\backup{ }
%%%%%%%%%%%%%%%%%%%%%%%%%%%%%%%%%%%%%%%%%%%%%%%%%%%%%%%%%%%%%%%%%%%%%%%%%%%%%
\end{qualitypage}


%%%%%%%%%%%%%%%%%%%%%%%%%%%%%%%%%%%%%%%%%%%%%%%%%%%%%%%%%%%%%%%%%%%%%%%%%%%%%
%%
%%           !!!!!!!!!  PAGE 6 = Last page  !!!!!!!!!!!!!
%%
%%%%%%%%%%%%%%%%%%%%%%%%%%%%%%%%%%%%%%%%%%%%%%%%%%%%%%%%%%%%%%%%%%%%%%%%%%%%%
\newpage        % Page 6 
\putnumberLarge

\begin{prevobspage}[][]{}
%%%%%%%%%%%%%%%%%%%%%%%%%%%%%%%%%%%%%%%%%%%%%%%%%%%%%%%%%%%%%%%%%%%%%%%%%%%%%
%%              PREVIOUS OBSERVING PERIODS AND RESULTS
%%
% Please list your observing periods at the NOT within the last three years 
% and provide a brief status report on your analysis of the data. List also 
% your publications from NOT observations during the last three years. 
%
%%  Please type this information in the { } field below:
%% 
\prevobs{ }
%%%%%%%%%%%%%%%%%%%%%%%%%%%%%%%%%%%%%%%%%%%%%%%%%%%%%%%%%%%%%%%%%%%%%%%%%%%%%
\end{prevobspage}

\begin{extrapage}[][]{}
%%%%%%%%%%%%%%%%%%%%%%%%%%%%%%%%%%%%%%%%%%%%%%%%%%%%%%%%%%%%%%%%%%%%%%%%%%%%%
%%              ADDITIONAL INFO
%%
% Use this box to provide any additional information/remarks not covered 
% by the items above, e.g. related proposals to other telescopes, dates to 
% be avoided for the observations, etc. Do not use if not necessary.  
%
%%  Please type in the { } field below:
%% 
\addinfo{ }
%%%%%%%%%%%%%%%%%%%%%%%%%%%%%%%%%%%%%%%%%%%%%%%%%%%%%%%%%%%%%%%%%%%%%%%%%%%%%
\end{extrapage}


\end{document}
%%%%%%%%%%%%%%%%%%%%%%%%%%%%%%%%%%%%%%%%%%%%%%%%%%%%%%%%%%%%%%%%%%%%%%%%%%%%%
% 
%                       End of template file
% 
%%%%%%%%%%%%%%%%%%%%%%%%%%%%%%%%%%%%%%%%%%%%%%%%%%%%%%%%%%%%%%%%%%%%%%%%%%%%%

                    
\begin{document}
\def\propnumber{}                       % Only to be filled in by NOT staff
\putnumberHuge
\nothead

%%%%%%%%%%%%%%%%%%%%%%%%%%%%%%%%%%%%%%%%%%%%%%%%%%%%%%%%%%%%%%%%%%%%%%%%%%%%%
%%                                                                         %%
%%                    *** NOTE TO APPLICANTS ***                           %%
%%                                                                         %%
%%%%%%%%%%%%%%%%%%%%%%%%%%%%%%%%%%%%%%%%%%%%%%%%%%%%%%%%%%%%%%%%%%%%%%%%%%%%%
%%                                                                         %%
%%          NOT PROPOSAL TEMPLATE FILE FOR OBSERVING PERIOD 48             %%
%%                                                                         %%
%%                   OCTOBER 1, 2013 - APRIL 1, 2014                       %%
%%                                                                         %%
%%                                                                         %%
%% Please take care to fill in the fields of this proposal form as         %%
%% indicated, following the instructions and advice provided in the        %%
%% header of each section. Run LaTex on your completed form and verify     %%
%% the result before submitting to check that it runs correctly and        %%
%% do not overfill any of the boxes, and that it produces a total of no    %%
%% more than six (6) printed pages.                                        %%
%%                                                                         %%
%% Be sure to always use the correct version of the not-style file for     %%
%% the period in question. This template is only valid for period 48.      %%
%%                                                                         %%
%% Never change the format of the template or style file. Proposals        %%
%% that do not comply with the correct version of the style and            %%
%% template files will be rejected.                                        %%
%%                                                                         %%
%% Name the file PIname.tex (e.g. johanson.tex) and any figure file(s)     %%
%% as PInameA.ps (and PInameB.ps). After verifying that the proposal       %%
%% can be properly processed, submit the file(s) as (separate) attached    %%
%% file(s) by e-mail to the address:                                       %%
%%                                                                         %%
%%                          proposal@not.iac.es                            %%
%%                                                                         %%
%% with the word ``Proposal'' in the 'Subject' field and as text in the    %%
%% body of the message. The latter is important when you use certain       %%
%% mailers as the proposal might otherwise not be parsed correctly by      %%
%% our automatic procedure.                                                %%
%%                                                                         %%
%% Do not compress the files or combine them in a tar file. Do not         %%
%% submit the style file.                                                  %%
%%                                                                         %%
%% Any questions regarding the proposal procedure may be submitted to      %%
%% the same e-mail address, giving ``Question'' as the 'Subject'.          %%
%%                                                                         %%
%% If you submit more than one proposal, please name the file              %%
%% PIname1.tex, PIname2.tex, etc., and any figure files accordingly.       %%
%% Only one proposal should be submitted at the time.                      %%
%%                                                                         %%
%% For more information on the Nordic Optical Telescope see:               %%
%%                                                                         %%
%%                       http://www.not.iac.es/                            %%
%%                                                                         %%
%%%%%%%%%%%%%%%%%%%%%%%%%%%%%%%%%%%%%%%%%%%%%%%%%%%%%%%%%%%%%%%%%%%%%%%%%%%%%


\begin{titpage}{}
%%%%%%%%%%%%%%%%%%%%%%%%%%%%%%%%%%%%%%%%%%%%%%%%%%%%%%%%%%%%%%%%%%%%%%%%%%%%%
%                          PROPOSAL TITLE
% Type title of proposal in the { } below - one line only!
%
\proptitle{Measuring the Rotation Curve of the Elusive NGC 5963: The Adventure.}
%%%%%%%%%%%%%%%%%%%%%%%%%%%%%%%%%%%%%%%%%%%%%%%%%%%%%%%%%%%%%%%%%%%%%%%%%%%%%
\end{titpage}


\begin{abspage}[][]{}
%%%%%%%%%%%%%%%%%%%%%%%%%%%%%%%%%%%%%%%%%%%%%%%%%%%%%%%%%%%%%%%%%%%%%%%%%%%%%
%%                             ABSTRACT
%
% Please type the Abstract of the proposal into the { } below
% Do not exceed the space provided 
%
\propabstract{}
%%%%%%%%%%%%%%%%%%%%%%%%%%%%%%%%%%%%%%%%%%%%%%%%%%%%%%%%%%%%%%%%%%%%%%%%%%%%%
\end{abspage}


\begin{adrinvpage}{}
%%%%%%%%%%%%%%%%%%%%%%%%%%%%%%%%%%%%%%%%%%%%%%%%%%%%%%%%%%%%%%%%%%%%%%%%%%%%%
%%                      PRINCIPAL INVESTIGATOR
%
% Name and address of Principal Investigator (PI)
% 
% NB: The PI has full responsibility for the content of this proposal!
%
% Please fill in the appropriate { } below:
%
\piname{}               % name of PI
\piinst{}               % PI institute
\picoun{SE}               % PI country (ISO code: DK,FI,IS,NO,SE,..) 
\piaddr{}               % PI postal address
\piteln{}               % PI telephone number
\pifaxn{}               % PI fax number
\pimail{}               % PI email address
%%%%%%%%%%%%%%%%%%%%%%%%%%%%%%%%%%%%%%%%%%%%%%%%%%%%%%%%%%%%%%%%%%%%%%%%%%%%%
\end{adrinvpage}


\begin{coinvestpage}{}
%%%%%%%%%%%%%%%%%%%%%%%%%%%%%%%%%%%%%%%%%%%%%%%%%%%%%%%%%%%%%%%%%%%%%%%%%%%%%
%%                          CO-INVESTIGATORS
% 
% Name and institute of co-investigators
% Please fill in the { } { } fields below (2 Co-Is per line):
%
% There is room for up to 10 CoIs. Even if the project involves more 
% than 10 CoIs, please do not list more than 10 CoIs 
%
%        {Name1, Institute1}  {Name2, Institute2}
%
\coinvest{ }{ }  % {Name1, Institute1}   {Name2, Institute2} 
\coinvest{ }{ }  % {Name3, Institute3}   etc
\coinvest{ }{ }
\coinvest{ }{ }
\coinvest{ }{ }
% 
%%%%%%%%%%%%%%%%%%%%%%%%%%%%%%%%%%%%%%%%%%%%%%%%%%%%%%%%%%%%%%%%%%%%%%%%%%%%%
\end{coinvestpage}


\begin{omthesispage}{}
%%%%%%%%%%%%%%%%%%%%%%%%%%%%%%%%%%%%%%%%%%%%%%%%%%%%%%%%%%%%%%%%%%%%%%%%%%%%%
%%                        THESIS PROJECTS
%
% If this proposal concerns a PhD thesis work at Nordic Institute,
% please provide: name of the student, institute, name of supervisor, 
% and expected time of completion.
%
% Please type in the { } field below:
\thesis{} 
%%%%%%%%%%%%%%%%%%%%%%%%%%%%%%%%%%%%%%%%%%%%%%%%%%%%%%%%%%%%%%%%%%%%%%%%%%%%%
\end{omthesispage}


\begin{nightspage}{}
%%%%%%%%%%%%%%%%%%%%%%%%%%%%%%%%%%%%%%%%%%%%%%%%%%%%%%%%%%%%%%%%%%%%%%%%%%%%%
%%                       REQUESTED OBSERVING RUN(S)
%% 
%% NB: In the following, an ``Observing run'' is a single, contiguous block 
%% of time with a single instrument. If your project requires more than one
%% such run, e.g. at different times and/or with different instruments, 
%% please identify each run as A, B, C,... and specify your requirements 
%% for each on a separate line as specified below.
%%
%%%%%%%%%%%%%%%%%%%%%%%%%%%%%%%%%%%%%%%%%%%%%%%%%%%%%%%%%%%%%%%%%%%%%%%%%%%%%
%%
%% Give requested no. of nights/hours as a number (not word) and specify
%% the unit: as N (nights) or H (hours) 
%%
%% Indicate desired Moon phases as D=dark/G=grey/N=no restriction
%% 
%% Indicate the seeing requirements: 0.7 (max 0.7 arcsec seeing), 
%%      1.0 (max 1.0 arcsec), 1.3 (max 1.3 arcsec), or N (no restriction)
%%
%% If the programme require specific sky condition, enter these here
%%      P = photometric conditions required
%%      C = clear conditions required
%%      T = thin clouds/cirrus acceptable
%% 
%% Please fill in the relevant information in the {   } fields below, and 
%% duplicate the entire block if more than one run is requested
%%
%% N.B. A maximum of 6 runs per proposal can be specified
%%
%%%%%%%%%%%%%%%%%%%%%%%%%%%%%%%%%%%%%%%%%%%%%%%%%%%%%%%%%%%%%%%%%%%%%%%%%%%%%
%%
%%%%%%%%%%%%%%%%%%%%%%%%%%%%%% Run A %%%%%%%%%%%%%%%%%%%%%%%%%%%%%%%%%%%%%%%%
%
\nrunid{A }      % Put your run id (A, B, C, ...) here
\ninstr{ALFOSC }      % Put instrument name here
\ntimer{XXX H }      % Put requested time, in numbers, with unit (N or H), e.g 5 N
\nmonth{May }      % Put preferred month(s) here
\nmoonp{B }      % Put requested moon phase here: D, G or N
\nsemax{ }      % Put maximum allowed seeing here: (0.7,1.0,1.5,N)
\nskyco{ }      % Put required photometric condition here
%
%%%%%%%%%%%%%%%%%%%%%%%%%%%%%%%%%%%%%%%%%%%%%%%%%%%%%%%%%%%%%%%%%%%%%%%%%%%%%
\end{nightspage}

\begin{numnightspage}{}
%%%%%%%%%%%%%%%%%%%%%%%%%%%%%%%%%%%%%%%%%%%%%%%%%%%%%%%%%%%%%%%%%%%%%%%%%%%%%
%%                  TIME BEFORE/AFTER PRESENT REQUEST
%%
% Number of nights already awarded to project. More details, e.g. on 
% instrumentation and outcome of previous observations can be given in box 17
%
% Please type in the { } field below:
% 
\numalr{}
%
% Number of nights needed to complete project (excluding those requested).
%
\numrem{}
%%%%%%%%%%%%%%%%%%%%%%%%%%%%%%%%%%%%%%%%%%%%%%%%%%%%%%%%%%%%%%%%%%%%%%%%%%%%%
\end{numnightspage}


\begin{servicepage}[][]{}
%%%%%%%%%%%%%%%%%%%%%%%%%%%%%%%%%%%%%%%%%%%%%%%%%%%%%%%%%%%%%%%%%%%%%%%%%%%%%
%%                         SERVICE CONSTRAINTS
%
%
% All projects will be considered for execution in service/queue mode. In
% case your project can not be done in service/queue mode, please give a
% justification in the { } field below:
\service{ }
%%%%%%%%%%%%%%%%%%%%%%%%%%%%%%%%%%%%%%%%%%%%%%%%%%%%%%%%%%%%%%%%%%%%%%%%%%%%%
\end{servicepage}


\begin{schedpage}[][]{}
%%%%%%%%%%%%%%%%%%%%%%%%%%%%%%%%%%%%%%%%%%%%%%%%%%%%%%%%%%%%%%%%%%%%%%%%%%%%%
%%                         SCHEDULING CONSTRAINTS
%
%
% Any other special constraints on the scheduling?
%
% E.g. time critical scheduling, or required baseline vs phase coverage
% for monitoring programs, response time for target of opportunity, 
% simultaneous observations, impossible dates, etc... 
%
% Please type in the { } field below:
%
\schedconstr{ }
%%%%%%%%%%%%%%%%%%%%%%%%%%%%%%%%%%%%%%%%%%%%%%%%%%%%%%%%%%%%%%%%%%%%%%%%%%%%%
\end{schedpage}


%%%%%%%%%%%%%%%%%%%%%%%%%%%%%%%%%%%%%%%%%%%%%%%%%%%%%%%%%%%%%%%%%%%%%%%%%%%%%
%%
%%              !!!!!!!!!!!!! PAGE 2 !!!!!!!!!!!!!
%%
%%%%%%%%%%%%%%%%%%%%%%%%%%%%%%%%%%%%%%%%%%%%%%%%%%%%%%%%%%%%%%%%%%%%%%%%%%%%%
\newpage        % Page 2  
\putnumberLarge

\begin{scienpage}[][]{}
%%%%%%%%%%%%%%%%%%%%%%%%%%%%%%%%%%%%%%%%%%%%%%%%%%%%%%%%%%%%%%%%%%%%%%%%%%%%%
%%                     SCIENTIFIC JUSTIFICATION                            %%
%%                                                                         %%
%% Note: This should be self-contained and not refer to previous proposals.%%
%%                                                                         %%
%% Describe first the scientific background and main goals of the proposal.%%
%% As OPC members cannot be experts in every field, it is CRUCIAL that you %% 
%% outline the general scientific context CLEARLY and in a manner that can %% 
%% be understood also by a non-specialist in your field.                   %% 
%%                                                                         %% 
%% Then argue - equally clearly! - how your proposed observing project     %% 
%% will contribute significantly to advancing our understanding of the     %% 
%% issue. Key references to the literature should be given.                %% 
%%                                                                         %%
%% Finally, describe how the data reduction and scientific analysis will   %%
%% be done, and document that the team possesses the required expertise.   %%
%%                                                                         %%
%% All text and figures should fit on the following two pages (page 2 and  %%
%% page 3), but text may spill over on page 3. All figures and references  %%
%% should be placed on page 3.                                             %%
%%                                                                         %%
%%%%%%%%%%%%%%%%%%%%%%%%%%%%%%%%%%%%%%%%%%%%%%%%%%%%%%%%%%%%%%%%%%%%%%%%%%%%%
%%
% Please type your text into the { } field below
\scijust{Dark matter was first termed in a paper from 1933 [ref] by Fritz
Zwicky. He used the virial theorem to calculate the gravitational mass of
the galaxies in the Coma cluster and found a discrepancy between the
measured mass and their expected luminosity. He referred to this
"missing mass" as "dunkle materie". Today astronomers have accumulated
convincing evidence of dark matter from independent observations such
as galaxy rotation curves, gravitational lensing, measurements of
the cosmic microwave background, baryon acoustic oscillations
, supernovae distance measurements, Lyman-alpha forest measurements 
of distant galaxies and in structure formation scenarios.
\par
According to the spectacularly successful Planck mission, the dark matter
part of the energy in the universe is a staggering 26.8\%
compared with the 4.9\% of ordinary matter. Even though the consensus among scientist
today is that dark matter consists of Weakly Interacting Massive Particles
(WIMPs), no official detections of these elusive particles have been made and
the hunt for these particles is one of the major undertakings of modern physics.
In what better way to make aspiring student of astronomy more comfortable
with observational instruments, than for them to "see" for
themselves what the "fuss" is all about? The reproducibility of science is
after all one of the fundamental pillars of science itself.
By the guidance of past and present mentors we therefore propose to use the
NOT telescope
to measure the rotation curve of NGC 5963, fit a light+dark mass profile 
to the acquired data and determine the stellar/dark matter mass components of
this galaxy. \\

\noindent \textbf{The need for a new observation and its selection:} The target in consideration, NGC 5963 is of the type Low Surface Brightness (LSB) galaxy (Romanishin, Strom \& Strom 1982 ApJ 252, 77) which are usual targets for dark matter studies due to their peculiar mass to light ratio. NGC 5963 is no exception to such studies (e.g. Bosma et al. 1988, A\& A 198, 100). However, the latest  direct observations of NGC 5963 we could find in the literature were taken over a decade ago (Simon et al. 2004 ASPC, 327, 18), which speaks for the acquisition of newer observations. 
\par
Another virtue of the selected target is that if photometric images of good enough quality of the galaxy is provided, they might give insight to its anomaly underluminous nature (Zackrisson et al. in preparation). NGC 5963 strongly deviates from the expected Tully-Fisher (TF) relation (Springob et al. 2007 ApJS, 172, 599) by being underluminous and/or having far greater non-baryonic mass than expected. Newer observations with the NOT may aid in uncovering why this is so.
 
   }
%%%%%%%%%%%%%%%%%%%%%%%%%%%%%%%%%%%%%%%%%%%%%%%%%%%%%%%%%%%%%%%%%%%%%%%%%%%%%
\end{scienpage}

%%%%%%%%%%%%%%%%%%%%%%%%%%%%%%%%%%%%%%%%%%%%%%%%%%%%%%%%%%%%%%%%%%%%%%%%%%%%%
%%
%%              !!!!!!!!!!!!! PAGE 3 !!!!!!!!!!!!!
%%
%%%%%%%%%%%%%%%%%%%%%%%%%%%%%%%%%%%%%%%%%%%%%%%%%%%%%%%%%%%%%%%%%%%%%%%%%%%%%
\newpage        % Page 3 
\putnumberLarge

\begin{scienpagec}[][]{}
%%%%%%%%%%%%%%%%%%%%%%%%%%%%%%%%%%%%%%%%%%%%%%%%%%%%%%%%%%%%%%%%%%%%%%%%%%%%%
%%              SCIENTIFIC JUSTIFICATION (CONTINUED)                       %%
%%                                                                         %%
%% Place References and any Figures here.                                  %%
%%                                                                         %%
%%%%%%%%%%%%%%%%%%%%%%%%%%%%%%%%%%%%%%%%%%%%%%%%%%%%%%%%%%%%%%%%%%%%%%%%%%%%%
% 
% Please type the rest of your text into the { } field below:
\scijustc{ }
%
%%%%%%%%%%%%%%%%%%%%%%%%%%%%%%%%%%%%%%%%%%%%%%%%%%%%%%%%%%%%%%%%%%%%%%%%%%%%%
%%                                FIGURES:                                 %%
%%                                                                         %%
%% Up to two postscript figures may be included                            %%
%%                                                                         %%
%% NB colour figures are not supported. All figures will be printed in     %%
%%    black and white and any colour information in the figures will be    %%
%%    disregarded                                                          %%
%%                                                                         %%
%% To enter a figure, uncomment the lines below, fill in the name of       %%
%% your .eps file, and provide a short caption where indicated             %%
%%                                                                         %%
%%%%%%%%%%%%%%%%%%%%%%%%%%%%%%%%%%%%%%%%%%%%%%%%%%%%%%%%%%%%%%%%%%%%%%%%%%%%%
%%
%%
%% NB: There should be no spaces in the \psfig argument below 
%% -  otherwise psfig will fail!
%%
%
%   Figure 1:   change 'PInameA.eps' to name of the file containing your 
%               figure. You may need to adjust the width and angle below,
%               and possibly the bounding box in the postscript-file.
%               Provide a short caption in the \captone{} field. Be sure 
%               to leave an empty line between the psfig and caption
%               command
%
%\psfig{file=PInameA.eps,width=12cm,angle=0,clip=}
%
%\captone{}
%
%
%   Figure 2:   change 'PInameB.eps' to name of the file containing your 
%               figure. You may need to adjust the width and angle below,
%               and possibly the bounding box in the postscript-file.
%               Provide a short caption in the \capttwo{} field. Be sure 
%               to leave an empty line between the psfig and caption
%               command
%
%\psfig{file=PInameB.eps,width=12cm,angle=0,clip=}
%
%\capttwo{}
%
%
%%%%%%%%%%%%%%%%%%%%%%%%%%%%%%%%%%%%%%%%%%%%%%%%%%%%%%%%%%%%%%%%%%%%%%%%%%%%%
\end{scienpagec}



%%%%%%%%%%%%%%%%%%%%%%%%%%%%%%%%%%%%%%%%%%%%%%%%%%%%%%%%%%%%%%%%%%%%%%%%%%%%%
%%
%%              !!!!!!!!!!!!! PAGE 4 !!!!!!!!!!!!!
%%
%%%%%%%%%%%%%%%%%%%%%%%%%%%%%%%%%%%%%%%%%%%%%%%%%%%%%%%%%%%%%%%%%%%%%%%%%%%%%
\newpage  % Page 4 
\putnumberLarge

\begin{justificpage}[][]{}
%%%%%%%%%%%%%%%%%%%%%%%%%%%%%%%%%%%%%%%%%%%%%%%%%%%%%%%%%%%%%%%%%%%%%%%%%%%%%
%%                   TECHNICAL JUSTIFICATION
%
%  Describe how the observations will be performed, so the feasibility of the 
%  project becomes clear. This should be self-contained and not rely on 
%  information given elsewhere. Describe the S/N ratio calculations you have 
%  used to justify the number of nights and the Moon phases you request 
%  (no correction for expected weather conditions should be applied). Also,
%  describe why the instrumental set-up is adequate for the objective of the
%  proposed observations (e.g., if the resolution of spectra is adequate to 
%  resolve the spectroscopic features you want to study).
%
% Please type in the { } field below:
\techjust{ }
%%%%%%%%%%%%%%%%%%%%%%%%%%%%%%%%%%%%%%%%%%%%%%%%%%%%%%%%%%%%%%%%%%%%%%%%%%%%%
\end{justificpage}




\begin{instrempage}[][]{}
%%%%%%%%%%%%%%%%%%%%%%%%%%%%%%%%%%%%%%%%%%%%%%%%%%%%%%%%%%%%%%%%%%%%%%%%%%%%%
%
%
%                       INSTRUMENT CONFIGURATIONS:
%
% Please specify the instrument configurations you want to use as fully as 
% possible, using the setup definitions provided below. You must uncomment
% the relevant setup definitions, i.e. remove the `%' sign in front of them,
% in order to make these lines take effect.
%
% Uncomment only the lines related to instrument configuration(s) needed
% for the acquisition of your planned  observations. Detailed information 
% on the available instruments is provided in the relevant users' manuals 
% (see http://www.not.iac.es/observing/proposals/).
%
% You also need to specify the Run ids (as defined in box 6) for which the 
% selected setups are valid. Put the run ID in the first {}, for example:
%
%       \NOTconfig{A}{ALFOSC}{Standard imaging-filters}{UBVRi}
%
%
%-----------------------------------------------------------------------
%----------------------------- ALFOSC ----------------------------------
%-----------------------------------------------------------------------
%
\NOTconfig{}{ALFOSC}{Standard imaging-filters}{UBVRi}
%\NOTconfig{}{ALFOSC}{Imaging-filters for ALFOSC}{provide filter No.}
%\NOTconfig{}{ALFOSC}{Imaging-filters for FASU}{provide filter No.}
%
%\NOTconfig{}{ALFOSC}{Fast-photometry}{Multi-windowing mode}
%
%\NOTconfig{}{ALFOSC}{Lin-Pol-imaging}{Polaroids}
%\NOTconfig{}{ALFOSC}{Lin-Pol-imaging}{Calcite+half-wave-plate}
%\NOTconfig{}{ALFOSC}{Cir-Pol-imaging}{Calcite+quarter-wave-plate}
%\NOTconfig{}{ALFOSC}{Lin-Pol-imaging}{WeDoWo}
%
%\NOTconfig{}{ALFOSC}{Lin-Pol-spectroscopy}{Calcite+half-wave-plate}
%\NOTconfig{}{ALFOSC}{Cir-Pol-spectroscopy}{Calcite+quarter-wave-plate}
%\NOTconfig{}{ALFOSC}{Pol-spectroscopy}{give grism number(s)}
%\NOTconfig{}{ALFOSC}{Pol-spectroscopy}{Polarimetric slitlet\#1.0"}
%\NOTconfig{}{ALFOSC}{Pol-spectroscopy}{Polarimetric slitlet\#1.4"}
%\NOTconfig{}{ALFOSC}{Pol-spectroscopy}{Polarimetric slitlet\#1.8"}
%
%\NOTconfig{}{ALFOSC}{Spectroscopy}{ADC}
%
%\NOTconfig{}{ALFOSC}{Spectro-long-slit}{give grism number(s)}
\NOTconfig{}{ALFOSC}{Spectro-long-slit}{1.0, 1.3}
\NOTconfig{}{ALFOSC}{Spectro-long-slit}{Grism\#8}
%\NOTconfig{}{ALFOSC}{Spectro-long-slit}{provide 2nd-order blocking filter No.}
%
%\NOTconfig{}{ALFOSC}{Multi-Object-Spectro}{provide HERE the number of masks}
%\NOTconfig{}{ALFOSC}{Multi-Object-Spectro}{give grism number(s)}
%\NOTconfig{}{ALFOSC}{Multi-Object-Spectro}{provide required slitwidth(s)}
%\NOTconfig{}{ALFOSC}{Multi-Object-Spectro}{Pre-imaging required}
%
%\NOTconfig{}{ALFOSC}{Spectro-Echelle}{Echelle Grism\#9}
%\NOTconfig{}{ALFOSC}{Spectro-Echelle}{Echelle Grism\#13}
%\NOTconfig{}{ALFOSC}{Spectro-Echelle}{0.7arcsec-slit}
%\NOTconfig{}{ALFOSC}{Spectro-Echelle}{0.8arcsec-slit}
%\NOTconfig{}{ALFOSC}{Spectro-Echelle}{1.0arcsec-slit}
%\NOTconfig{}{ALFOSC}{Spectro-Echelle}{1.2arcsec-slit}
%\NOTconfig{}{ALFOSC}{Spectro-Echelle}{1.6arcsec-slit}
%\NOTconfig{}{ALFOSC}{Spectro-Echelle}{1.8arcsec-slit}
%\NOTconfig{}{ALFOSC}{Spectro-Echelle}{2.2arcsec-slit}
%\NOTconfig{}{ALFOSC}{Spectro-Echelle}{Cross-disperser\#10}
%\NOTconfig{}{ALFOSC}{Spectro-Echelle}{Cross-disperser\#11}
%\NOTconfig{}{ALFOSC}{Spectro-Echelle}{Cross-disperser\#12}
%
%
%-----------------------------------------------------------------------
%----------------------------- NOTCam-----------------------------------
%-----------------------------------------------------------------------
%
%\NOTconfig{}{NOTCam}{Standard imaging-filters}{ZYJHKs}
%\NOTconfig{}{NOTCam}{Imaging-filters for NOTCam}{provide filter No.}
%
%\NOTconfig{}{NOTCam}{Imaging}{Wide-field camera}
%\NOTconfig{}{NOTCam}{Imaging}{High-res camera}
%
%\NOTconfig{}{NOTCam}{Spectro-long-slit}{Grism\#1}
%\NOTconfig{}{NOTCam}{Spectro-long-slit}{Wide-field camera slit 0.6''}
%\NOTconfig{}{NOTCam}{Spectro-long-slit}{High-res camera slit 0.2''}
%
%\NOTconfig{}{NOTCam}{Spectroscopy}{ADC}
%
%\NOTconfig{}{NOTCam}{Spectroscopy}{Wide-field camera}
%\NOTconfig{}{NOTCam}{Spectroscopy}{High-res camera}
%
%
%-----------------------------------------------------------------------
%----------------------------- FIES ------------------------------------
%-----------------------------------------------------------------------
%
% Please note that the simultaneous-ThAr mode can only be used in
% combination with either the High-res or Med-res fiber.
%
% Please note that the spec-pol mode can only be used in combination
% with the Med-Res fiber. The useful wavelength range is 370-630 nm.
%
%\NOTconfig{}{FIES}{Spectro-Echelle}{Low-Res Fiber}
%\NOTconfig{}{FIES}{Spectro-Echelle}{Med-Res Fiber}
%\NOTconfig{}{FIES}{Spectro-Echelle}{High-Res Fiber}
%
%\NOTconfig{}{FIES}{Spectro-Echelle}{Simultaneous-ThAr mode}
%
%\NOTconfig{}{FIES}{Spectro-Echelle}{Spec-Pol mode}
%
%\NOTconfig{}{FIES}{Spectro-Echelle}{ADC}
%
%
%-----------------------------------------------------------------------
%----------------------------- MOSCA -----------------------------------
%-----------------------------------------------------------------------
%
%\NOTconfig{}{MOSCA}{Standard imaging-filters}{UBVRI}
%\NOTconfig{}{MOSCA}{Standard imaging-filters}{ugriz}
%\NOTconfig{}{MOSCA}{Imaging-filters for FASU}{provide filter No.}
%
%
%-----------------------------------------------------------------------
%----------------------------- StanCam ---------------------------------
%-----------------------------------------------------------------------
%
%\NOTconfig{}{StanCam}{Standard imaging-filters}{UBVRIz}
%\NOTconfig{}{StanCam}{Imaging-filters for StanCam}{provide filter No.}
%
%
%-----------------------------------------------------------------------
%----------------------------- TurPol ----------------------------------
%-----------------------------------------------------------------------
%
% TurPol is not a common-user instrument. Normal support at the telescope 
% is provided, but only limited trouble-shooting can be made by NOT staff.
%
%\NOTconfig{}{TurPol}{Lin-Pol}{Half-wave-plate}
%\NOTconfig{}{TurPol}{Cir+Lin-Pol}{Quarter-wave-plate}
%\NOTconfig{}{TurPol}{Standard photometry}{UBVRI}
% 
%
%-----------------------------------------------------------------------
%----------------------------- Visitor Instruments ---------------------
%-----------------------------------------------------------------------
%
% If you wish to use a special instrument for the project, please contact
% the NOT director (Johannes Andersen, email: ja@astro.ku.dk), or the Head
% of Operations (Thomas Augusteijn, email: tau@not.iac.es) well in advance
% of submitting your proposal.
%
%\NOTconfig{}{Name of instrument}{key capability}{link to instrument URL}
%
%
%%%%%%%%%%%%%%%%%%%%%%%%%%%%%%%%%%%%%%%%%%%%%%%%%%%%%%%%%%%%%%%%%%%%%%%%%%%%%
%%                      REMARKS ON INSTRUMENT SETUP
%%
% If you have any special remarks or requirements, e.g. use of non standard
% observing modes, own equipment etc., please uncomment the %\remark{} line 
% below and provide the information in the {} field. 
% If necessary, more detailed information can be provided in BOX 18.
%
%\remark{} 
%%%%%%%%%%%%%%%%%%%%%%%%%%%%%%%%%%%%%%%%%%%%%%%%%%%%%%%%%%%%%%%%%%%%%%%%%%%%%
\end{instrempage}


%%%%%%%%%%%%%%%%%%%%%%%%%%%%%%%%%%%%%%%%%%%%%%%%%%%%%%%%%%%%%%%%%%%%%%%%%%%%%
%%
%%              !!!!!!!!!!!!! PAGE 5 !!!!!!!!!!!!!
%%
%%%%%%%%%%%%%%%%%%%%%%%%%%%%%%%%%%%%%%%%%%%%%%%%%%%%%%%%%%%%%%%%%%%%%%%%%%%%%
\newpage        % Page 5 
\putnumberLarge

\begin{objlistpage}{}
%%%%%%%%%%%%%%%%%%%%%%%%%%%%%%%%%%%%%%%%%%%%%%%%%%%%%%%%%%%%%%%%%%%%%%%%%%%%%
%%                            TARGET LIST                                  %% 
%% Target list with coordinates, or intervals in R.A. and Decl.            %%
%% of (sample of) objects:                                                 %%
%%                                                                         %%
%%%%%%%%%%%%%%%%%%%%%%%%%%%%%%%%%%%%%%%%%%%%%%%%%%%%%%%%%%%%%%%%%%%%%%%%%%%%%
%%                                                                         %%
%% Please fill in the relevant information for one object in the {   }     %%
%% fields below. Duplicate the line up to 20 objects; otherwise give a     %%
%% sample list and describe the full target list in the Remarks.           %%
%%                                                                         %%
%% Please specify the passband in which object magnitudes are given        %%
%%                                                                         %%
%% Targets are specified with the \target{#1}{#2}{#3}{#4}{#5}{#6}{#7}      %%
%% command, where the 7 arguments are:                                     %%
%%                                                                         %%
%%  #1: Run IDs (as defined box 6) for which target is relevant            %%
%%  #2: Short target name                                                  %%
%%  #3: Right ascension (2000)                                             %%
%%  #4: Declination (2000)                                                 %%
%%  #5: Target magnitude                                                   %%
%%  #6: Target diameter (in arcmin)                                        %% 
%%  #7: Additional Info                                                    %%
%%                                                                         %%
%%   Example:                                                              %%
%%                                                                         %%
%%   \target{A}{M31}{00:42:44}{+41:16:09}{V=4.4}{190}{$v_r=-300$ km/s}     %%
%%                                                                         %%
%%%%%%%%%%%%%%%%%%%%%%%%%%%%%%%%%%%%%%%%%%%%%%%%%%%%%%%%%%%%%%%%%%%%%%%%%%%%%
\target{A}{NGC 5963}{15 33 27.8}{+56 33 35}{B=12.70}{1.52}{} % duplicate as needed
%%%%%%%%%%%%%%%%%%%%%%%%%%%%%%%%%%%%%%%%%%%%%%%%%%%%%%%%%%%%%%%%%%%%%%%%%%%%%
%% Please type any Remarks on the target list in the {   } field below: 
%%
\remark{ }
%%%%%%%%%%%%%%%%%%%%%%%%%%%%%%%%%%%%%%%%%%%%%%%%%%%%%%%%%%%%%%%%%%%%%%%%%%%%%
\end{objlistpage}

\begin{qualitypage}[][]{}
%%%%%%%%%%%%%%%%%%%%%%%%%%%%%%%%%%%%%%%%%%%%%%%%%%%%%%%%%%%%%%%%%%%%%%%%%%%%%
%%                       BACKUP PROGRAMME
%%
% For projects needing excellent image quality or photometric conditions, 
% give a short description of possible backup programme. A backup programme 
% is also needed in case an unfavourable wind speed or direction may prevent
% you from executing your main programme.    
%
% - If no backup programme is provided, justify why none is needed. Failure
%   to ensure that the telescope will be used productively under all 
%   circumstances will lower the rating of your proposal!
%
% - The instrumental setup for the backup programme should normally be the 
%   same as for the main programme; however, standby instrumentation (e.g. 
%   StanCam or FIES) may be used instead. Please specify the instrument you 
%   will use.
% 
% Please type your description in the { } field below:
%
\backup{ }
%%%%%%%%%%%%%%%%%%%%%%%%%%%%%%%%%%%%%%%%%%%%%%%%%%%%%%%%%%%%%%%%%%%%%%%%%%%%%
\end{qualitypage}


%%%%%%%%%%%%%%%%%%%%%%%%%%%%%%%%%%%%%%%%%%%%%%%%%%%%%%%%%%%%%%%%%%%%%%%%%%%%%
%%
%%           !!!!!!!!!  PAGE 6 = Last page  !!!!!!!!!!!!!
%%
%%%%%%%%%%%%%%%%%%%%%%%%%%%%%%%%%%%%%%%%%%%%%%%%%%%%%%%%%%%%%%%%%%%%%%%%%%%%%
\newpage        % Page 6 
\putnumberLarge

\begin{prevobspage}[][]{}
%%%%%%%%%%%%%%%%%%%%%%%%%%%%%%%%%%%%%%%%%%%%%%%%%%%%%%%%%%%%%%%%%%%%%%%%%%%%%
%%              PREVIOUS OBSERVING PERIODS AND RESULTS
%%
% Please list your observing periods at the NOT within the last three years 
% and provide a brief status report on your analysis of the data. List also 
% your publications from NOT observations during the last three years. 
%
%%  Please type this information in the { } field below:
%% 
\prevobs{ }
%%%%%%%%%%%%%%%%%%%%%%%%%%%%%%%%%%%%%%%%%%%%%%%%%%%%%%%%%%%%%%%%%%%%%%%%%%%%%
\end{prevobspage}

\begin{extrapage}[][]{}
%%%%%%%%%%%%%%%%%%%%%%%%%%%%%%%%%%%%%%%%%%%%%%%%%%%%%%%%%%%%%%%%%%%%%%%%%%%%%
%%              ADDITIONAL INFO
%%
% Use this box to provide any additional information/remarks not covered 
% by the items above, e.g. related proposals to other telescopes, dates to 
% be avoided for the observations, etc. Do not use if not necessary.  
%
%%  Please type in the { } field below:
%% 
\addinfo{ }
%%%%%%%%%%%%%%%%%%%%%%%%%%%%%%%%%%%%%%%%%%%%%%%%%%%%%%%%%%%%%%%%%%%%%%%%%%%%%
\end{extrapage}


\end{document}
%%%%%%%%%%%%%%%%%%%%%%%%%%%%%%%%%%%%%%%%%%%%%%%%%%%%%%%%%%%%%%%%%%%%%%%%%%%%%
% 
%                       End of template file
% 
%%%%%%%%%%%%%%%%%%%%%%%%%%%%%%%%%%%%%%%%%%%%%%%%%%%%%%%%%%%%%%%%%%%%%%%%%%%%%

                    
\begin{document}
\def\propnumber{}                       % Only to be filled in by NOT staff
\putnumberHuge
\nothead

%%%%%%%%%%%%%%%%%%%%%%%%%%%%%%%%%%%%%%%%%%%%%%%%%%%%%%%%%%%%%%%%%%%%%%%%%%%%%
%%                                                                         %%
%%                    *** NOTE TO APPLICANTS ***                           %%
%%                                                                         %%
%%%%%%%%%%%%%%%%%%%%%%%%%%%%%%%%%%%%%%%%%%%%%%%%%%%%%%%%%%%%%%%%%%%%%%%%%%%%%
%%                                                                         %%
%%          NOT PROPOSAL TEMPLATE FILE FOR OBSERVING PERIOD 48             %%
%%                                                                         %%
%%                   OCTOBER 1, 2013 - APRIL 1, 2014                       %%
%%                                                                         %%
%%                                                                         %%
%% Please take care to fill in the fields of this proposal form as         %%
%% indicated, following the instructions and advice provided in the        %%
%% header of each section. Run LaTex on your completed form and verify     %%
%% the result before submitting to check that it runs correctly and        %%
%% do not overfill any of the boxes, and that it produces a total of no    %%
%% more than six (6) printed pages.                                        %%
%%                                                                         %%
%% Be sure to always use the correct version of the not-style file for     %%
%% the period in question. This template is only valid for period 48.      %%
%%                                                                         %%
%% Never change the format of the template or style file. Proposals        %%
%% that do not comply with the correct version of the style and            %%
%% template files will be rejected.                                        %%
%%                                                                         %%
%% Name the file PIname.tex (e.g. johanson.tex) and any figure file(s)     %%
%% as PInameA.ps (and PInameB.ps). After verifying that the proposal       %%
%% can be properly processed, submit the file(s) as (separate) attached    %%
%% file(s) by e-mail to the address:                                       %%
%%                                                                         %%
%%                          proposal@not.iac.es                            %%
%%                                                                         %%
%% with the word ``Proposal'' in the 'Subject' field and as text in the    %%
%% body of the message. The latter is important when you use certain       %%
%% mailers as the proposal might otherwise not be parsed correctly by      %%
%% our automatic procedure.                                                %%
%%                                                                         %%
%% Do not compress the files or combine them in a tar file. Do not         %%
%% submit the style file.                                                  %%
%%                                                                         %%
%% Any questions regarding the proposal procedure may be submitted to      %%
%% the same e-mail address, giving ``Question'' as the 'Subject'.          %%
%%                                                                         %%
%% If you submit more than one proposal, please name the file              %%
%% PIname1.tex, PIname2.tex, etc., and any figure files accordingly.       %%
%% Only one proposal should be submitted at the time.                      %%
%%                                                                         %%
%% For more information on the Nordic Optical Telescope see:               %%
%%                                                                         %%
%%                       http://www.not.iac.es/                            %%
%%                                                                         %%
%%%%%%%%%%%%%%%%%%%%%%%%%%%%%%%%%%%%%%%%%%%%%%%%%%%%%%%%%%%%%%%%%%%%%%%%%%%%%


\begin{titpage}{}
%%%%%%%%%%%%%%%%%%%%%%%%%%%%%%%%%%%%%%%%%%%%%%%%%%%%%%%%%%%%%%%%%%%%%%%%%%%%%
%                          PROPOSAL TITLE
% Type title of proposal in the { } below - one line only!
%
\proptitle{Measuring the Rotation Curve of the Elusive NGC 5963: The Adventure.}
%%%%%%%%%%%%%%%%%%%%%%%%%%%%%%%%%%%%%%%%%%%%%%%%%%%%%%%%%%%%%%%%%%%%%%%%%%%%%
\end{titpage}


\begin{abspage}[][]{}
%%%%%%%%%%%%%%%%%%%%%%%%%%%%%%%%%%%%%%%%%%%%%%%%%%%%%%%%%%%%%%%%%%%%%%%%%%%%%
%%                             ABSTRACT
%
% Please type the Abstract of the proposal into the { } below
% Do not exceed the space provided 
%
\propabstract{We propose to use the NOT telescope to obtain deep surface spectroscopy
and photometry for the low surface brightness galaxy NGC 5963. We will measure the H$\alpha$
line emission from the ionized gas around stars to unfold
its rotation curve. A light+dark mass profile fit to the measured spectroscopic data will be done
to determine the galaxy’s stellar/dark matter components. We intend to use the photometric data
to investigate the targets underluminous nature}
%%%%%%%%%%%%%%%%%%%%%%%%%%%%%%%%%%%%%%%%%%%%%%%%%%%%%%%%%%%%%%%%%%%%%%%%%%%%%
\end{abspage}


\begin{adrinvpage}{}
%%%%%%%%%%%%%%%%%%%%%%%%%%%%%%%%%%%%%%%%%%%%%%%%%%%%%%%%%%%%%%%%%%%%%%%%%%%%%
%%                      PRINCIPAL INVESTIGATOR
%
% Name and address of Principal Investigator (PI)
% 
% NB: The PI has full responsibility for the content of this proposal!
%
% Please fill in the appropriate { } below:
%
\piname{Simon Perfunkel}               % name of PI
\piinst{Stockholm University}               % PI institute
\picoun{SE}               % PI country (ISO code: DK,FI,IS,NO,SE,..) 
\piaddr{106 91 Stockholm}               % PI postal address
\piteln{3141592}               % PI telephone number
\pifaxn{really? 90s where are you?}               % PI fax number
\pimail{soundofscience@robinson.com}               % PI email address
%%%%%%%%%%%%%%%%%%%%%%%%%%%%%%%%%%%%%%%%%%%%%%%%%%%%%%%%%%%%%%%%%%%%%%%%%%%%%
\end{adrinvpage}


\begin{coinvestpage}{}
%%%%%%%%%%%%%%%%%%%%%%%%%%%%%%%%%%%%%%%%%%%%%%%%%%%%%%%%%%%%%%%%%%%%%%%%%%%%%
%%                          CO-INVESTIGATORS
% 
% Name and institute of co-investigators
% Please fill in the { } { } fields below (2 Co-Is per line):
%
% There is room for up to 10 CoIs. Even if the project involves more 
% than 10 CoIs, please do not list more than 10 CoIs 
%
%        {Name1, Institute1}  {Name2, Institute2}
%
\coinvest{Oh Long Johnson}{106 91 Stockholm }  % {Name1, Institute1}   {Name2, Institute2} 
\coinvest{Oh Don Piano}{ 106 91 Stockholm}  % {Name3, Institute3}   etc
\coinvest{ }{ }
\coinvest{ }{ }
\coinvest{ }{ }
% 
%%%%%%%%%%%%%%%%%%%%%%%%%%%%%%%%%%%%%%%%%%%%%%%%%%%%%%%%%%%%%%%%%%%%%%%%%%%%%
\end{coinvestpage}


\begin{omthesispage}{}
%%%%%%%%%%%%%%%%%%%%%%%%%%%%%%%%%%%%%%%%%%%%%%%%%%%%%%%%%%%%%%%%%%%%%%%%%%%%%
%%                        THESIS PROJECTS
%
% If this proposal concerns a PhD thesis work at Nordic Institute,
% please provide: name of the student, institute, name of supervisor, 
% and expected time of completion.
%
% Please type in the { } field below:
\thesis{} 
%%%%%%%%%%%%%%%%%%%%%%%%%%%%%%%%%%%%%%%%%%%%%%%%%%%%%%%%%%%%%%%%%%%%%%%%%%%%%
\end{omthesispage}


\begin{nightspage}{}
%%%%%%%%%%%%%%%%%%%%%%%%%%%%%%%%%%%%%%%%%%%%%%%%%%%%%%%%%%%%%%%%%%%%%%%%%%%%%
%%                       REQUESTED OBSERVING RUN(S)
%% 
%% NB: In the following, an ``Observing run'' is a single, contiguous block 
%% of time with a single instrument. If your project requires more than one
%% such run, e.g. at different times and/or with different instruments, 
%% please identify each run as A, B, C,... and specify your requirements 
%% for each on a separate line as specified below.
%%
%%%%%%%%%%%%%%%%%%%%%%%%%%%%%%%%%%%%%%%%%%%%%%%%%%%%%%%%%%%%%%%%%%%%%%%%%%%%%
%%
%% Give requested no. of nights/hours as a number (not word) and specify
%% the unit: as N (nights) or H (hours) 
%%
%% Indicate desired Moon phases as D=dark/G=grey/N=no restriction
%% 
%% Indicate the seeing requirements: 0.7 (max 0.7 arcsec seeing), 
%%      1.0 (max 1.0 arcsec), 1.3 (max 1.3 arcsec), or N (no restriction)
%%
%% If the programme require specific sky condition, enter these here
%%      P = photometric conditions required
%%      C = clear conditions required
%%      T = thin clouds/cirrus acceptable
%% 
%% Please fill in the relevant information in the {   } fields below, and 
%% duplicate the entire block if more than one run is requested
%%
%% N.B. A maximum of 6 runs per proposal can be specified
%%
%%%%%%%%%%%%%%%%%%%%%%%%%%%%%%%%%%%%%%%%%%%%%%%%%%%%%%%%%%%%%%%%%%%%%%%%%%%%%
%%
%%%%%%%%%%%%%%%%%%%%%%%%%%%%%% Run A %%%%%%%%%%%%%%%%%%%%%%%%%%%%%%%%%%%%%%%%
%
\nrunid{A }      % Put your run id (A, B, C, ...) here
\ninstr{ALFOSC }      % Put instrument name here
\ntimer{7 H }      % Put requested time, in numbers, with unit (N or H), e.g 5 N
\nmonth{May }      % Put preferred month(s) here
\nmoonp{B }      % Put requested moon phase here: D, G or N
\nsemax{1.5 }      % Put maximum allowed seeing here: (0.7,1.0,1.5,N)
\nskyco{P }      % Put required photometric condition here
%
%%%%%%%%%%%%%%%%%%%%%%%%%%%%%%%%%%%%%%%%%%%%%%%%%%%%%%%%%%%%%%%%%%%%%%%%%%%%%
\end{nightspage}

\begin{numnightspage}{}
%%%%%%%%%%%%%%%%%%%%%%%%%%%%%%%%%%%%%%%%%%%%%%%%%%%%%%%%%%%%%%%%%%%%%%%%%%%%%
%%                  TIME BEFORE/AFTER PRESENT REQUEST
%%
% Number of nights already awarded to project. More details, e.g. on 
% instrumentation and outcome of previous observations can be given in box 17
%
% Please type in the { } field below:
% 
\numalr{0}
%
% Number of nights needed to complete project (excluding those requested).
%
\numrem{1}
%%%%%%%%%%%%%%%%%%%%%%%%%%%%%%%%%%%%%%%%%%%%%%%%%%%%%%%%%%%%%%%%%%%%%%%%%%%%%
\end{numnightspage}


\begin{servicepage}[][]{}
%%%%%%%%%%%%%%%%%%%%%%%%%%%%%%%%%%%%%%%%%%%%%%%%%%%%%%%%%%%%%%%%%%%%%%%%%%%%%
%%                         SERVICE CONSTRAINTS
%
%
% All projects will be considered for execution in service/queue mode. In
% case your project can not be done in service/queue mode, please give a
% justification in the { } field below:
\service{Yes }
%%%%%%%%%%%%%%%%%%%%%%%%%%%%%%%%%%%%%%%%%%%%%%%%%%%%%%%%%%%%%%%%%%%%%%%%%%%%%
\end{servicepage}


\begin{schedpage}[][]{}
%%%%%%%%%%%%%%%%%%%%%%%%%%%%%%%%%%%%%%%%%%%%%%%%%%%%%%%%%%%%%%%%%%%%%%%%%%%%%
%%                         SCHEDULING CONSTRAINTS
%
%
% Any other special constraints on the scheduling?
%
% E.g. time critical scheduling, or required baseline vs phase coverage
% for monitoring programs, response time for target of opportunity, 
% simultaneous observations, impossible dates, etc... 
%
% Please type in the { } field below:
%
\schedconstr{No }
%%%%%%%%%%%%%%%%%%%%%%%%%%%%%%%%%%%%%%%%%%%%%%%%%%%%%%%%%%%%%%%%%%%%%%%%%%%%%
\end{schedpage}


%%%%%%%%%%%%%%%%%%%%%%%%%%%%%%%%%%%%%%%%%%%%%%%%%%%%%%%%%%%%%%%%%%%%%%%%%%%%%
%%
%%              !!!!!!!!!!!!! PAGE 2 !!!!!!!!!!!!!
%%
%%%%%%%%%%%%%%%%%%%%%%%%%%%%%%%%%%%%%%%%%%%%%%%%%%%%%%%%%%%%%%%%%%%%%%%%%%%%%
\newpage        % Page 2  
\putnumberLarge

\begin{scienpage}[][]{}
%%%%%%%%%%%%%%%%%%%%%%%%%%%%%%%%%%%%%%%%%%%%%%%%%%%%%%%%%%%%%%%%%%%%%%%%%%%%%
%%                     SCIENTIFIC JUSTIFICATION                            %%
%%                                                                         %%
%% Note: This should be self-contained and not refer to previous proposals.%%
%%                                                                         %%
%% Describe first the scientific background and main goals of the proposal.%%
%% As OPC members cannot be experts in every field, it is CRUCIAL that you %% 
%% outline the general scientific context CLEARLY and in a manner that can %% 
%% be understood also by a non-specialist in your field.                   %% 
%%                                                                         %% 
%% Then argue - equally clearly! - how your proposed observing project     %% 
%% will contribute significantly to advancing our understanding of the     %% 
%% issue. Key references to the literature should be given.                %% 
%%                                                                         %%
%% Finally, describe how the data reduction and scientific analysis will   %%
%% be done, and document that the team possesses the required expertise.   %%
%%                                                                         %%
%% All text and figures should fit on the following two pages (page 2 and  %%
%% page 3), but text may spill over on page 3. All figures and references  %%
%% should be placed on page 3.                                             %%
%%                                                                         %%
%%%%%%%%%%%%%%%%%%%%%%%%%%%%%%%%%%%%%%%%%%%%%%%%%%%%%%%%%%%%%%%%%%%%%%%%%%%%%
%%
% Please type your text into the { } field below
\scijust{Dark matter was first termed in a paper from 1933 by Fritz
Zwicky. He used the virial theorem to calculate the gravitational mass of
the galaxies in the Coma cluster and found a discrepancy between the
measured mass and their expected luminosity. He referred to this
"missing mass" as "dunkle materie". Today scientists have accumulated
convincing evidence of dark matter from many independent theoretical
and observational studies, such
as galaxy rotation curves (Rubin, V. C et al. 1980 ApJ 238 , 471)
, velocity dispersions (Faber, S. M.; Jackson, R. E. 1976 ApJ 204, 668)
, gravitational lensing (Refregier, Alexandre 2003 ARA&A 41, 645)
, measurements of the cosmic microwave background (Hinshaw, G et al. 2009 ApJS 180, 225)
, baryon acoustic oscillations (Percival, Will J et al. 2007 MNRAS 381, 1053)
, supernovae distance measurements (Komatsu, E et al. 2009 ApJS 180, 330)
, Lyman-alpha forest measurements (Viel, Matteo et al. 2009 MNRAS 399L , 39) of distant galaxies
 and in structure formation scenarios (Springel, Volker et al. 2005 Natur 435, 629).
\par
According to the successful Planck mission (Planck Collaboration 2013 arXiv 1303, 5062), the dark matter
part of the total energy in the universe is a staggering 26.8\%
compared with the 4.9\% of ordinary matter. Even though the consensus among scientist
today is that dark matter consists of Weakly Interacting Massive Particles
(WIMPs), no official detections of these elusive particles have been made and
the hunt for these particles is one of the major undertakings of modern physics.
In what better way to make aspiring astronomy students more comfortable
with observational techniques, than to let them discover for
themselves the existence of dark matter? The reproducibility of science is
after all one of the fundamental pillars of science itself.
By the guidance of past and present mentors we therefore propose to use the
NOT telescope
to measure the rotation curve of NGC 5963, fit a light+dark mass profile 
to the acquired data and determine the stellar/dark matter mass components of
this galaxy. The rotation curve will not be derived by direct measurements 
of the velocity of the stars in the galaxy but will be inferred by measurements
of the H$\alpha$ line emitted from the ionized gas around stars.   \\

\noindent \textbf{The need for a new observation and its selection:} The target in consideration, NGC 5963 is of the type Low Surface Brightness (LSB) galaxy (Romanishin, Strom \& Strom 1982 ApJ 252, 77) which are usual targets for dark matter studies due to their peculiar mass to light ratio. NGC 5963 is no exception to such studies (e.g. Bosma et al. 1988, A\& A 198, 100). However, the latest  direct observations of NGC 5963 we could find in the literature were taken over a decade ago (Simon et al. 2004 ASPC, 327, 18), which speaks for the acquisition of newer observations. 
\par
Another virtue of the selected target is that if photometric images of good enough quality of the galaxy is provided, they might give insight to its anomaly underluminous nature (Zackrisson et al. in preparation). NGC 5963 strongly deviates from the expected Tully-Fisher (TF) relation (Springob et al. 2007 ApJS, 172, 599) by being underluminous and/or having far greater non-baryonic mass than expected. Newer observations with the NOT may aid in uncovering why this is so.
 
   }
%%%%%%%%%%%%%%%%%%%%%%%%%%%%%%%%%%%%%%%%%%%%%%%%%%%%%%%%%%%%%%%%%%%%%%%%%%%%%
\end{scienpage}

%%%%%%%%%%%%%%%%%%%%%%%%%%%%%%%%%%%%%%%%%%%%%%%%%%%%%%%%%%%%%%%%%%%%%%%%%%%%%
%%
%%              !!!!!!!!!!!!! PAGE 3 !!!!!!!!!!!!!
%%
%%%%%%%%%%%%%%%%%%%%%%%%%%%%%%%%%%%%%%%%%%%%%%%%%%%%%%%%%%%%%%%%%%%%%%%%%%%%%
%\newpage        % Page 3 
%\putnumberLarge

%\begin{scienpagec}[][]{}
%%%%%%%%%%%%%%%%%%%%%%%%%%%%%%%%%%%%%%%%%%%%%%%%%%%%%%%%%%%%%%%%%%%%%%%%%%%%%
%%              SCIENTIFIC JUSTIFICATION (CONTINUED)                       %%
%%                                                                         %%
%% Place References and any Figures here.                                  %%
%%                                                                         %%
%%%%%%%%%%%%%%%%%%%%%%%%%%%%%%%%%%%%%%%%%%%%%%%%%%%%%%%%%%%%%%%%%%%%%%%%%%%%%
% 
% Please type the rest of your text into the { } field below:
%\scijustc{ }
%
%%%%%%%%%%%%%%%%%%%%%%%%%%%%%%%%%%%%%%%%%%%%%%%%%%%%%%%%%%%%%%%%%%%%%%%%%%%%%
%%                                FIGURES:                                 %%
%%                                                                         %%
%% Up to two postscript figures may be included                            %%
%%                                                                         %%
%% NB colour figures are not supported. All figures will be printed in     %%
%%    black and white and any colour information in the figures will be    %%
%%    disregarded                                                          %%
%%                                                                         %%
%% To enter a figure, uncomment the lines below, fill in the name of       %%
%% your .eps file, and provide a short caption where indicated             %%
%%                                                                         %%
%%%%%%%%%%%%%%%%%%%%%%%%%%%%%%%%%%%%%%%%%%%%%%%%%%%%%%%%%%%%%%%%%%%%%%%%%%%%%
%%
%%
%% NB: There should be no spaces in the \psfig argument below 
%% -  otherwise psfig will fail!
%%
%
%   Figure 1:   change 'PInameA.eps' to name of the file containing your 
%               figure. You may need to adjust the width and angle below,
%               and possibly the bounding box in the postscript-file.
%               Provide a short caption in the \captone{} field. Be sure 
%               to leave an empty line between the psfig and caption
%               command
%
%\psfig{file=PInameA.eps,width=12cm,angle=0,clip=}
%
%\captone{}
%
%
%   Figure 2:   change 'PInameB.eps' to name of the file containing your 
%               figure. You may need to adjust the width and angle below,
%               and possibly the bounding box in the postscript-file.
%               Provide a short caption in the \capttwo{} field. Be sure 
%               to leave an empty line between the psfig and caption
%               command
%
%\psfig{file=PInameB.eps,width=12cm,angle=0,clip=}
%
%\capttwo{}
%
%
%%%%%%%%%%%%%%%%%%%%%%%%%%%%%%%%%%%%%%%%%%%%%%%%%%%%%%%%%%%%%%%%%%%%%%%%%%%%%
%\end{scienpagec}



%%%%%%%%%%%%%%%%%%%%%%%%%%%%%%%%%%%%%%%%%%%%%%%%%%%%%%%%%%%%%%%%%%%%%%%%%%%%%
%%
%%              !!!!!!!!!!!!! PAGE 4 !!!!!!!!!!!!!
%%
%%%%%%%%%%%%%%%%%%%%%%%%%%%%%%%%%%%%%%%%%%%%%%%%%%%%%%%%%%%%%%%%%%%%%%%%%%%%%
\newpage  % Page 4 
\putnumberLarge

\begin{justificpage}[][]{}
%%%%%%%%%%%%%%%%%%%%%%%%%%%%%%%%%%%%%%%%%%%%%%%%%%%%%%%%%%%%%%%%%%%%%%%%%%%%%
%%                   TECHNICAL JUSTIFICATION
%
%  Describe how the observations will be performed, so the feasibility of the 
%  project becomes clear. This should be self-contained and not rely on 
%  information given elsewhere. Describe the S/N ratio calculations you have 
%  used to justify the number of nights and the Moon phases you request 
%  (no correction for expected weather conditions should be applied). Also,
%  describe why the instrumental set-up is adequate for the objective of the
%  proposed observations (e.g., if the resolution of spectra is adequate to 
%  resolve the spectroscopic features you want to study).
%
% Please type in the { } field below:
\techjust{Because the target NGC 5963 is of low surface brightness nature, we aim for fairly deep spectra (1.5-2h) in order to obtain reducible data from the outer skirts of the galaxy. We used spectral data from the \emph{Sloand Digital Sky Survey} (SDSS) in order to calculate the flux of the H$\alpha$ line which we will also look at. The flux was calculated from the equivalent width and continuum of the line and then converted into magnitude using Vega as a reference star. A somewhat average of the known flux from Vega at similar wavelengths as the 6565\AA  we are interested in was used in the conversion and we get a magnitude for our line of interest as $m_{H\alpha} = 19.3$. Given an integration time of 2h we would achieve a \emph{Signal to Noise ratio} $(S/N)$ of $\approx 8.5$ according to the \emph{Nordic Optical Telescope}'s own calculator using the $\#8$ grism. In case that seeing is bad due to high air mass we would instead use the $\#7$ grism in order to boost our S/N very slightly, sacrificing a bit of a resolution. We assume that we may need up to 30 min of overhead time and thus need 2.5h in total for the spectroscopy. \\

\noindent Although dark conditions would have been optimal for our photometric observations due to the target's faint surface brightness, we can make due with a bright full moon as well. Just as with the spectroscopy we aim for very deep imaging in order to reach the very faintest outskirts of the galaxy and enable a closer study of the outer morphology. Using the SDSS photometric data of NGC 5963 and calculating the surface brightness for different sizes of apertures we get a central surface brightness of $Sb_c = 20.35$ and on average $Sb_{avg} = 22.1$. Again using the NOT exposure time calculator tool we require an exposure time of at least 20 min, which would provide us with a $S/N> 3$ in all bands \(UBVRI\) without risk of saturation. Given 20 minutes on average for each of the 5 bands and including 10 min of overhead we would require a total time of 2.5h for our photometry observations. \\

\noindent As we only have 1 target but 5 bands + spectroscopy, the total time needed would be 5h including overhead for our observations. Our target NGC 5963 possess the fortunate position in the sky during the designated days for observations that it is visible during the entire nights. Thus, we do not demand 5 consecutive hours of observations but are capable of splitting it up into several segments. \\

\noindent Should opportunity present itself for more time at the telescope we would use this time for photometric and spectrophotometric calibrations. If time would allow it we would also try to obtain even deeper photometry.}
%%%%%%%%%%%%%%%%%%%%%%%%%%%%%%%%%%%%%%%%%%%%%%%%%%%%%%%%%%%%%%%%%%%%%%%%%%%%%
\end{justificpage}




\begin{instrempage}[][]{}
%%%%%%%%%%%%%%%%%%%%%%%%%%%%%%%%%%%%%%%%%%%%%%%%%%%%%%%%%%%%%%%%%%%%%%%%%%%%%
%
%
%                       INSTRUMENT CONFIGURATIONS:
%
% Please specify the instrument configurations you want to use as fully as 
% possible, using the setup definitions provided below. You must uncomment
% the relevant setup definitions, i.e. remove the `%' sign in front of them,
% in order to make these lines take effect.
%
% Uncomment only the lines related to instrument configuration(s) needed
% for the acquisition of your planned  observations. Detailed information 
% on the available instruments is provided in the relevant users' manuals 
% (see http://www.not.iac.es/observing/proposals/).
%
% You also need to specify the Run ids (as defined in box 6) for which the 
% selected setups are valid. Put the run ID in the first {}, for example:
%
%       \NOTconfig{A}{ALFOSC}{Standard imaging-filters}{UBVRi}
%
%
%-----------------------------------------------------------------------
%----------------------------- ALFOSC ----------------------------------
%-----------------------------------------------------------------------
%
\NOTconfig{}{ALFOSC}{Standard imaging-filters}{UBVRi}
%\NOTconfig{}{ALFOSC}{Imaging-filters for ALFOSC}{provide filter No.}
%\NOTconfig{}{ALFOSC}{Imaging-filters for FASU}{provide filter No.}
%
%\NOTconfig{}{ALFOSC}{Fast-photometry}{Multi-windowing mode}
%
%\NOTconfig{}{ALFOSC}{Lin-Pol-imaging}{Polaroids}
%\NOTconfig{}{ALFOSC}{Lin-Pol-imaging}{Calcite+half-wave-plate}
%\NOTconfig{}{ALFOSC}{Cir-Pol-imaging}{Calcite+quarter-wave-plate}
%\NOTconfig{}{ALFOSC}{Lin-Pol-imaging}{WeDoWo}
%
%\NOTconfig{}{ALFOSC}{Lin-Pol-spectroscopy}{Calcite+half-wave-plate}
%\NOTconfig{}{ALFOSC}{Cir-Pol-spectroscopy}{Calcite+quarter-wave-plate}
%\NOTconfig{}{ALFOSC}{Pol-spectroscopy}{give grism number(s)}
%\NOTconfig{}{ALFOSC}{Pol-spectroscopy}{Polarimetric slitlet\#1.0"}
%\NOTconfig{}{ALFOSC}{Pol-spectroscopy}{Polarimetric slitlet\#1.4"}
%\NOTconfig{}{ALFOSC}{Pol-spectroscopy}{Polarimetric slitlet\#1.8"}
%
%\NOTconfig{}{ALFOSC}{Spectroscopy}{ADC}
%
%\NOTconfig{}{ALFOSC}{Spectro-long-slit}{give grism number(s)}
\NOTconfig{}{ALFOSC}{Spectro-long-slit}{1.0, 1.3}
\NOTconfig{}{ALFOSC}{Spectro-long-slit}{Grism \#7, \#8}
%\NOTconfig{}{ALFOSC}{Spectro-long-slit}{provide 2nd-order blocking filter No.}
%
%\NOTconfig{}{ALFOSC}{Multi-Object-Spectro}{provide HERE the number of masks}
%\NOTconfig{}{ALFOSC}{Multi-Object-Spectro}{give grism number(s)}
%\NOTconfig{}{ALFOSC}{Multi-Object-Spectro}{provide required slitwidth(s)}
%\NOTconfig{}{ALFOSC}{Multi-Object-Spectro}{Pre-imaging required}
%
%\NOTconfig{}{ALFOSC}{Spectro-Echelle}{Echelle Grism\#9}
%\NOTconfig{}{ALFOSC}{Spectro-Echelle}{Echelle Grism\#13}
%\NOTconfig{}{ALFOSC}{Spectro-Echelle}{0.7arcsec-slit}
%\NOTconfig{}{ALFOSC}{Spectro-Echelle}{0.8arcsec-slit}
%\NOTconfig{}{ALFOSC}{Spectro-Echelle}{1.0arcsec-slit}
%\NOTconfig{}{ALFOSC}{Spectro-Echelle}{1.2arcsec-slit}
%\NOTconfig{}{ALFOSC}{Spectro-Echelle}{1.6arcsec-slit}
%\NOTconfig{}{ALFOSC}{Spectro-Echelle}{1.8arcsec-slit}
%\NOTconfig{}{ALFOSC}{Spectro-Echelle}{2.2arcsec-slit}
%\NOTconfig{}{ALFOSC}{Spectro-Echelle}{Cross-disperser\#10}
%\NOTconfig{}{ALFOSC}{Spectro-Echelle}{Cross-disperser\#11}
%\NOTconfig{}{ALFOSC}{Spectro-Echelle}{Cross-disperser\#12}
%
%
%-----------------------------------------------------------------------
%----------------------------- NOTCam-----------------------------------
%-----------------------------------------------------------------------
%
%\NOTconfig{}{NOTCam}{Standard imaging-filters}{ZYJHKs}
%\NOTconfig{}{NOTCam}{Imaging-filters for NOTCam}{provide filter No.}
%
%\NOTconfig{}{NOTCam}{Imaging}{Wide-field camera}
%\NOTconfig{}{NOTCam}{Imaging}{High-res camera}
%
%\NOTconfig{}{NOTCam}{Spectro-long-slit}{Grism\#1}
%\NOTconfig{}{NOTCam}{Spectro-long-slit}{Wide-field camera slit 0.6''}
%\NOTconfig{}{NOTCam}{Spectro-long-slit}{High-res camera slit 0.2''}
%
%\NOTconfig{}{NOTCam}{Spectroscopy}{ADC}
%
%\NOTconfig{}{NOTCam}{Spectroscopy}{Wide-field camera}
%\NOTconfig{}{NOTCam}{Spectroscopy}{High-res camera}
%
%
%-----------------------------------------------------------------------
%----------------------------- FIES ------------------------------------
%-----------------------------------------------------------------------
%
% Please note that the simultaneous-ThAr mode can only be used in
% combination with either the High-res or Med-res fiber.
%
% Please note that the spec-pol mode can only be used in combination
% with the Med-Res fiber. The useful wavelength range is 370-630 nm.
%
%\NOTconfig{}{FIES}{Spectro-Echelle}{Low-Res Fiber}
%\NOTconfig{}{FIES}{Spectro-Echelle}{Med-Res Fiber}
%\NOTconfig{}{FIES}{Spectro-Echelle}{High-Res Fiber}
%
%\NOTconfig{}{FIES}{Spectro-Echelle}{Simultaneous-ThAr mode}
%
%\NOTconfig{}{FIES}{Spectro-Echelle}{Spec-Pol mode}
%
%\NOTconfig{}{FIES}{Spectro-Echelle}{ADC}
%
%
%-----------------------------------------------------------------------
%----------------------------- MOSCA -----------------------------------
%-----------------------------------------------------------------------
%
%\NOTconfig{}{MOSCA}{Standard imaging-filters}{UBVRI}
%\NOTconfig{}{MOSCA}{Standard imaging-filters}{ugriz}
%\NOTconfig{}{MOSCA}{Imaging-filters for FASU}{provide filter No.}
%
%
%-----------------------------------------------------------------------
%----------------------------- StanCam ---------------------------------
%-----------------------------------------------------------------------
%
%\NOTconfig{}{StanCam}{Standard imaging-filters}{UBVRIz}
%\NOTconfig{}{StanCam}{Imaging-filters for StanCam}{provide filter No.}
%
%
%-----------------------------------------------------------------------
%----------------------------- TurPol ----------------------------------
%-----------------------------------------------------------------------
%
% TurPol is not a common-user instrument. Normal support at the telescope 
% is provided, but only limited trouble-shooting can be made by NOT staff.
%
%\NOTconfig{}{TurPol}{Lin-Pol}{Half-wave-plate}
%\NOTconfig{}{TurPol}{Cir+Lin-Pol}{Quarter-wave-plate}
%\NOTconfig{}{TurPol}{Standard photometry}{UBVRI}
% 
%
%-----------------------------------------------------------------------
%----------------------------- Visitor Instruments ---------------------
%-----------------------------------------------------------------------
%
% If you wish to use a special instrument for the project, please contact
% the NOT director (Johannes Andersen, email: ja@astro.ku.dk), or the Head
% of Operations (Thomas Augusteijn, email: tau@not.iac.es) well in advance
% of submitting your proposal.
%
%\NOTconfig{}{Name of instrument}{key capability}{link to instrument URL}
%
%
%%%%%%%%%%%%%%%%%%%%%%%%%%%%%%%%%%%%%%%%%%%%%%%%%%%%%%%%%%%%%%%%%%%%%%%%%%%%%
%%                      REMARKS ON INSTRUMENT SETUP
%%
% If you have any special remarks or requirements, e.g. use of non standard
% observing modes, own equipment etc., please uncomment the %\remark{} line 
% below and provide the information in the {} field. 
% If necessary, more detailed information can be provided in BOX 18.
%
%\remark{} 
%%%%%%%%%%%%%%%%%%%%%%%%%%%%%%%%%%%%%%%%%%%%%%%%%%%%%%%%%%%%%%%%%%%%%%%%%%%%%
\end{instrempage}


%%%%%%%%%%%%%%%%%%%%%%%%%%%%%%%%%%%%%%%%%%%%%%%%%%%%%%%%%%%%%%%%%%%%%%%%%%%%%
%%
%%              !!!!!!!!!!!!! PAGE 5 !!!!!!!!!!!!!
%%
%%%%%%%%%%%%%%%%%%%%%%%%%%%%%%%%%%%%%%%%%%%%%%%%%%%%%%%%%%%%%%%%%%%%%%%%%%%%%
\newpage        % Page 5 
\putnumberLarge

\begin{objlistpage}{}
%%%%%%%%%%%%%%%%%%%%%%%%%%%%%%%%%%%%%%%%%%%%%%%%%%%%%%%%%%%%%%%%%%%%%%%%%%%%%
%%                            TARGET LIST                                  %% 
%% Target list with coordinates, or intervals in R.A. and Decl.            %%
%% of (sample of) objects:                                                 %%
%%                                                                         %%
%%%%%%%%%%%%%%%%%%%%%%%%%%%%%%%%%%%%%%%%%%%%%%%%%%%%%%%%%%%%%%%%%%%%%%%%%%%%%
%%                                                                         %%
%% Please fill in the relevant information for one object in the {   }     %%
%% fields below. Duplicate the line up to 20 objects; otherwise give a     %%
%% sample list and describe the full target list in the Remarks.           %%
%%                                                                         %%
%% Please specify the passband in which object magnitudes are given        %%
%%                                                                         %%
%% Targets are specified with the \target{#1}{#2}{#3}{#4}{#5}{#6}{#7}      %%
%% command, where the 7 arguments are:                                     %%
%%                                                                         %%
%%  #1: Run IDs (as defined box 6) for which target is relevant            %%
%%  #2: Short target name                                                  %%
%%  #3: Right ascension (2000)                                             %%
%%  #4: Declination (2000)                                                 %%
%%  #5: Target magnitude                                                   %%
%%  #6: Target diameter (in arcmin)                                        %% 
%%  #7: Additional Info                                                    %%
%%                                                                         %%
%%   Example:                                                              %%
%%                                                                         %%
%%   \target{A}{M31}{00:42:44}{+41:16:09}{V=4.4}{190}{$v_r=-300$ km/s}     %%
%%                                                                         %%
%%%%%%%%%%%%%%%%%%%%%%%%%%%%%%%%%%%%%%%%%%%%%%%%%%%%%%%%%%%%%%%%%%%%%%%%%%%%%
\target{A}{NGC 5963}{15 33 27.8}{+56 33 35}{B=12.70}{1.52}{} % duplicate as needed
%%%%%%%%%%%%%%%%%%%%%%%%%%%%%%%%%%%%%%%%%%%%%%%%%%%%%%%%%%%%%%%%%%%%%%%%%%%%%
%% Please type any Remarks on the target list in the {   } field below: 
%%
\remark{Target coordinates refer to EquJ2000.}
%%%%%%%%%%%%%%%%%%%%%%%%%%%%%%%%%%%%%%%%%%%%%%%%%%%%%%%%%%%%%%%%%%%%%%%%%%%%%
\end{objlistpage}

\begin{qualitypage}[][]{}
%%%%%%%%%%%%%%%%%%%%%%%%%%%%%%%%%%%%%%%%%%%%%%%%%%%%%%%%%%%%%%%%%%%%%%%%%%%%%
%%                       BACKUP PROGRAMME
%%
% For projects needing excellent image quality or photometric conditions, 
% give a short description of possible backup programme. A backup programme 
% is also needed in case an unfavourable wind speed or direction may prevent
% you from executing your main programme.    
%
% - If no backup programme is provided, justify why none is needed. Failure
%   to ensure that the telescope will be used productively under all 
%   circumstances will lower the rating of your proposal!
%
% - The instrumental setup for the backup programme should normally be the 
%   same as for the main programme; however, standby instrumentation (e.g. 
%   StanCam or FIES) may be used instead. Please specify the instrument you 
%   will use.
% 
% Please type your description in the { } field below:
%
\backup{Fika. }
%%%%%%%%%%%%%%%%%%%%%%%%%%%%%%%%%%%%%%%%%%%%%%%%%%%%%%%%%%%%%%%%%%%%%%%%%%%%%
\end{qualitypage}


%%%%%%%%%%%%%%%%%%%%%%%%%%%%%%%%%%%%%%%%%%%%%%%%%%%%%%%%%%%%%%%%%%%%%%%%%%%%%
%%
%%           !!!!!!!!!  PAGE 6 = Last page  !!!!!!!!!!!!!
%%
%%%%%%%%%%%%%%%%%%%%%%%%%%%%%%%%%%%%%%%%%%%%%%%%%%%%%%%%%%%%%%%%%%%%%%%%%%%%%
\newpage        % Page 6 
\putnumberLarge

\begin{prevobspage}[][]{}
%%%%%%%%%%%%%%%%%%%%%%%%%%%%%%%%%%%%%%%%%%%%%%%%%%%%%%%%%%%%%%%%%%%%%%%%%%%%%
%%              PREVIOUS OBSERVING PERIODS AND RESULTS
%%
% Please list your observing periods at the NOT within the last three years 
% and provide a brief status report on your analysis of the data. List also 
% your publications from NOT observations during the last three years. 
%
%%  Please type this information in the { } field below:
%% 
\prevobs{No former observations have been made at the NOT by the PI nor Co-investigator.  }
%%%%%%%%%%%%%%%%%%%%%%%%%%%%%%%%%%%%%%%%%%%%%%%%%%%%%%%%%%%%%%%%%%%%%%%%%%%%%
\end{prevobspage}

\begin{extrapage}[][]{}
%%%%%%%%%%%%%%%%%%%%%%%%%%%%%%%%%%%%%%%%%%%%%%%%%%%%%%%%%%%%%%%%%%%%%%%%%%%%%
%%              ADDITIONAL INFO
%%
% Use this box to provide any additional information/remarks not covered 
% by the items above, e.g. related proposals to other telescopes, dates to 
% be avoided for the observations, etc. Do not use if not necessary.  
%
%%  Please type in the { } field below:
%% 
\addinfo{This proposal is part of the Observational Technique II course at Stockholm University.}
%%%%%%%%%%%%%%%%%%%%%%%%%%%%%%%%%%%%%%%%%%%%%%%%%%%%%%%%%%%%%%%%%%%%%%%%%%%%%
\end{extrapage}


\end{document}
%%%%%%%%%%%%%%%%%%%%%%%%%%%%%%%%%%%%%%%%%%%%%%%%%%%%%%%%%%%%%%%%%%%%%%%%%%%%%
% 
%                       End of template file
% 
%%%%%%%%%%%%%%%%%%%%%%%%%%%%%%%%%%%%%%%%%%%%%%%%%%%%%%%%%%%%%%%%%%%%%%%%%%%%%

